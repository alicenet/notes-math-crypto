\chapter{Pairing-Based Cryptography}
\label{chap:pairing}

\Gls{ethereum}~\cite{EthereumYellowpaper} has a \gls{bilinear}
included in its precompiled contracts
with the ability to perform a pairing check, elliptic curve addition,
and elliptic curve point multiplication.
This makes it possible to compute BLS signatures~\cite{BLSSignatures}.
In this chapter we will talk about BLS signatures
and what else can be accomplished with a \gls{bilinear}.
All of these operations may be performed within Solidity
\glspl{smart contract}.
On the \gls{ethereum} blockchain, \glspl{smart contract} are programs
which may automatically perform actions in response
to changes in state.
As a brief aside, we begin by discussing cryptography in \gls{ethereum} in 
Chapter~\ref{sec:pairing_ethereum}.

Throughout this chapter, we are assuming the \gls{bilinear}
is over \glspl{elliptic curve}.



\section{Cryptography in Ethereum}
\label{sec:pairing_ethereum}

We take a brief digression to spend some time talking about
\gls{ethereum}~\cite{EthereumYellowpaper}.
The main focus here will be on the cryptographic protocols
available within \gls{ethereum}.

\Gls{ethereum} is a blockchain that allows for \glspl{smart contract}.
A \emph{\gls{smart contract}} is a program which is able to perform
operations in response to inputs.
These operations include the ability to perform signature verification
and other cryptographic operations such as hashing.
Other arithmetical operations are available as well.

In particular, \gls{ethereum} has the ability to cheaply perform
certain operations involving \gls{pairingcrypto}.
One available type of \gls{bilinear} is based on a
Barreto--Naehrig Curve~\cite{BNCurves2005}.
In particular, let $e:G_{1}\times G_{2}\to G_{T}$ be the \gls{bilinear}.
We can cheaply perform addition and scalar multiplication in $G_{1}$.
We also have the ability to perform a pairing check.
Given $a_{i}\in G_{1}$ and $b_{i}\in G_{2}$ for $i\in\braces{1,\cdots,\ell}$,
we define

\begin{equation}
    \PairingCheck{}\parens{a_{1}, b_{1}, a_{2}, b_{2}, \cdots,
            a_{\ell},b_{\ell}}
        \mathDef{} \begin{cases}
            \texttt{true} \quad &\text{if }1 =\prod_{k=1}^{\ell}e(a_{k},b_{k})\\
            \texttt{false} \quad &\text{otherwise}
        \end{cases}.
\end{equation}

\noindent
When we restrict ourselves to the case when $\ell=2$, we see that

\begin{equation}
    \PairingCheck{}\parens{a_{1}, b_{1}, a_{2}, b_{2}} = \texttt{true}
\end{equation}

\noindent
when

\begin{equation}
    e(a_{1},b_{1}) \cdot e(a_{2},b_{2}) = 1
\end{equation}

\noindent
or

\begin{equation}
    e(a_{1},b_{1}^{-1}) = e(a_{2},b_{2}).
\end{equation}

\noindent
This is the form that will be used when performing
BLS signature validation;
we discuss BLS signatures in Chapter~\ref{sec:pairing_bls}.

At this time, no inexpensive operations in $G_{2}$ are available
in \gls{ethereum}~\cite{EthereumYellowpaper}.
Thus, in order to ensure valid information in $G_{2}$,
we will need to submit the corresponding information
in $G_{1}$ and then validate it using a \PairingCheck{}.



\section{BLS Signatures}
\label{sec:pairing_bls}

We previously talked about \glspl{signature} in Chapters~\ref{chap:signatures}
and \ref{chap:elliptic}.
We now talk about \glspl{signature} which arise from \glspl{bilinear}
and discuss their importance.
This builds on our discussion of \glspl{bilinear}
from Chapter~\ref{sec:math_pairings}.

Let us suppose $e:G_{1}\times G_{2}\to G_{T}$ is our \gls{bilinear}
with $\abs{G_{1}} = \abs{G_{2}} = \abs{G_{T}} = q$ for $q$ prime.
Let $g_{2}\in G_{2}$ be a generator.
Let $x \stackrel{\$}{\leftarrow} \Z_{q}$ be the private key
and $g_{2}^{x}$ be the public key.
Let us suppose that $H:\braces{0,1}^{*}\to G_{1}$ is a \gls{hash function}.
In this setting, $H$ is a hash-to-curve function,
as we map arbitrary bit strings to elements of $G_{1}$.
Given a message $m\in\braces{0,1}^{*}$, we have the BLS signature

\begin{equation}
    \sigma \mathDef{} \brackets{H(m)}^{x}.
\end{equation}

\noindent
To verify this signature, we check that

\begin{equation}
    e\parens{\sigma, g_{2}} \stackrel{?}{=} e\parens{H(m), g_{2}^{x}}.
\end{equation}

\noindent
This is equivalent to

\begin{equation}
    e\parens{\sigma, g_{2}^{-1}} \cdot e\parens{H(m), g_{2}^{x}}
        \stackrel{?}{=} 1.
\end{equation}

\noindent
This can be computed by performing a \PairingCheck{}:

\begin{equation}
    \PairingCheck{}(\sigma, g_{2}^{-1}, H(m), g_{2}^{x})
        \stackrel{?}{=} \texttt{true}.
\end{equation}

\noindent
As mentioned previously, this operation may be performed
cheaply on \gls{ethereum}~\cite{EthereumYellowpaper}.

This signature scheme was originally referred to as a \emph{short signature}
due to the size of $\sigma$; see the original publication~\cite{BLSSignatures}.
When using the \gls{bilinear} on \gls{ethereum},
BLS signatures are 64 bytes uncompressed and 33 bytes compressed;
this is smaller than any of the other signatures mentioned in
Chapter~\ref{sec:signatures_comparison}.
The public keys are also 128 bytes uncompressed.



\section{Hash-to-Curve Functions}

For BLS signatures, we require a method of hashing arbitrary messages
into \glspl{cyclic group}.
In the previous section, we \emph{assumed} such functions exist.
Now we describe how they are formed.
We want to map into an \gls{elliptic curve} $E/\F_{p}$.
Furthermore, we will assume throughout that we have access to a standard
\gls{hash function} $H:\braces{0,1}^{*}\to\braces{0,1}^{b}$.
In practice, $H$ will likely be \ShaTwo{}, \ShaThree{},
or \Keccak{}.

In this section, we switch from multiplicative notation
to additive notation for \glspl{group}.

\subsection{One Insecure Hash-to-Curve Function}

Constructing a hash-to-curve function is nontrivial;
we now present a method that should \emph{never} be used because
it breaks security assumptions.

Suppose we have a public key $x\cdot P$
and standard \gls{hash function} $H:\braces{0,1}^{*}\to\braces{0,1}^{b}$.
There are some who may want to use the hash-to-curve function

\begin{equation}
    H_{2C}(m) \mathDef{} H(m)\cdot P.
\end{equation}

\noindent
In this way, for Alice with public key $a\cdot P$, a signature would be

\begin{align}
    \sigma &\mathDef{} a\cdot\parens{H(m)\cdot P} \nonumber\\
        &= \parens{a\cdot H(m)}\cdot P.
\end{align}

\noindent
This is \emph{horrible} and completely breaks all security,
because we do not even need to know the private key to construct this signature:
all that is needed is the public key $x\cdot P$:

\begin{align}
    H(m)\cdot\parens{x\cdot P} &= \parens{H(m)\cdot x}\cdot P
            \nonumber\\
        &= \parens{x\cdot H(m)}\cdot P.
\end{align}

\noindent
\emph{This} is a clear example of why care is required when designing
cryptographic protocols.
Using this method, it is possible for forge arbitrary signatures
for any public key.

\subsection{Nondeterministic Hash-to-Curve Algorithm}

The first hash-to-curve algorithms used a nondeterministic method;
it was called \textsc{MapToGroup} in~\cite{BLSSignatures}.
The method is simple in practice, but deterministic methods are preferred
because the number of computational operations may be bounded.
We will not discuss nondeterministic methods any further.

\subsection{Initial Hash-to-Curve Discussion}

There does not appear to be a good way to map an arbitrary bit string
into a \gls{group}.
To simplify the matter, we use a two-step process:
first, we hash arbitrary bit strings to a \gls{finite field};
next, map from the \gls{finite field} into the \gls{elliptic curve}.
This two-step process is now standard;
see~\cite{brier2010efficient,fouque2012indifferentiable,BonehWahby2019}.
In the end, we have a \gls{hash function} $H:\braces{0,1}^{*}\to G$
with

\begin{equation}
    H(m) = f(H_{1}(m)) + f(H_{2}(m)).
    \label{eq:pairing_full_hash}
\end{equation}

\noindent
In this case, $H_{i}:\braces{0,1}^{*}\to\F$ is a \gls{hash function}
to a \gls{finite field} while $f:\F\to G$ is a deterministic mapping
from a \gls{finite field} $\F$ to the \gls{elliptic curve} $G$.
Furthermore, $H_{1}$ and $H_{2}$ are independent \glspl{hash function}.

\subsection{Hashing to Finite Fields}

We first focus on mapping deterministically into $\F_{p}$.
This is an easy task.
The main challenge will be to control the nonuniformity.

Suppose we have a \gls{random oracle} $\widetilde{H}:\braces{0,1}^{*}\to\Z_{N}$.
We are interested in knowing how much $H(m) \mod p$
deviates from the uniform distribution.
It turns out that the deviations are small provided $N$ is large enough.
In particular, if $p$ is an $n$-bit prime number
and we want $k$ bits of security,
then we need $N\ge2^{n+k}$.
By choosing $N = 2^{m}$ for some $m$,
we can instantiate the \gls{random oracle} $\widetilde{H}$
using a standard \gls{hash function}.
The details are worked out in Appendix~\ref{app:crypto_hash-to-finite-field}.

Overall, it is relatively easy ensure the nonuniformity is negligible.
This describes the \glspl{hash function} $H_{1}$ and $H_{2}$
in Eq.~\eqref{eq:pairing_full_hash}.

\subsection{Deterministic Mapping from Finite Field to Elliptic Curve}

Deterministically mapping from a \gls{finite field} to an \gls{elliptic curve}
is much more challenging.
The exact method strongly depends on the specific type of \gls{elliptic curve}.
The good news is that there are generic mappings for a wide class
of \glspl{elliptic curve};
see~\cite{brier2010efficient,fouque2012indifferentiable,BonehWahby2019}.
This deterministic mapping is $f$
in Eq.~\eqref{eq:pairing_full_hash}.

\subsection{Ensuring Hash-to-Curve Uniformity}

At this point, we can deterministically map arbitrary data
into an \gls{elliptic curve}.
Hash function values should be uniformly distributed across
the entire range, though.

The previous work does not ensure this property.
It may be the case that the deterministic mapping only maps into
a portion of the \gls{elliptic curve};
in this way, not every point on the \gls{elliptic curve} may be reached.
This results in a nonuniform \gls{hash function}.
Thus, ensuring uniformity requires additional care.
The details are worked on in
in~\cite{brier2010efficient,fouque2012indifferentiable,BonehWahby2019}.
In the cases from the papers previously cited,
it turns out that all that is required are 2 independent
realizations of different \gls{finite field} points mapping
to the \gls{elliptic curve}.
This is why we require $H_{1}$ and $H_{2}$ to be independent
\glspl{hash function}
and then add their deterministic outputs in Eq.~\eqref{eq:pairing_full_hash}.

\subsection{Standardization}

The difficulty in ensuring proper hash-to-curve functions
has lead to a standardization effort;
one example is~\cite{rfc9380}.



\section{Multi-Signatures}

As previously discussed in Chapter~\ref{sec:pairing_bls},
BLS signatures allow for efficient signature generation and validation.
We will now focus on $n$-out-of-$n$ multi-signature schemes.
That is, $n$ participants want to sign the same message,
and we want to prove that all of them signed it.
Threshold signature schemes
are more involved and will be discussed in Chapter~\ref{sec:ss_threshold}
after working through secret sharing and \gls{distributed key generation}
protocols.
In a threshold signature scheme,
only a \emph{subset} of participants are required to sign
a message for it to be valid for the entire group.

Let us suppose that Alice and Bob wish to sign a message together.
As in Chapter~\ref{sec:pairing_bls}, we are assuming that we are working
with BLS signatures with a \gls{bilinear} $e$
and hash-to-curve function $H$.
We suppose that Alice has private key $x_{A}$ and
public key $A = g_{2}^{x_{A}}$;
Bob has private and public keys $x_{B}$ and $B = g_{2}^{x_{B}}$.

Let us suppose that Alice and Bob both sign message $m\in\braces{0,1}^{*}$
with signatures $\sigma_{A}$ and $\sigma_{B}$.
In this case, we know that

\begin{align}
    e\parens{\sigma_{A}, g_{2}} &= e\parens{H(m), g_{2}^{x_{A}}}
        \nonumber\\
    e\parens{\sigma_{B}, g_{2}} &= e\parens{H(m), g_{2}^{x_{B}}}.
\end{align}

Suppose we set

\begin{equation}
    \sigma = \sigma_{A}\sigma_{B}.
\end{equation}

\noindent
In this case, we know

\begin{align}
    e\parens{\sigma, g_{2}} &= e\parens{\sigma_{A}\sigma_{B}, g_{2}}
            \nonumber\\
        &= e\parens{\sigma_{A}, g_{2}}\cdot e\parens{\sigma_{B}, g_{2}}
            \nonumber\\
        &= e\parens{H(m), g_{2}^{x_{A}}}\cdot e\parens{H(m), g_{2}^{x_{B}}}
            \nonumber\\
        &= e\parens{H(m), g_{2}}^{x_{A}}\cdot e\parens{H(m), g_{2}}^{x_{B}}
            \nonumber\\
        &= e\parens{H(m), g_{2}}^{x_{A}+x_{B}}
            \nonumber\\
        &= e\parens{H(m), g_{2}^{x_{A}+x_{B}}}
            \nonumber\\
        &= e\parens{H(m), A\cdot B}.
\end{align}

\noindent
Thus, in order to prove that both Alice and Bob signed the message $m$,
we only need to combine the signatures and the public keys.
This reduces the cost from 2 signature verifications
to 1 signature verification.

The same property holds in general.
In particular, suppose we have private keys $\braces{x_{k}}_{k=1}^{n}$
and public keys $\braces{g_{2}^{x_{k}}}_{k=1}^{n}$.
If each participant signs the same message $m$ producing
signatures $\braces{\sigma_{k}}_{k=1}^{n}$,
we set

\begin{align}
    \sigma &= \prod_{k=1}^{n} \sigma_{k}
        \nonumber\\
    X &= \prod_{k=1}^{n} g_{2}^{x_{k}}.
\end{align}

\noindent
We then see

\begin{align}
    e\parens{\sigma, g_{2}} &= e\parens{\prod_{k=1}^{n}\sigma_{k}, g_{2}}
            \nonumber\\
        &= \prod_{k=1}^{n} e\parens{\sigma_{k}, g_{2}}
            \nonumber\\
        &= \prod_{k=1}^{n} e\parens{H(m), g_{2}^{x_{k}}}
            \nonumber\\
        &= \prod_{k=1}^{n} e\parens{H(m), g_{2}}^{x_{k}}
            \nonumber\\
        &= e\parens{H(m), g_{2}}^{\sum_{k=1}^{n}x_{k}}
            \nonumber\\
        &= e\parens{H(m), g_{2}^{\sum_{k=1}^{n}x_{k}}}
            \nonumber\\
        &= e\parens{H(m), X}.
\end{align}

\noindent
As we can see, the $n$-out-of-$n$ case is the same as the 2-out-of-2
case with Alice and Bob.
In the $n$-out-of-$n$ case, we reduce the computation from $n$
signature verifications
to 1 signature verification at the cost of aggregating $n$ signatures
and public keys.
Because signature verification is expensive,
this is generally worthwhile.



\section{BBS Signatures}
\label{sec:pairing_bbs}

Another signature scheme which uses a \gls{bilinear}
is the BBS signature scheme.
The scheme is specified in Alg.~\ref{alg:bbs};
this specification is based on~\cite[Figure 3]{cryptoeprint:2023/275}
(also see~\cite{tessaro2023revisiting}).
The original description may be found
in~\cite{boneh2004short,cryptoeprint:2004/174},
where the focus is on short group signatures.

\begin{algorithm}[t]
\caption{BBS Signature Scheme}
\label{alg:bbs}
\begin{algorithmic}[1]
\Require $h_{i}$ is a random generator for $G_{1}$
\Require $\abs{G_{1}} = p$ prime
\Require $m_{i}\in \Z_{p}$
\Procedure{BBS-Sign}{$x$, $\braces{m_{i}}_{i=1}^{n}$}
    \State $C \algAssign{} g_{1} \prod_{i=1}^{n}h_{i}^{m_{i}}$
    \State $e \chooseRandom{} \Z_{p}$
    \State $A \algAssign{} C^{\frac{1}{x+e}}$
    \State $\sigma \algAssign{} \parens{A,e}$
    \State \Return $\sigma$
\EndProcedure
\State
\Procedure{BBS-Verify}{$X$, $\braces{m_{i}}_{i=1}^{n}$, $\sigma$}
    \State $\parens{A,e} \algAssign{} \sigma$
    \State $C \algAssign{} g_{1} \prod_{i=1}^{n}h_{i}^{m_{i}}$
    \If{$e(A, g_{2}^{e}X) = e(C, g_{2})$}
        \State \Return \textsf{True}
    \Else
        \State \Return \textsf{False}
    \EndIf
\EndProcedure
\end{algorithmic}
\end{algorithm}


As before, we let $x\chooseRandom{}\Z_{p}^{*}$ be the private key
and $X\mathDef{} g_{2}^{x}$ be the public key.
Here, we will look at the case where there is just one message $m$ to sign;
we let $h$ be a random generator of $G_{1}$.
In this case, we have

\begin{align}
    C &\mathDef{} g_{1}h^{m} \nonumber\\
    A &\mathDef{} C^{\frac{1}{x + e}}
\end{align}

\noindent
for $e\chooseRandom{} \Z_{p}$.
From here, we see

\begin{align}
    e\parens{A, g_{2}^{e}X} &= e\parens{C^{\frac{1}{x+e}}, g_{2}^{x+e}}
            \nonumber\\
        &= e\parens{C, g_{2}}.
\end{align}

\noindent
Thus, the signature will validate.
The general case is similar.

A discussion on proofs of knowledge of BBS signatures may be found in
Chapter~\ref{sec:zkproofs_bbs};
this is valuable when looking to develop
\emph{anonymous credentials}
(see \cite[Introduction]{cryptoeprint:2023/275} and the references therein).



\section{Concluding Discussion}

We have only scratched the surface of \gls{pairingcrypto}.
Additional material may be found in~\cite{PairingBasedCrypto}.

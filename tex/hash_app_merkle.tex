\section{Merkle Trees I}
\label{sec:merkle_trees}

Another useful application of \glspl{hash function} are
\emph{\glspl{merkle tree}}.
In this section, we will focus on \glspl{merkle tree} involving
leaves which are a power-of-two.
We will look at trees with an arbitrary number of leaves in
Chap.~\ref{sec:merkle_trees_2}.
In Chap.~\ref{sec:sparse_merkle_trees},
we will look at \emph{\glspl{smt}}.

\subsection{Intuition}

It is challenging to send large amounts of data at one time.
It would be convenient to chop the data into small pieces
and then send those small pieces of data separately.
The challenge is then to ensure the validity of each small
piece of data.

A \emph{\gls{merkle tree}} can help solve this problem.
It uses one cryptographic hash value to represent the \emph{entire}
collection of data.
At the same time, individual blocks of data can be retrieved
along with its associated \emph{Merkle Proof}:
a proof which shows the individual data block is valid.

\subsection{Definition}

The algorithm defining the \gls{merkle tree} root hash can be found in
Alg.~\ref{alg:merkle_tree};
see Alg.~\ref{alg:merkle_proof} for the proof of inclusion.
These definitions are modified
from~\cite[Chapter 8.9]{BonehShoupGraduateApplied}.
This is valid only for trees with leaves
which are a power-of-two;
as mentioned, we will discuss the situation involving
an arbitrary number of leaves in Chap.~\ref{sec:merkle_trees_2}.

\begin{algorithm}[p]
\caption{Compute Merkle Tree root hash}
\label{alg:merkle_tree}
\begin{algorithmic}[1]
\Require $n = 2^{k}$ for some $k\ge1$; $H$ is a \gls{hash function}.
\Procedure{MerkleTreeRoot}{$x_{0}, x_{1}, \cdots, x_{n-1}$}
    \For {$i=0; i < n; i{+}{+}$}
        \State $y_{i} \algAssign{} H(x_{i})$
        \Comment{Compute leaf hashes}
    \EndFor
    \For {$i=0; i < n-1; i{+}{+}$}
        \State $y_{n+i} \algAssign{} H(y_{2i}\|y_{2i+1})$
        \Comment{Compute internal hashes}
    \EndFor
    \State \Return $y_{2n-2}$
    \Comment{Return root hash}
\EndProcedure
\end{algorithmic}
\end{algorithm}

\begin{algorithm}[p]
\caption{Validate Merkle Proof of Inclusion}
\label{alg:merkle_proof}
\begin{algorithmic}[1]
\Require $n = 2^{k}$ for some $k\ge1$; $H$ is a \gls{hash function}; $\text{len}(\pi) = \log_{2}(n)$.
\Procedure{ValidateMerkleProof}{$x, i, y_{2n-2}, \pi$}
    \State $\hat{y}_{i} \algAssign{} H(x)$
    \Comment{Compute leaf hash}
    \State $\mu \algAssign{} i$
    \For {$k=0; k < \log_{2}(n); k{+}{+}$}
        \If {$\mu$ odd}
            \State $\ell \algAssign{} (\mu-1)/2$
            \State $\nu \algAssign{} n + \ell$
            \State $\hat{y}_{\nu} \algAssign{} H(\pi_{k}\|\hat{y}_{\mu})$
        \Else
            \State $\ell \algAssign{} \mu/2$
            \State $\nu \algAssign{} n + \ell$
            \State $\hat{y}_{\nu} \algAssign{} H(\hat{y}_{\mu}\|\pi_{k})$
        \EndIf
        \State $\mu \algAssign{} \nu$
    \EndFor
    \If {$\hat{y}_{2n-2} = y_{2n-2}$}
        \State \Return \texttt{true}
    \Else
        \State \Return \texttt{false}
    \EndIf
\EndProcedure
\end{algorithmic}
\end{algorithm}


\subsection{Continued Discussion}

As mentioned above, there is an associated proof of inclusion
for each piece of data.
These individual proofs are inexpensive.
It is possible to perform multiple inclusion-proofs at once;
this is a called a \emph{Merkle Multi-Proof}.
We will not give a formal definition of a multi-proof
but see Example~\ref{example:hash_app_merkle_tree_multi-proof}
for an example.

We note the following details which are useful
but which will not be discussed any further here.
In addition to proofs of inclusion,
it is also possible to have \emph{proofs of exclusion};
doing this requires additional assumptions on the
input data~\cite{EfficientSMT}.
Furthermore, some implementations may use prefixes when hashing;
the purpose of this is to provide
second-preimage resistance~\cite{rfc6962,rfc9162}.
The need for this will be shown in
Example~\ref{example:hash_app_merkle_tree_security}.

\subsection{Examples}

\begin{example}[2-Layer Merkle Tree]
\exampleCodeReference{examples/hash\_applications/merkle\_tree\_md5\_2.py}

We begin by starting with an example involving 4 blocks of data;
see the \gls{merkle tree} in Figure~\ref{fig:merkle_tree_2}
for the example.

\begin{figure}[p]
\centering
\begin{subfigure}{\textwidth}
    \centering
    \begin{tikzpicture}
        [level 1/.style={sibling distance=40mm},
         level 2/.style={sibling distance=20mm}]
        \node[circle,draw] (root) {$H$}
            child {node[circle,draw] {$H$}
                child {node[circle,draw] {$H$}
                    child {node[rectangle,draw] {$x_{1}$}
                    edge from parent[<-,thick]}
                edge from parent[<-,thick]
                    node[left] {$y_{1}$}
                }
                child {node[circle,draw] {$H$}
                    child {node[rectangle,draw] {$x_{2}$}
                    edge from parent[<-,thick]}
                edge from parent[<-,thick]
                    node[right] {$y_{2}$}
                }
            edge from parent[<-,thick]
                node[left] {$y_{5}$}
            }
            child {node[circle,draw] {$H$}
                child {node[circle,draw] {$H$}
                    child {node[rectangle,draw] {$x_{3}$}
                    edge from parent[<-,thick]}
                edge from parent[<-,thick]
                    node[left] {$y_{3}$}
                }
                child {node[circle,draw] {$H$}
                    child {node[rectangle,draw] {$x_{4}$}
                    edge from parent[<-,thick]}
                edge from parent[<-,thick]
                    node[right] {$y_{4}$}
                }
            edge from parent[<-,thick]
                node[right] {$y_{6}$}
            };
        \node[above=of root] (header) {};
        \draw[->,thick] (root) to node[auto,swap] {$y_{7}$} (header);
    \end{tikzpicture}
    \caption{2-layer Merkle Tree}
    \label{fig:merkle_tree_2}
\end{subfigure}

\begin{subfigure}{\textwidth}
    \centering
    \begin{tikzpicture}
        [level 1/.style={sibling distance=40mm},
         level 2/.style={sibling distance=20mm}]
        \node[circle,draw] (root) {$H$}
            child {node[circle,draw] {$H$}
                child {node[circle,draw] {$H$}
                    child {node[rectangle,draw] {$x_{1}$}
                    edge from parent[<-,thick]}
                edge from parent[<-,thick]
                    node[left] {$y_{1}$}
                }
                child {node[circle,draw] {$H$}
                    child {node[rectangle,draw] {$x_{2}$}
                    edge from parent[<-,thick]}
                edge from parent[<-,thick]
                    node[right] {$y_{2}$}
                }
            edge from parent[<-,thick]
                node[left,red] {$y_{5}$}
            }
            child {node[circle,draw] {$H$}
                child {node[circle,draw] {$H$}
                    child {node[rectangle,draw,very thick, blue] {$x_{3}$}
                    edge from parent[<-,thick]}
                edge from parent[<-,thick]
                    node[left] {$y_{3}$}
                }
                child {node[circle,draw] {$H$}
                    child {node[rectangle,draw] {$x_{4}$}
                    edge from parent[<-,thick]}
                edge from parent[<-,thick]
                    node[right,red] {$y_{4}$}
                }
            edge from parent[<-,thick]
                node[right] {$y_{6}$}
            };
        \node[above=of root] (header) {};
        \draw[->,thick] (root) to node[auto,swap] {$y_{7}$} (header);
    \end{tikzpicture}
    \caption{2-layer Merkle Tree with Merkle Proof of $x_{3}$.}
    \label{fig:merkle_tree_2_proof}
\end{subfigure}
\caption[2-Layer Merkle Tree and Merkle Proof]{Here
    we have a 2-layer Merkle Tree.
    There is also a Merkle Proof for $x_{3}$.}
\end{figure}


In this case, we see that we have the following:

\begin{align}
    y_{0} &= H(x_{0})
        &
    y_{4} &= H(y_{0}\|y_{1}) \nonumber\\
    y_{1} &= H(x_{1})
        &
    y_{5} &= H(y_{2}\|y_{3}) \nonumber\\
    y_{2} &= H(x_{2})
        &
    y_{6} &= H(y_{4}\|y_{5}). \nonumber\\
    y_{3} &= H(x_{3})
    \label{eq:merkle_tree_2_def}
\end{align}

\noindent
In this case, $y_{6}$ is our root hash.

To make this more practical, we will work through an example
using \MDFive{}; see Listing~\ref{list:merkle_tree_md5_2}.
We have the data

\begin{align}
    x_{0} &= \texttt{00000000000000000000000000000000}
        \nonumber\\
    x_{1} &= \texttt{11111111111111111111111111111111}
        \nonumber\\
    x_{2} &= \texttt{22222222222222222222222222222222}
        \nonumber\\
    x_{3} &= \texttt{33333333333333333333333333333333}.
\end{align}

\lstset{
    showstringspaces=false,
    frame=single
}
\lstinputlisting[
    label=list:merkle_tree_md5_2,
    float,
    language=Python,
    caption={2-Layer Merkle Tree using \MDFive{}}]{code/hash_applications/merkle_tree_md5_2.py}


\noindent
After working through all the hashing in Eq.~\eqref{eq:merkle_tree_2_def},
we have the root hash

\begin{equation}
    y_{6} = \texttt{40c0b71ca488d334d266beecac02a16c}.
\end{equation}
\end{example}

\begin{example}[2-Layer Merkle Tree Proof]
\exampleCodeReference{examples/hash\_applications/merkle\_tree\_md5\_2\_proof.py}

We will now look at Merkle Proof showing that $x_{2}$
is a member of this tree at position 2;
see Figure~\ref{fig:merkle_tree_2_proof}
and Listing~\ref{list:merkle_tree_md5_proof}.
Our data and proof is

\lstset{
    showstringspaces=false,
    frame=single
}
\lstinputlisting[
    label=list:merkle_tree_md5_proof,
    float,
    language=Python,
    caption={2-Layer Merkle Proof using \MDFive{}}]{code/hash_applications/merkle_tree_md5_2_proof.py}


\begin{align}
    x_{2} &= \texttt{22222222222222222222222222222222}
        \nonumber\\
    y_{3} &= \texttt{28bfcf057ec5a48f18c3154c1f2bd324}
        \nonumber\\
    y_{4} &= \texttt{b05df6fba6c1c53e8ddb98ffd386ffc8}.
\end{align}

\noindent
Using these values, we see

\begin{align}
    \hat{y}_{2} &= H(x_{2}) \nonumber\\
        &= \texttt{fbc3cf71d993ca7bec2664357ccdac2b}
            \nonumber\\
    \hat{y}_{5} &= H(\hat{y}_{2}, y_{3}) \nonumber\\
        &= \texttt{8f7a2a2dcd297872689e177953270f37}
            \nonumber\\
    \hat{y}_{6} &= H(y_{4}, \hat{y}_{5}) \nonumber\\
        &= \texttt{40c0b71ca488d334d266beecac02a16c}.
\end{align}

\noindent
Because $\hat{y}_{6} = y_{6}$, we have shown that $x_{2}$ is the data
at position 2 in our \gls{merkle tree}.
This completes the proof;
it required computing 3 hashes.
\end{example}

\begin{example}[3-Layer Merkle Tree]
\exampleCodeReference{examples/hash\_applications/merkle\_tree\_md5\_3.py}

A 3-layer \gls{merkle tree} can be found in Figure~\ref{fig:merkle_tree_3};
also see Listing~\ref{list:merkle_tree_md5_3}.
In this case, we see that we have the following:

\begin{figure}
\centering
\begin{subfigure}{\textwidth}
    \centering
    \begin{tikzpicture}
        [level 1/.style={sibling distance=80mm},
         level 2/.style={sibling distance=40mm},
         level 3/.style={sibling distance=20mm}]
        \node[circle,draw] (root) {$H$}
            child {node[circle,draw] {$H$}
                child {node[circle,draw] {$H$}
                    child {node[circle,draw] {$H$}
                        child {node[rectangle,draw] {$x_{1}$}
                        edge from parent[<-,thick]}
                    edge from parent[<-,thick]
                        node[left] {$y_{1}$}
                    }
                    child {node[circle,draw] {$H$}
                        child {node[rectangle,draw] {$x_{2}$}
                        edge from parent[<-,thick]}
                    edge from parent[<-,thick]
                        node[right] {$y_{2}$}
                    }
                edge from parent[<-,thick]
                    node[left] {$y_{9}$}
                }
                child {node[circle,draw] {$H$}
                    child {node[circle,draw] {$H$}
                        child {node[rectangle,draw] {$x_{3}$}
                        edge from parent[<-,thick]}
                    edge from parent[<-,thick]
                        node[left] {$y_{3}$}
                    }
                    child {node[circle,draw] {$H$}
                        child {node[rectangle,draw] {$x_{4}$}
                        edge from parent[<-,thick]}
                    edge from parent[<-,thick]
                        node[right] {$y_{4}$}
                    }
                edge from parent[<-,thick]
                    node[right] {$y_{10}$}
                }
            edge from parent[<-,thick]
                node[left] {$y_{13}$}
            }
            child {node[circle,draw] {$H$}
                child {node[circle,draw] {$H$}
                    child {node[circle,draw] {$H$}
                        child {node[rectangle,draw] {$x_{5}$}}
                    edge from parent[<-,thick]
                        node[left] {$y_{5}$}
                    }
                    child {node[circle,draw] {$H$}
                        child {node[rectangle,draw] {$x_{6}$}}
                    edge from parent[<-,thick]
                        node[right] {$y_{6}$}
                    }
                edge from parent[<-,thick]
                    node[left] {$y_{11}$}
                }
                child {node[circle,draw] {$H$}
                    child {node[circle,draw] {$H$}
                        child {node[rectangle,draw] {$x_{7}$}
                        edge from parent[<-,thick]}
                    edge from parent[<-,thick]
                        node[left] {$y_{7}$}
                    }
                    child {node[circle,draw] {$H$}
                        child {node[rectangle,draw] {$x_{8}$}
                        edge from parent[<-,thick]}
                    edge from parent[<-,thick]
                        node[right] {$y_{8}$}
                    }
                edge from parent[<-,thick]
                    node[right] {$y_{12}$}
                }
            edge from parent[<-,thick]
                node[right] {$y_{14}$}
            };
        \node[above=of root] (header) {};
        \draw[->,thick] (root) to node[auto,swap] {$y_{15}$} (header);
    \end{tikzpicture}
    \caption{3-layer Merkle Tree}
    \label{fig:merkle_tree_3}
\end{subfigure}

\begin{subfigure}{\textwidth}
    \centering
    \begin{tikzpicture}
        [level 1/.style={sibling distance=80mm},
         level 2/.style={sibling distance=40mm},
         level 3/.style={sibling distance=20mm}]
        \node[circle,draw] (root) {$H$}
            child {node[circle,draw] {$H$}
                child {node[circle,draw] {$H$}
                    child {node[circle,draw] {$H$}
                        child {node[rectangle,draw] {$x_{1}$}
                        edge from parent[<-,thick]}
                    edge from parent[<-,thick]
                        node[left] {$y_{1}$}
                    }
                    child {node[circle,draw] {$H$}
                        child {node[rectangle,draw] {$x_{2}$}
                        edge from parent[<-,thick]}
                    edge from parent[<-,thick]
                        node[right] {$y_{2}$}
                    }
                edge from parent[<-,thick]
                    node[left] {$y_{9}$}
                }
                child {node[circle,draw] {$H$}
                    child {node[circle,draw] {$H$}
                        child {node[rectangle,draw] {$x_{3}$}
                        edge from parent[<-,thick]}
                    edge from parent[<-,thick]
                        node[left] {$y_{3}$}
                    }
                    child {node[circle,draw] {$H$}
                        child {node[rectangle,draw] {$x_{4}$}
                        edge from parent[<-,thick]}
                    edge from parent[<-,thick]
                        node[right] {$y_{4}$}
                    }
                edge from parent[<-,thick]
                    node[right] {$y_{10}$}
                }
            edge from parent[<-,thick]
                node[left,red] {$y_{13}$}
            }
            child {node[circle,draw] {$H$}
                child {node[circle,draw] {$H$}
                    child {node[circle,draw] {$H$}
                        child {node[rectangle,draw,very thick,blue] {$x_{5}$}}
                    edge from parent[<-,thick]
                        node[left] {$y_{5}$}
                    }
                    child {node[circle,draw] {$H$}
                        child {node[rectangle,draw] {$x_{6}$}}
                    edge from parent[<-,thick]
                        node[right,red] {$y_{6}$}
                    }
                edge from parent[<-,thick]
                    node[left] {$y_{11}$}
                }
                child {node[circle,draw] {$H$}
                    child {node[circle,draw] {$H$}
                        child {node[rectangle,draw] {$x_{7}$}
                        edge from parent[<-,thick]}
                    edge from parent[<-,thick]
                        node[left] {$y_{7}$}
                    }
                    child {node[circle,draw] {$H$}
                        child {node[rectangle,draw] {$x_{8}$}
                        edge from parent[<-,thick]}
                    edge from parent[<-,thick]
                        node[right] {$y_{8}$}
                    }
                edge from parent[<-,thick]
                    node[right,red] {$y_{12}$}
                }
            edge from parent[<-,thick]
                node[right] {$y_{14}$}
            };
        \node[above=of root] (header) {};
        \draw[->,thick] (root) to node[auto,swap] {$y_{15}$} (header);
    \end{tikzpicture}
    \caption{3-layer Merkle Tree with Merkle Proof of $x_{5}$}
    \label{fig:merkle_tree_3_proof}
\end{subfigure}
\caption[3-Layer Merkle Tree and Merkle Proof]{Here
    is a 3-layer Merkle Tree.
    There is also a Merkle Proof for $x_{5}$.}
\end{figure}

\lstset{
    showstringspaces=false,
    frame=single
}
\lstinputlisting[
    label=list:merkle_tree_md5_3,
    float,
    language=Python,
    caption={3-Layer Merkle Tree using \MDFive{}}]{code/hash_applications/merkle_tree_md5_3.py}


\begin{align}
    y_{0}  &= H(x_{0})
        &
    y_{8}  &= H(y_{0}\|y_{1}) \nonumber\\
    y_{1}  &= H(x_{1})
        &
    y_{9}  &= H(y_{2}\|y_{3}) \nonumber\\
    y_{2}  &= H(x_{2})
        &
    y_{10} &= H(y_{4}\|y_{5}) \nonumber\\
    y_{3}  &= H(x_{3})
        &
    y_{11} &= H(y_{6}\|y_{7}) \nonumber\\
    y_{4}  &= H(x_{4})
        &
    y_{12} &= H(y_{8}\|y_{9}) \nonumber\\
    y_{5}  &= H(x_{5})
        &
    y_{13} &= H(y_{10}\|y_{11}) \nonumber\\
    y_{6}  &= H(x_{6})
        &
    y_{14} &= H(y_{12}\|y_{13}). \nonumber\\
    y_{7}  &= H(x_{7})
    \label{eq:merkle_tree_3_def}
\end{align}

\noindent
Here, $y_{14}$ is the root hash.

For our data, we set

\begin{align}
    x_{0} &= \texttt{00000000000000000000000000000000}
        \nonumber\\
    x_{1} &= \texttt{01010101010101010101010101010101}
        \nonumber\\
    x_{2} &= \texttt{02020202020202020202020202020202}
        \nonumber\\
    x_{3} &= \texttt{03030303030303030303030303030303}
        \nonumber\\
    x_{4} &= \texttt{04040404040404040404040404040404}
        \nonumber\\
    x_{5} &= \texttt{05050505050505050505050505050505}
        \nonumber\\
    x_{6} &= \texttt{06060606060606060606060606060606}
        \nonumber\\
    x_{7} &= \texttt{07070707070707070707070707070707}.
    \label{eq:merkle_tree_3_data}
\end{align}

\noindent
We have the root hash

\begin{equation}
    y_{14} = \texttt{18446692e822581d02b096e1b77c9fff}.
    \label{eq:merkle_tree_3_root_hash}
\end{equation}
\end{example}

\begin{example}[3-Layer Merkle Tree Proof]
\exampleCodeReference{examples/hash\_applications/merkle\_tree\_md5\_3\_proof.py}

We continue our previous example to show a Merkle proof showing
$x_{4}$ is an element of our \gls{merkle tree};
see Figure~\ref{fig:merkle_tree_3_proof}
and Listing~\ref{list:merkle_tree_md5_3_proof}.
Our data and proof is

\lstset{
    showstringspaces=false,
    frame=single
}
\lstinputlisting[
    label=list:merkle_tree_md5_3_proof,
    float,
    language=Python,
    caption={3-Layer Merkle Proof using \MDFive{}}]{code/hash_applications/merkle_tree_md5_3_proof.py}


\begin{align}
    x_{4}  &= \texttt{04040404040404040404040404040404}
        \nonumber\\
    y_{5}  &= \texttt{af52282db55243f4c147ba5d7fb1155a}
        \nonumber\\
    y_{11} &= \texttt{70d0669eae8c7a5dce3b3ff4ccf4adbb}
        \nonumber\\
    y_{12} &= \texttt{a3f21dba8fa8de359220c29c00467556}.
\end{align}

\noindent
Using these values, we see

\begin{align}
    \hat{y}_{4}  &= H(x_{4}) \nonumber\\
        &= \texttt{b40ebeba833c1c07e74d9e4c6ebb8230}
            \nonumber\\
    \hat{y}_{10} &= H(\hat{y}_{4}, y_{5}) \nonumber\\
        &= \texttt{524856d86cb1006e9e9e9149f36b2875}
            \nonumber\\
    \hat{y}_{13} &= H(\hat{y}_{10}, y_{11}) \nonumber\\
        &= \texttt{da95d6882598892592ccd54bc89f7bdb
            }
            \nonumber\\
    \hat{y}_{14} &= H(y_{12}, \hat{y}_{13}) \nonumber\\
        &= \texttt{18446692e822581d02b096e1b77c9fff}.
\end{align}

\noindent
Because $\hat{y}_{14} = y_{14}$ in Eq.~\eqref{eq:merkle_tree_3_root_hash},
we have shown that $x_{4}$ is the data at position 4
in our \gls{merkle tree}.
This completes the proof;
it required computing 4 hashes.
\end{example}

\begin{example}[3-Layer Merkle Tree Multi-Proof]
\label{example:hash_app_merkle_tree_multi-proof}
\exampleCodeReference{examples/hash\_applications/merkle\_tree\_md5\_3\_multi-proof.py}

We continue the previous example working with a 3-layer \gls{merkle tree}
by showing an example of a Merkle Multi-Proof.
In this case, we want to proof the inclusion of
$x_{1}$, $x_{3}$, and $x_{6}$;
see Figure~\ref{fig:merkle_tree_3_multi-proof}
and Listing~\ref{list:merkle_tree_md5_3_multi-proof}.

\begin{figure}
\centering
\begin{tikzpicture}
    [level 1/.style={sibling distance=80mm},
     level 2/.style={sibling distance=40mm},
     level 3/.style={sibling distance=20mm}]
    \node[circle,draw] (root) {$H$}
        child {node[circle,draw] {$H$}
            child {node[circle,draw] {$H$}
                child {node[circle,draw] {$H$}
                    child {node[rectangle,draw] {$x_{0}$}
                    edge from parent[<-,thick]}
                edge from parent[<-,thick]
                    node[left,red] {$y_{0}$}
                }
                child {node[circle,draw] {$H$}
                    child {node[rectangle,draw,very thick, blue] {$x_{1}$}
                    edge from parent[<-,thick]}
                edge from parent[<-,thick]
                    node[right] {$y_{1}$}
                }
            edge from parent[<-,thick]
                node[left] {$y_{8}$}
            }
            child {node[circle,draw] {$H$}
                child {node[circle,draw] {$H$}
                    child {node[rectangle,draw] {$x_{2}$}
                    edge from parent[<-,thick]}
                edge from parent[<-,thick]
                    node[left,red] {$y_{2}$}
                }
                child {node[circle,draw] {$H$}
                    child {node[rectangle,draw,very thick, blue] {$x_{3}$}
                    edge from parent[<-,thick]}
                edge from parent[<-,thick]
                    node[right] {$y_{3}$}
                }
            edge from parent[<-,thick]
                node[right] {$y_{9}$}
            }
        edge from parent[<-,thick]
            node[left] {$y_{12}$}
        }
        child {node[circle,draw] {$H$}
            child {node[circle,draw] {$H$}
                child {node[circle,draw] {$H$}
                    child {node[rectangle,draw] {$x_{4}$}}
                edge from parent[<-,thick]
                    node[left] {$y_{4}$}
                }
                child {node[circle,draw] {$H$}
                    child {node[rectangle,draw] {$x_{5}$}}
                edge from parent[<-,thick]
                    node[right] {$y_{5}$}
                }
            edge from parent[<-,thick]
                node[left,red] {$y_{10}$}
            }
            child {node[circle,draw] {$H$}
                child {node[circle,draw] {$H$}
                    child {node[rectangle,draw,very thick, blue] {$x_{6}$}
                    edge from parent[<-,thick]}
                edge from parent[<-,thick]
                    node[left] {$y_{6}$}
                }
                child {node[circle,draw] {$H$}
                    child {node[rectangle,draw] {$x_{7}$}
                    edge from parent[<-,thick]}
                edge from parent[<-,thick]
                    node[right,red] {$y_{7}$}
                }
            edge from parent[<-,thick]
                node[right] {$y_{11}$}
            }
        edge from parent[<-,thick]
            node[right] {$y_{13}$}
        };
    \node[above=of root] (header) {};
    \draw[->,thick] (root) to node[auto,swap] {$y_{14}$} (header);
\end{tikzpicture}
\caption[3-Layer Merkle Tree and Merkle Multi-Proof]{Here
    is a 3-layer Merkle Tree with a Merkle Multi-Proof of
    $x_{1}$, $x_{3}$, and $x_{6}$.}
\label{fig:merkle_tree_3_multi-proof}
\end{figure}

\lstset{
    showstringspaces=false,
    frame=single
}
\lstinputlisting[
    label=list:merkle_tree_md5_3_multi-proof,
    float,
    language=Python,
    caption={3-Layer Merkle Multi-Proof using \MDFive{}}]{code/hash_applications/merkle_tree_md5_3_multi-proof.py}


We have the data

\begin{align}
    x_{1} &= \texttt{01010101010101010101010101010101}
        \nonumber\\
    x_{3} &= \texttt{03030303030303030303030303030303}
        \nonumber\\
    x_{6} &= \texttt{06060606060606060606060606060606}.
\end{align}

\noindent
Our proof is

\begin{align}
    y_{0}  &= \texttt{4ae71336e44bf9bf79d2752e234818a5}
        \nonumber\\
    y_{2}  &= \texttt{437b25ad27df2f61eb14c6400ae98309}
        \nonumber\\
    y_{7}  &= \texttt{a4aa9a02aa65f31d17dce9f944a57ab2}
        \nonumber\\
    y_{10} &= \texttt{524856d86cb1006e9e9e9149f36b2875}.
\end{align}

\noindent
We have the following computed values:

\begin{align}
    \hat{y}_{1} &= H(x_{1}) \nonumber\\
        &= \texttt{24311d9abc4077123c2c9a167afbe754}
            \nonumber\\
    \hat{y}_{3} &= H(x_{3}) \nonumber\\
        &= \texttt{b09931959951caa2ac110feebbd16750}
            \nonumber\\
    \hat{y}_{6} &= H(x_{6}) \nonumber\\
        &= \texttt{e5f636381c9702b63aa666ef2a6f8e20}
            \nonumber\\
    \hat{y}_{8} &= H(y_{0}\|\hat{y}_{1}) \nonumber\\
        &= \texttt{f8a9500bb41d7312abd46b0d01404fca}
            \nonumber\\
    \hat{y}_{9} &= H(y_{2}\|\hat{y}_{3}) \nonumber\\
        &= \texttt{8bfbb2226583b489fbc3cdfee7adc2c7}
            \nonumber\\
    \hat{y}_{11} &= H(\hat{y}_{6},y_{7}) \nonumber\\
        &= \texttt{70d0669eae8c7a5dce3b3ff4ccf4adbb}
            \nonumber\\
    \hat{y}_{12} &= H(\hat{y}_{8}\|\hat{y}_{9}) \nonumber\\
        &= \texttt{a3f21dba8fa8de359220c29c00467556}
            \nonumber\\
    \hat{y}_{13} &= H(y_{10}\|\hat{y}_{11}) \nonumber\\
        &= \texttt{da95d6882598892592ccd54bc89f7bdb}
            \nonumber\\
    \hat{y}_{14} &= H(\hat{y}_{12}\|\hat{y}_{13}) \nonumber\\
        &= \texttt{18446692e822581d02b096e1b77c9fff}.
\end{align}

\noindent
Because $\hat{y}_{14} = y_{14}$, we have shown that
$x_{1}$, $x_{3}$, and $x_{6}$ are valid and are at the specified positions.
This completes the proof.
This proof requires computing 9 hash values,
which is less than the 12 hash computations required to prove the inclusions
separately ($12 = 3\cdot 4$; 3 proofs involving 4 hash computations each).
\end{example}


\begin{example}[3-Layer Merkle Tree Security]
\label{example:hash_app_merkle_tree_security}
\exampleCodeReference{examples/hash\_applications/merkle\_tree\_md5\_3\_security.py}

We continue the previous example working with a 3-layer \gls{merkle tree}
by showing what can go wrong by using the standard definitions
given in Algs.~\ref{alg:merkle_tree} and \ref{alg:merkle_proof}.

In particular, we will show that the value

\begin{equation}
    x \mathDef{} y_{4}\|y_{5}
\end{equation}

\noindent
is a ``leaf node'' in the tree;
see Figure~\ref{fig:merkle_tree_3_security}
and Listing~\ref{list:merkle_tree_md5_3_security}.
The values $y_{4}$ and $y_{5}$ are derived from the data
in Eq.~\eqref{eq:merkle_tree_3_data}.

\begin{figure}
\centering
\begin{tikzpicture}
    [level 1/.style={sibling distance=80mm},
     level 2/.style={sibling distance=40mm},
     level 3/.style={sibling distance=20mm}]
    \node[circle,draw] (root) {$H$}
        child {node[circle,draw] {$H$}
            child {node[circle,draw] {$H$}
                child {node[circle,draw] {$H$}
                    child {node[rectangle,draw] {$x_{0}$}
                    edge from parent[<-,thick]}
                edge from parent[<-,thick]
                    node[left] {$y_{0}$}
                }
                child {node[circle,draw] {$H$}
                    child {node[rectangle,draw] {$x_{1}$}
                    edge from parent[<-,thick]}
                edge from parent[<-,thick]
                    node[right] {$y_{1}$}
                }
            edge from parent[<-,thick]
                node[left=3pt] {$y_{8}$}
            }
            child {node[circle,draw] {$H$}
                child {node[circle,draw] {$H$}
                    child {node[rectangle,draw] {$x_{2}$}
                    edge from parent[<-,thick]}
                edge from parent[<-,thick]
                    node[left] {$y_{2}$}
                }
                child {node[circle,draw] {$H$}
                    child {node[rectangle,draw] {$x_{3}$}
                    edge from parent[<-,thick]}
                edge from parent[<-,thick]
                    node[right] {$y_{3}$}
                }
            edge from parent[<-,thick]
                node[right=3pt] {$y_{9}$}
            }
        edge from parent[<-,thick]
            node[left=9pt,red] {$y_{12}$}
        }
        child {node[circle,draw] {$H$}
            child {node[circle,draw] {$H$}
                child {node[rectangle,draw,very thick,blue] {$x$}
                    edge from parent[<-,thick]}
            edge from parent[<-,thick]
                node[left=3pt] {$y$}
            }
            child {node[circle,draw] {$H$}
                child {node[circle,draw] {$H$}
                    child {node[rectangle,draw] {$x_{6}$}
                    edge from parent[<-,thick]}
                edge from parent[<-,thick]
                    node[left] {$y_{6}$}
                }
                child {node[circle,draw] {$H$}
                    child {node[rectangle,draw] {$x_{7}$}
                    edge from parent[<-,thick]}
                edge from parent[<-,thick]
                    node[right] {$y_{7}$}
                }
            edge from parent[<-,thick]
                node[right=3pt,red] {$y_{11}$}
            }
        edge from parent[<-,thick]
            node[right=10pt] {$y_{13}$}
        };
    \node[above=of root] (header) {};
    \draw[->,thick] (root) to node[auto,swap] {$y_{14}$} (header);
\end{tikzpicture}
\caption[3-Layer Merkle Tree and Security]{Here
    is a 3-layer \gls{merkle tree} with a Merkle Proof for showing
    that $x \mathDef{} y_{4}||y_{5}$ is a ``leaf node'' in the tree.
    This is possible if there is not difference between hashing
    leaf nodes and interior nodes in a \gls{merkle tree}.}
\label{fig:merkle_tree_3_security}
\end{figure}

\lstset{
    showstringspaces=false,
    frame=single
}
\lstinputlisting[
    label=list:merkle_tree_md5_3_security,
    float,
    language=Python,
    caption={Security of 3-Layer Merkle Proof using \MDFive{}}]{code/hash_applications/merkle_tree_md5_3_security.py}


Our data and proof is

\begin{align}
    x  &=
    \texttt{b40ebeba833c1c07e74d9e4c6ebb8230af52282db55243f4c147ba5d7fb1155a}
        \nonumber\\
    y_{11} &= \texttt{70d0669eae8c7a5dce3b3ff4ccf4adbb}
        \nonumber\\
    y_{12} &= \texttt{a3f21dba8fa8de359220c29c00467556}.
\end{align}

\noindent
Using these values, we see

\begin{align}
    \hat{y} &= H(x) \nonumber\\
        &= \texttt{524856d86cb1006e9e9e9149f36b2875}
            \nonumber\\
    \hat{y}_{13} &= H(\hat{y}, y_{11}) \nonumber\\
        &= \texttt{da95d6882598892592ccd54bc89f7bdb
            }
            \nonumber\\
    \hat{y}_{14} &= H(y_{12}, \hat{y}_{13}) \nonumber\\
        &= \texttt{18446692e822581d02b096e1b77c9fff}.
\end{align}

\noindent
Because $\hat{y}_{14} = y_{14}$ in Eq.~\eqref{eq:merkle_tree_3_root_hash},
we have ``proven'' that $x$ is a leaf in our \gls{merkle tree}.
The reason for this will be discussed more in
Chapter~\ref{ssec:mt2_security}.
\end{example}



\section{Merkle Trees II}
\label{sec:merkle_trees_2}

The previous section focused on generic \glspl{merkle tree} when
the number of leaves are a power-of-two.
We now look at the more general situation involving an
arbitrary number of leaves.
First, we start by looking more closely at \gls{merkle tree}
security concerns.

\subsection{Merkle Tree Security}
\label{ssec:mt2_security}

In Example~\ref{example:hash_app_merkle_tree_security},
we were able to ``prove'' that another data value
was present in the \gls{merkle tree}.
The fact such easy ``proofs'' exist and may be readily found
scares cryptographers
(who have a ``professional paranoia mindset''~\cite[Chapter 1.12]{CryptoEng}).
This is unacceptable;
it should be impractical to find any data
and a valid inclusion proof if that data is not present within the
\gls{merkle tree}.

This problem arises because the same hash algorithm is used
for both leaf and interior nodes.
One way to get around this is to use domain separation
within the definition of the leaf hashes and interior hashes;
this choice is used in~\cite{rfc6962,rfc9162}.
In particular, leaf hashes begin with the zero byte \texttt{0x00}
while other hashes (interior hashes) start with the one byte \texttt{0x01}.
The stated reason is that
``this domain separation is required to give
second preimage resistance.''~\cite[Section 2.1.1]{rfc9162}.

\subsection{Merkle Trees with Arbitrary Leaves}

Our \gls{merkle tree} definitions in Algs.~\ref{alg:merkle_tree}
and \ref{alg:merkle_proof} only allowed for leaves
which were a power-of-two.
While it would be possible to always use this
by padding with a specific value,
we would prefer to not have this restriction.

To get around this, a convention must be followed.
One choice may be found in~\cite{rfc6962,rfc9162};
another convention is specified in the OpenZeppelin \gls{merkle tree}
library~\cite{MerkleTreeOZ}.

\subsubsection{Certificate Transparency Convention}

We will first look at the convention in~\cite{rfc6962,rfc9162},
although we will use the general hash from~\cite{rfc9162}
rather than \ShaTwo{}-256 from~\cite{rfc6962}.
The precise definition for computing the \gls{merkle tree}
root hash may be found in Alg.~\ref{alg:merkle_tree_ctv};
the definition is defined using \gls{recursion}.
Some visualizations of these trees may be found in
Figures~\ref{fig:merkle_tree_ctv_5},
\ref{fig:merkle_tree_ctv_6},
\ref{fig:merkle_tree_ctv_7}, and
\ref{fig:merkle_tree_ctv_8}.
Algorithms for constructing inclusion proofs and verifying them
may be found in~\cite{rfc9162}.
As mentioned above, second preimage security is ensured by
using domain separation.

\begin{algorithm}[t]
\caption{Compute Merkle Tree root hash using \gls{recursion}
    from~\cite{rfc9162}}
\label{alg:merkle_tree_ctv}
\begin{algorithmic}[1]
\Procedure{MerkleRootCTV}{$x_{0}, x_{1}, \cdots, x_{n-1}$}
    \If {$n = 0$}
        \State \Return $\textsf{Hash}()$
        \Comment{Hash of empty byte string}
    \ElsIf {$n = 1$}
        \State \Return $\textsf{Hash}(\texttt{0x00}, x_{0})$
    \Else
        \State $k \algAssign{}
            \max\braces{2^{\ell} \mid \ell\in\N \land 2^{\ell} < n}$
        \Comment{$k$ is largest power-of-two strictly less than $n$}
        \State $h_{0} \algAssign{}
            \textsc{MerkleRootCTV}(x_{0}, x_{1}, \cdots, x_{k-1})$
        \State $h_{1} \algAssign{}
            \textsc{MerkleRootCTV}(x_{k}, x_{k+1}, \cdots, x_{n-1})$
        \State \Return $\textsf{Hash}(\texttt{0x01}, h_{0}, h_{1})$
    \EndIf
\EndProcedure
\end{algorithmic}
\end{algorithm}

\begin{figure}[p]
\centering
\begin{tikzpicture}
    [level 1/.style={sibling distance=80mm},
     level 2/.style={sibling distance=40mm},
     level 3/.style={sibling distance=20mm}]
    \node[circle,draw] (root) {$H$}
        child {node[circle,draw] {$H$}
            child {node[circle,draw] {$H$}
                child {node[rectangle,draw,rounded corners] {$H$}
                    child {node[rectangle,draw] {$x_{0}$}
                    edge from parent[<-,thick]}
                edge from parent[<-,thick]
                    node[left] {$y_{0}$}
                }
                child {node[rectangle,draw,rounded corners] {$H$}
                    child {node[rectangle,draw] {$x_{1}$}
                    edge from parent[<-,thick]}
                edge from parent[<-,thick]
                    node[right] {$y_{1}$}
                }
            edge from parent[<-,thick]
                node[left=3pt] {$y_{4}$}
            }
            child {node[circle,draw] {$H$}
                child {node[rectangle,draw,rounded corners] {$H$}
                    child {node[rectangle,draw] {$x_{2}$}
                    edge from parent[<-,thick]}
                edge from parent[<-,thick]
                    node[left] {$y_{2}$}
                }
                child {node[rectangle,draw,rounded corners] {$H$}
                    child {node[rectangle,draw] {$x_{3}$}
                    edge from parent[<-,thick]}
                edge from parent[<-,thick]
                    node[right] {$y_{3}$}
                }
            edge from parent[<-,thick]
                node[right=3pt] {$y_{5}$}
            }
        edge from parent[<-,thick]
            node[left=9pt] {$y_{6}$}
        }
        child {node[rectangle,draw,rounded corners] {$H$}
            child {node[rectangle,draw] {$x_{4}$}
            edge from parent[<-,thick]}
        edge from parent[<-,thick]
            node[right=10pt] {$y_{7}$}
        };
    \node[above=of root] (header) {};
    \draw[->,thick] (root) to node[auto,swap] {$y_{8}$} (header);
\end{tikzpicture}
\caption{Merkle Tree with 5 leaves using
    recursive definition from~\cite{rfc9162}.}
\label{fig:merkle_tree_ctv_5}
\end{figure}

\begin{figure}[p]
\centering
\begin{tikzpicture}
    [level 1/.style={sibling distance=80mm},
     level 2/.style={sibling distance=40mm},
     level 3/.style={sibling distance=20mm}]
    \node[circle,draw] (root) {$H$}
        child {node[circle,draw] {$H$}
            child {node[circle,draw] {$H$}
                child {node[rectangle,draw,rounded corners] {$H$}
                    child {node[rectangle,draw] {$x_{0}$}
                    edge from parent[<-,thick]}
                edge from parent[<-,thick]
                    node[left] {$y_{0}$}
                }
                child {node[rectangle,draw,rounded corners] {$H$}
                    child {node[rectangle,draw] {$x_{1}$}
                    edge from parent[<-,thick]}
                edge from parent[<-,thick]
                    node[right] {$y_{1}$}
                }
            edge from parent[<-,thick]
                node[left=3pt] {$y_{4}$}
            }
            child {node[circle,draw] {$H$}
                child {node[rectangle,draw,rounded corners] {$H$}
                    child {node[rectangle,draw] {$x_{2}$}
                    edge from parent[<-,thick]}
                edge from parent[<-,thick]
                    node[left] {$y_{2}$}
                }
                child {node[rectangle,draw,rounded corners] {$H$}
                    child {node[rectangle,draw] {$x_{3}$}
                    edge from parent[<-,thick]}
                edge from parent[<-,thick]
                    node[right] {$y_{3}$}
                }
            edge from parent[<-,thick]
                node[right=3pt] {$y_{5}$}
            }
        edge from parent[<-,thick]
            node[left=9pt] {$y_{8}$}
        }
        child {node[circle,draw] {$H$}
            child {node[rectangle,draw,rounded corners] {$H$}
                child {node[rectangle,draw] {$x_{4}$}
                edge from parent[<-,thick]}
            edge from parent[<-,thick]
                node[left=3pt] {$y_{6}$}
            }
            child {node[rectangle,draw,rounded corners] {$H$}
                child {node[rectangle,draw] {$x_{5}$}
                edge from parent[<-,thick]}
            edge from parent[<-,thick]
                node[right=3pt] {$y_{7}$}
            }
        edge from parent[<-,thick]
            node[right=10pt] {$y_{9}$}
        };
    \node[above=of root] (header) {};
    \draw[->,thick] (root) to node[auto,swap] {$y_{10}$} (header);
\end{tikzpicture}
\caption{Merkle Tree with 6 leaves using
    recursive definition from~\cite{rfc9162}.}
\label{fig:merkle_tree_ctv_6}
\end{figure}

\begin{figure}[p]
\centering
\begin{tikzpicture}
    [level 1/.style={sibling distance=80mm},
     level 2/.style={sibling distance=40mm},
     level 3/.style={sibling distance=20mm}]
    \node[circle,draw] (root) {$H$}
        child {node[circle,draw] {$H$}
            child {node[circle,draw] {$H$}
                child {node[rectangle,draw,rounded corners] {$H$}
                    child {node[rectangle,draw] {$x_{0}$}
                    edge from parent[<-,thick]}
                edge from parent[<-,thick]
                    node[left] {$y_{0}$}
                }
                child {node[rectangle,draw,rounded corners] {$H$}
                    child {node[rectangle,draw] {$x_{1}$}
                    edge from parent[<-,thick]}
                edge from parent[<-,thick]
                    node[right] {$y_{1}$}
                }
            edge from parent[<-,thick]
                node[left=3pt] {$y_{6}$}
            }
            child {node[circle,draw] {$H$}
                child {node[rectangle,draw,rounded corners] {$H$}
                    child {node[rectangle,draw] {$x_{2}$}
                    edge from parent[<-,thick]}
                edge from parent[<-,thick]
                    node[left] {$y_{2}$}
                }
                child {node[rectangle,draw,rounded corners] {$H$}
                    child {node[rectangle,draw] {$x_{3}$}
                    edge from parent[<-,thick]}
                edge from parent[<-,thick]
                    node[right] {$y_{3}$}
                }
            edge from parent[<-,thick]
                node[right=3pt] {$y_{7}$}
            }
        edge from parent[<-,thick]
            node[left=9pt] {$y_{10}$}
        }
        child {node[circle,draw] {$H$}
            child {node[circle,draw] {$H$}
                child {node[rectangle,draw,rounded corners] {$H$}
                    child {node[rectangle,draw] {$x_{4}$}
                    edge from parent[<-,thick]}
                edge from parent[<-,thick]
                    node[left] {$y_{4}$}
                }
                child {node[rectangle,draw,rounded corners] {$H$}
                    child {node[rectangle,draw] {$x_{5}$}
                    edge from parent[<-,thick]}
                edge from parent[<-,thick]
                    node[right] {$y_{5}$}
                }
            edge from parent[<-,thick]
                node[left=3pt] {$y_{8}$}
            }
            child {node[rectangle,draw,rounded corners] {$H$}
                child {node[rectangle,draw] {$x_{6}$}
                edge from parent[<-,thick]}
            edge from parent[<-,thick]
                node[right=3pt] {$y_{9}$}
            }
        edge from parent[<-,thick]
            node[right=10pt] {$y_{11}$}
        };
    \node[above=of root] (header) {};
    \draw[->,thick] (root) to node[auto,swap] {$y_{12}$} (header);
\end{tikzpicture}
\caption{Merkle Tree with 7 leaves using
    recursive definition from~\cite{rfc9162}.}
\label{fig:merkle_tree_ctv_7}
\end{figure}

\begin{figure}[p]
\centering
\begin{tikzpicture}
    [level 1/.style={sibling distance=80mm},
     level 2/.style={sibling distance=40mm},
     level 3/.style={sibling distance=20mm}]
    \node[circle,draw] (root) {$H$}
        child {node[circle,draw] {$H$}
            child {node[circle,draw] {$H$}
                child {node[rectangle,draw,rounded corners] {$H$}
                    child {node[rectangle,draw] {$x_{0}$}
                    edge from parent[<-,thick]}
                edge from parent[<-,thick]
                    node[left] {$y_{0}$}
                }
                child {node[rectangle,draw,rounded corners] {$H$}
                    child {node[rectangle,draw] {$x_{1}$}
                    edge from parent[<-,thick]}
                edge from parent[<-,thick]
                    node[right] {$y_{1}$}
                }
            edge from parent[<-,thick]
                node[left=3pt] {$y_{8}$}
            }
            child {node[circle,draw] {$H$}
                child {node[rectangle,draw,rounded corners] {$H$}
                    child {node[rectangle,draw] {$x_{2}$}
                    edge from parent[<-,thick]}
                edge from parent[<-,thick]
                    node[left] {$y_{2}$}
                }
                child {node[rectangle,draw,rounded corners] {$H$}
                    child {node[rectangle,draw] {$x_{3}$}
                    edge from parent[<-,thick]}
                edge from parent[<-,thick]
                    node[right] {$y_{3}$}
                }
            edge from parent[<-,thick]
                node[right=3pt] {$y_{9}$}
            }
        edge from parent[<-,thick]
            node[left=9pt] {$y_{12}$}
        }
        child {node[circle,draw] {$H$}
            child {node[circle,draw] {$H$}
                child {node[rectangle,draw,rounded corners] {$H$}
                    child {node[rectangle,draw] {$x_{4}$}
                    edge from parent[<-,thick]}
                edge from parent[<-,thick]
                    node[left] {$y_{4}$}
                }
                child {node[rectangle,draw,rounded corners] {$H$}
                    child {node[rectangle,draw] {$x_{5}$}
                    edge from parent[<-,thick]}
                edge from parent[<-,thick]
                    node[right] {$y_{5}$}
                }
            edge from parent[<-,thick]
                node[left=3pt] {$y_{10}$}
            }
            child {node[circle,draw] {$H$}
                child {node[rectangle,draw,rounded corners] {$H$}
                    child {node[rectangle,draw] {$x_{6}$}
                    edge from parent[<-,thick]}
                edge from parent[<-,thick]
                    node[left] {$y_{6}$}
                }
                child {node[rectangle,draw,rounded corners] {$H$}
                    child {node[rectangle,draw] {$x_{7}$}
                    edge from parent[<-,thick]}
                edge from parent[<-,thick]
                    node[right] {$y_{7}$}
                }
            edge from parent[<-,thick]
                node[right=3pt] {$y_{11}$}
            }
        edge from parent[<-,thick]
            node[right=10pt] {$y_{13}$}
        };
    \node[above=of root] (header) {};
    \draw[->,thick] (root) to node[auto,swap] {$y_{14}$} (header);
\end{tikzpicture}
\caption{Merkle Tree with 8 leaves using
    recursive definition from~\cite{rfc9162}.}
\label{fig:merkle_tree_ctv_8}
\end{figure}


\subsubsection{OpenZeppelin Convention}

A slightly different convention is specified in the OpenZeppelin
\gls{merkle tree} library~\cite{MerkleTreeOZ};
this convention has been used on the \gls{ethereum} blockchain
within \glspl{smart contract}.
The definition for computing the \gls{merkle tree}
root hash may be found in Alg.~\ref{alg:merkle_tree_oz}.
Second preimage security is ensured by double-hashing the leaf nodes.
Some visualizations of these trees may be found in
Figures~\ref{fig:merkle_tree_oz_5},
\ref{fig:merkle_tree_oz_6},
\ref{fig:merkle_tree_oz_7}, and
\ref{fig:merkle_tree_oz_8}.
Additional algorithms may be found in~\cite{MerkleTreeOZ}.

\begin{algorithm}[t]
\caption{Compute Merkle Tree root hash from~\cite{MerkleTreeOZ}}
\label{alg:merkle_tree_oz}
\begin{algorithmic}[1]
\Procedure{MerkleRootOZ}{$\texttt{data}$}
    \State $n \algAssign{} \textsc{Length}(\texttt{data})$
    \State $\texttt{leaves} \algAssign{} \textsc{Make}(\text{array}, n)$
    \Comment{Initialize an empty array of length $n$}
    \For{$k=0$; $k < n$; $k{+}{+}$}
        \State $\texttt{leaves}[k] \algAssign{}
            \textsc{LeafHash}(\texttt{data}[k])$
    \EndFor
    \State $\texttt{tree} \algAssign{} \textsc{Make}(\text{array}, 2n-1)$
    \Comment{Initialize an empty array of length $2n-1$}
    \For{$k=0$; $k < n$; $k{+}{+}$}
        \State $\texttt{tree}[2n-2-k] \algAssign{} \texttt{leaves}[k])$
    \EndFor
    \For{$k=n-2$; $k\ge0$; $k{-}{-}$}
        \State $\texttt{tree}[k] \algAssign{}
            \textsc{InteriorHash}(\texttt{tree}[2k+1],\texttt{tree}[2k+2])$
    \EndFor
    \State \Return $\texttt{tree}[0]$
    \Comment{Return the root hash}
\EndProcedure

\State
\Procedure{LeafHash}{$x$}
    \State \Return $\textsf{Hash}(\textsf{Hash}(x))$
\EndProcedure

\State
\Procedure{InteriorHash}{$a$, $b$}
    \If {$\text{int}(a) < \text{int}(b)$}
        \State \Return $\textsf{Hash}(a, b)$
    \Else
        \State \Return $\textsf{Hash}(b, a)$
    \EndIf
\EndProcedure
\end{algorithmic}
\end{algorithm}

\begin{figure}[p]
\centering
\begin{tikzpicture}
    [level 1/.style={sibling distance=80mm},
     level 2/.style={sibling distance=40mm},
     level 3/.style={sibling distance=20mm}]
    \node[circle,draw] (root) {$H$}
        child {node[circle,draw] {$H$}
            child {node[circle,draw] {$H$}
                child {node[rectangle,draw,rounded corners] {$H$}
                    child {node[rectangle,draw] {$x_{1}$}
                    edge from parent[<-,thick]}
                edge from parent[<-,thick]
                    node[left] {$L_{1}$}
                }
                child {node[rectangle,draw,rounded corners] {$H$}
                    child {node[rectangle,draw] {$x_{0}$}
                    edge from parent[<-,thick]}
                edge from parent[<-,thick]
                    node[right] {$L_{0}$}
                }
            edge from parent[<-,thick]
                node[left=4pt] {$I_{3}$}
            }
            child {node[rectangle,draw,rounded corners] {$H$}
                child {node[rectangle,draw] {$x_{4}$}
                edge from parent[<-,thick]}
            edge from parent[<-,thick]
                node[right=4pt] {$L_{4}$}
            }
        edge from parent[<-,thick]
            node[left=12pt] {$I_{1}$}
        }
        child {node[circle,draw] {$H$}
            child {node[rectangle,draw,rounded corners] {$H$}
                child {node[rectangle,draw] {$x_{3}$}
                edge from parent[<-,thick]}
            edge from parent[<-,thick]
                node[left=4pt] {$L_{3}$}
            }
            child {node[rectangle,draw,rounded corners] {$H$}
                child {node[rectangle,draw] {$x_{2}$}
                edge from parent[<-,thick]}
            edge from parent[<-,thick]
                node[right=4pt] {$L_{2}$}
            }
        edge from parent[<-,thick]
            node[right=11pt] {$I_{2}$}
        };
    \node[above=of root] (header) {};
    \draw[->,thick] (root) to node[auto,swap] {$I_{0}$} (header);
\end{tikzpicture}
\caption{Merkle Tree with 5 leaves using definition from~\cite{MerkleTreeOZ}.}
\label{fig:merkle_tree_oz_5}
\end{figure}

\begin{figure}[p]
\centering
\begin{tikzpicture}
    [level 1/.style={sibling distance=80mm},
     level 2/.style={sibling distance=40mm},
     level 3/.style={sibling distance=20mm}]
    \node[circle,draw] (root) {$H$}
        child {node[circle,draw] {$H$}
            child {node[circle,draw] {$H$}
                child {node[rectangle,draw,rounded corners] {$H$}
                    child {node[rectangle,draw] {$x_{3}$}
                    edge from parent[<-,thick]}
                edge from parent[<-,thick]
                    node[left] {$L_{3}$}
                }
                child {node[rectangle,draw,rounded corners] {$H$}
                    child {node[rectangle,draw] {$x_{2}$}
                    edge from parent[<-,thick]}
                edge from parent[<-,thick]
                    node[right] {$L_{2}$}
                }
            edge from parent[<-,thick]
                node[left=4pt] {$I_{3}$}
            }
            child {node[circle,draw] {$H$}
                child {node[rectangle,draw,rounded corners] {$H$}
                    child {node[rectangle,draw] {$x_{1}$}
                    edge from parent[<-,thick]}
                edge from parent[<-,thick]
                    node[left] {$L_{1}$}
                }
                child {node[rectangle,draw,rounded corners] {$H$}
                    child {node[rectangle,draw] {$x_{0}$}
                    edge from parent[<-,thick]}
                edge from parent[<-,thick]
                    node[right] {$L_{0}$}
                }
            edge from parent[<-,thick]
                node[right=4pt] {$I_{4}$}
            }
        edge from parent[<-,thick]
            node[left=12pt] {$I_{1}$}
        }
        child {node[circle,draw] {$H$}
            child {node[rectangle,draw,rounded corners] {$H$}
                child {node[rectangle,draw] {$x_{5}$}
                edge from parent[<-,thick]}
            edge from parent[<-,thick]
                node[left=4pt] {$L_{5}$}
            }
            child {node[rectangle,draw,rounded corners] {$H$}
                child {node[rectangle,draw] {$x_{4}$}
                edge from parent[<-,thick]}
            edge from parent[<-,thick]
                node[right=4pt] {$L_{4}$}
            }
        edge from parent[<-,thick]
            node[right=11pt] {$I_{2}$}
        };
    \node[above=of root] (header) {};
    \draw[->,thick] (root) to node[auto,swap] {$I_{0}$} (header);
\end{tikzpicture}
\caption{Merkle Tree with 6 leaves using definition from~\cite{MerkleTreeOZ}.}
\label{fig:merkle_tree_oz_6}
\end{figure}

\begin{figure}[p]
\centering
\begin{tikzpicture}
    [level 1/.style={sibling distance=80mm},
     level 2/.style={sibling distance=40mm},
     level 3/.style={sibling distance=20mm}]
    \node[circle,draw] (root) {$H$}
        child {node[circle,draw] {$H$}
            child {node[circle,draw] {$H$}
                child {node[rectangle,draw,rounded corners] {$H$}
                    child {node[rectangle,draw] {$x_{5}$}
                    edge from parent[<-,thick]}
                edge from parent[<-,thick]
                    node[left] {$L_{5}$}
                }
                child {node[rectangle,draw,rounded corners] {$H$}
                    child {node[rectangle,draw] {$x_{4}$}
                    edge from parent[<-,thick]}
                edge from parent[<-,thick]
                    node[right] {$L_{4}$}
                }
            edge from parent[<-,thick]
                node[left=4pt] {$I_{3}$}
            }
            child {node[circle,draw] {$H$}
                child {node[rectangle,draw,rounded corners] {$H$}
                    child {node[rectangle,draw] {$x_{3}$}
                    edge from parent[<-,thick]}
                edge from parent[<-,thick]
                    node[left] {$L_{3}$}
                }
                child {node[rectangle,draw,rounded corners] {$H$}
                    child {node[rectangle,draw] {$x_{2}$}
                    edge from parent[<-,thick]}
                edge from parent[<-,thick]
                    node[right] {$L_{2}$}
                }
            edge from parent[<-,thick]
                node[right=4pt] {$I_{4}$}
            }
        edge from parent[<-,thick]
            node[left=12pt] {$I_{1}$}
        }
        child {node[circle,draw] {$H$}
            child {node[circle,draw] {$H$}
                child {node[rectangle,draw,rounded corners] {$H$}
                    child {node[rectangle,draw] {$x_{1}$}
                    edge from parent[<-,thick]}
                edge from parent[<-,thick]
                    node[left] {$L_{1}$}
                }
                child {node[rectangle,draw,rounded corners] {$H$}
                    child {node[rectangle,draw] {$x_{0}$}
                    edge from parent[<-,thick]}
                edge from parent[<-,thick]
                    node[right] {$L_{0}$}
                }
            edge from parent[<-,thick]
                node[left=4pt] {$I_{5}$}
            }
            child {node[rectangle,draw,rounded corners] {$H$}
                child {node[rectangle,draw] {$x_{6}$}
                edge from parent[<-,thick]}
            edge from parent[<-,thick]
                node[right=4pt] {$L_{6}$}
            }
        edge from parent[<-,thick]
            node[right=11pt] {$I_{2}$}
        };
    \node[above=of root] (header) {};
    \draw[->,thick] (root) to node[auto,swap] {$I_{0}$} (header);
\end{tikzpicture}
\caption{Merkle Tree with 7 leaves using definition from~\cite{MerkleTreeOZ}.}
\label{fig:merkle_tree_oz_7}
\end{figure}

\begin{figure}[p]
\centering
\begin{tikzpicture}
    [level 1/.style={sibling distance=80mm},
     level 2/.style={sibling distance=40mm},
     level 3/.style={sibling distance=20mm}]
    \node[circle,draw] (root) {$H$}
        child {node[circle,draw] {$H$}
            child {node[circle,draw] {$H$}
                child {node[rectangle,draw,rounded corners] {$H$}
                    child {node[rectangle,draw] {$x_{7}$}
                    edge from parent[<-,thick]}
                edge from parent[<-,thick]
                    node[left] {$L_{7}$}
                }
                child {node[rectangle,draw,rounded corners] {$H$}
                    child {node[rectangle,draw] {$x_{6}$}
                    edge from parent[<-,thick]}
                edge from parent[<-,thick]
                    node[right] {$L_{6}$}
                }
            edge from parent[<-,thick]
                node[left=4pt] {$I_{3}$}
            }
            child {node[circle,draw] {$H$}
                child {node[rectangle,draw,rounded corners] {$H$}
                    child {node[rectangle,draw] {$x_{5}$}
                    edge from parent[<-,thick]}
                edge from parent[<-,thick]
                    node[left] {$L_{5}$}
                }
                child {node[rectangle,draw,rounded corners] {$H$}
                    child {node[rectangle,draw] {$x_{4}$}
                    edge from parent[<-,thick]}
                edge from parent[<-,thick]
                    node[right] {$L_{4}$}
                }
            edge from parent[<-,thick]
                node[right=4pt] {$I_{4}$}
            }
        edge from parent[<-,thick]
            node[left=12pt] {$I_{1}$}
        }
        child {node[circle,draw] {$H$}
            child {node[circle,draw] {$H$}
                child {node[rectangle,draw,rounded corners] {$H$}
                    child {node[rectangle,draw] {$x_{3}$}
                    edge from parent[<-,thick]}
                edge from parent[<-,thick]
                    node[left] {$L_{3}$}
                }
                child {node[rectangle,draw,rounded corners] {$H$}
                    child {node[rectangle,draw] {$x_{2}$}
                    edge from parent[<-,thick]}
                edge from parent[<-,thick]
                    node[right] {$L_{2}$}
                }
            edge from parent[<-,thick]
                node[left=4pt] {$I_{5}$}
            }
            child {node[circle,draw] {$H$}
                child {node[rectangle,draw,rounded corners] {$H$}
                    child {node[rectangle,draw] {$x_{1}$}
                    edge from parent[<-,thick]}
                edge from parent[<-,thick]
                    node[left] {$L_{1}$}
                }
                child {node[rectangle,draw,rounded corners] {$H$}
                    child {node[rectangle,draw] {$x_{0}$}
                    edge from parent[<-,thick]}
                edge from parent[<-,thick]
                    node[right] {$L_{0}$}
                }
            edge from parent[<-,thick]
                node[right=4pt] {$I_{6}$}
            }
        edge from parent[<-,thick]
            node[right=11pt] {$I_{2}$}
        };
    \node[above=of root] (header) {};
    \draw[->,thick] (root) to node[auto,swap] {$I_{0}$} (header);
\end{tikzpicture}
\caption{Merkle Tree with 8 leaves using definition from~\cite{MerkleTreeOZ}.}
\label{fig:merkle_tree_oz_8}
\end{figure}



\section{Sparse Merkle Trees}
\label{sec:sparse_merkle_trees}

\subsection{Discussion}

We continue our discussion of \glspl{merkle tree} by noting
another variation: \emph{\glspl{smt}}.
In this case, the number of leaves may be very large;
however, \emph{most} of the leaves are empty,
so it is still practical to work with the tree.

In particular, using a 256-bit \gls{hash function} such as
\ShaTwo{}-256 or \ShaThree{}-256,
there are a maximum of $2^{256}\simeq 10^{77}$ leaves.

Additionally, while when we discussed \glspl{merkle tree} in
Chapters~\ref{sec:merkle_trees} and \ref{sec:merkle_trees_2}
we placed no restriction on the key order,
in \glspl{smt} the keys will be sorted.
Furthermore, if there is no data for a particular key,
it is handled in a special way.

\subsection{Conventions}

There are various conventions for \glspl{smt};
this is needed in order to know how to handle empty leaves.
Some references include~\cite{EfficientSMT,CompactSMT,CompactMerkleMultiproofs,
RevocationTransparency2012,AngelaSMT}.

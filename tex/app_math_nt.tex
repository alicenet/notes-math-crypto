\section{Number Theory}
\label{app:math_nt}

We now have a longer discussion related to greatest common divisors
and how to compute them using the Euclidean Algorithm.
We also talk about the Extended Euclidean Algorithm.

\subsection{Euclidean Division}

We will need the following properties of the natural numbers and integers:

\begin{prop}[Well-Ordering Principle]
\label{prop:math_well_ordering_naturals}
Every nonempty set of natural numbers contains a minimal element.
\end{prop}

\begin{prop}[No Zero Divisors]
\label{prop:math_no_zero_divisors}
If $a,b\in\Z$ and $ab=0$, then $a=0$ or $b=0$.
\end{prop}

In our discussion about \gls{number theory},
we have repeatedly used the following result:

\begin{prop}[Euclidean Division]
\label{prop:math_euclid_division}
Given $a,b\in\N$ with $b\ge1$, there are \emph{unique} values
$q,r\in\N$ such that

\begin{equation}
    a = qb + r.
\end{equation}

\noindent
Here, $q$ is called the \emph{quotient} and $r$ is called the \emph{remainder}.
\end{prop}

We give a formal proof now.
We will use Propositions~\ref{prop:math_well_ordering_naturals}
and \ref{prop:math_no_zero_divisors}.

\begin{proof}[Proof of Proposition~\ref{prop:math_euclid_division}]
If $a=0$, then we see that we must have $q=0$ and $r=0$.

Thus, we assume that $a\ge1$.
Define

\begin{equation}
    S \mathDef{} \braces{a - qb \mid q\in\N \text{ and } (a-qb\ge0)}.
\end{equation}

\noindent
That is, $S$ contains all values of the form $a-qb$ which are nonnegative
for $q\in\N$.
Because $a\ge1$, $0\in\N$, and

\begin{equation}
    a = a - 0\cdot b,
\end{equation}

\noindent
we have that $a\in S$.
Thus, $S$ is nonempty.

As defined, $S$ is a nonempty collection of natural numbers.
We set $r \mathDef{} \min S$;
such a minimum exists by Proposition~\ref{prop:math_well_ordering_naturals}.
Now, $r\in S$, so we see that we have

\begin{equation}
    a = qb + r
\end{equation}

\noindent
for some $q\in\N$.

If $r\ge b$, then it would follow that

\begin{align}
    a &= qb + r \nonumber\\
        &= \parens{qb + b} + \parens{r-b} \nonumber\\
        &= \parens{q+1}b + \bar{r}.
\end{align}

\noindent
Here, we have $\bar{r} = r-b\ge0$.
Thus, we see that if this were possible, then $\bar{r}\in S$ and
$\bar{r} < r$.
By the minimality of $r$, this is not possible.
Thus, we have $0\le r < b$.
This finishes the proof of existence.

We now prove uniqueness.
First, we suppose that

\begin{align}
    a &= qb + r \nonumber\\
    a &= q'b + r' \nonumber\\
\end{align}

\noindent
for $q,q'\in\N$, $0\le r < b$, and $0\le r' < b$.
This implies that

\begin{equation}
    \parens{q-q'}b = r-r'.
\end{equation}

\noindent
Thus, we see that $b\mid r-r'$.
Now, $0\le r' < b$ implies that $0\ge -r' > -b$, so it follows
that

\begin{equation}
    b > r \ge r - r' > r - b\ge -b.
\end{equation}

\noindent
By combining these inequalities, we find

\begin{equation}
    \abs{r-r'} < b.
\end{equation}

\noindent
However, we also know that $b \mid r-r'$, so this implies that $r-r' = 0$.
Thus, we have

\begin{equation}
    \parens{q-q'}b = 0.
\end{equation}

\noindent
Because $b\ne0$, we have $q - q'=0$;
this follows from Proposition~\ref{prop:math_no_zero_divisors}.
This finishes the proof of uniqueness.
\end{proof}

We note that while we stated Proposition~\ref{prop:math_euclid_division}
as applying for $a\ge0$ and $b\ge1$, we could have modified it to the more
general situation when $a\in\Z$, $b\ne0$, and $0 \le r < \abs{b}$.
Even so, it is not of much additional value,
and the proof is essentially the same.


\subsection{Euclidean Algorithm}

To compute the greatest common divisor (gcd),
we use the Euclidean Algorithm;
see Alg.~\ref{alg:euclidean_alg}.
This is an efficient algorithm.

\begin{algorithm}[t]
\caption[Euclidean Division]{Perform Euclidean division}
\label{alg:euclidean_div}
\begin{algorithmic}[1]
\Require $a\ge 0$, $b\ge 1$
\Procedure{EuclideanDivision}{$a$, $b$}
    \State $q \algAssign{} \lfloor a/b\rfloor$
    \State $r \algAssign{} a - qb$
    \State \Return $q$, $r$
\EndProcedure
\end{algorithmic}

\end{algorithm}
\begin{algorithm}[t]
\caption[Euclidean Algorithm]{Compute the greatest common divisor}
\label{alg:euclidean_alg}
\begin{algorithmic}[1]
\Require $a\ge b\ge 1$
\Procedure{EuclideanAlgorithm}{$a$, $b$}
    \State $r_{0} \algAssign{} a$
    \State $r_{1} \algAssign{} b$
    \State $q_{1}$, $r_{2} \algAssign{} \textsc{EuclideanDivision}(r_{0},r_{1})$
    \State $k \algAssign{} 2$
    \While {$r_{k}\ne0$}
        \State $q_{k}$, $r_{k+1} \algAssign{}
            \textsc{EuclideanDivision}(r_{k-1},r_{k})$
        \State $k \algAssign{} k+1$
    \EndWhile
    \State $d \algAssign{} r_{k-1}$
    \Comment{Greatest Common Divisor}
    \State \Return $d$
\EndProcedure
\end{algorithmic}
\end{algorithm}


We will prove that it terminates, bound the number of iterations,
and prove that it converges to the gcd.

\begin{proof}[Proof that Alg.~\ref{alg:euclidean_alg} produces the
    Greatest Common Divisor]

We will start by proving convergence before we bound the number
of iterations.

\subsubsection*{Proof of Convergence}
First, we see that

\begin{equation}
    r_{0} \ge r_{1} > r_{2} > \cdots
\end{equation}

\noindent
with $r_{k}\ge0$.
This shows that $\braces{r_{k}}$ is a strictly decreasing sequence of natural
numbers (after $r_{0}$) bounded below by $0$;
thus, $r_{k}$ will reach $0$ in at most $b$ steps.
This shows the algorithm will eventually terminate.

At each step, we have

\begin{equation}
    r_{k-1} = q_{k}r_{k} + r_{k+1}.
\end{equation}

\noindent
At the end, we have

\begin{equation}
    r_{N-1} = q_{N}r_{N}.
\end{equation}

\noindent
We set

\begin{equation}
    d \mathDef{} r_{N},
\end{equation}

\noindent
and we will prove that $d$ is the gcd.

We notice that $d \mid r_{N-1}$.
We also see

\begin{align}
    r_{N-2} &= q_{N-1}r_{N-1} + r_{N} \nonumber\\
        &= \parens{q_{N-1}q_{N} + 1}d;
\end{align}

\noindent
thus, we also have $d \mid r_{N-2}$.
By continuing the same logic, we find that $d \mid r_{k}$ for all
$k\in\braces{0, 1, \cdots, N}$.
In particular, we have $d \mid r_{0}$ and $d \mid r_{1}$.
Written another way, we have $d \mid a$ and $d \mid b$.
It follows that $d$ is a \emph{common divisor} of $a$ and $b$.

Let $\ell\ge0$ be any other divisor of $a$ and $b$;
in particular, suppose

\begin{align}
    a &= \alpha \ell \nonumber\\
    b &= \beta \ell.
\end{align}

\noindent
Because $r_{0}=a$ and $r_{1}=b$, we have that $\ell\mid r_{0}$
and $\ell\mid r_{1}$.
We know

\begin{equation}
    a = q_{1}b + r_{2}.
\end{equation}

\noindent
Rearranging this, we see

\begin{align}
    r_{2} &= a - q_{1}b \nonumber\\
        &= \ell\parens{\alpha - q_{1}\beta}.
\end{align}

\noindent
This implies that we also have $\ell \mid r_{2}$.
Continuing this logic, we see that
$\ell \mid r_{k}$ for all $k\in\braces{0, 1, 2, \cdots, N}$.
It follows that $\ell \mid d$ because $d = r_{N}$.
Because $\ell \mid d$, we know that $\ell\le d$.
Thus, we see that $d$ is greater than or equal to \emph{all} divisors
of $a$ and $b$;
it is thus the \emph{greatest common divisor}.

\subsubsection*{Bounding the number of iterations}
We already showed that we will converge in at most $b$ steps.
We now produce a much better bound.

To do this, we need to look at the general recurrence relation more closely:

\begin{equation}
    r_{k} = q_{k+1}r_{k+1} + r_{k+2}.
\end{equation}

\noindent
We assume that $r_{k}\ge r_{k+1}$.
We always have $0\le r_{k+2} < r_{k+1}$,
but we can say more:
we always have

\begin{equation}
    r_{k+2} < \frac{r_{k}}{2}.
\end{equation}

We consider two cases:

\begin{itemize}
\item Case 1: $r_{k+1} > \frac{r_{k}}{2}$

In this case, we see

\begin{equation}
    0 \le r_{k} - r_{k+1} < \frac{r_{k}}{2}.
\end{equation}

\noindent
This equation combined with $r_{k+1} > \frac{r_{k}}{2}$
shows us that we have

\begin{equation}
    0 \le r_{k} - r_{k+1} < r_{k+1}.
\end{equation}

\noindent
This means that $q_{k+1} = 1$.
Thus, we have

\begin{align}
    r_{k} &= r_{k+1} + \parens{r_{k}-r_{k+1}} \nonumber\\
        &= r_{k+1} + r_{k+2}
\end{align}

\noindent
with

\begin{equation}
    r_{k+2} < \frac{r_{k}}{2}.
\end{equation}

\item Case 2: $r_{k+1} \le \frac{r_{k}}{2}$

Because we always have

\begin{equation}
    r_{k+2} < r_{k+1}
\end{equation}

\noindent
and we are assuming $r_{k+1} \le \frac{r_{k}}{2}$,
we see that

\begin{equation}
    r_{k+2} < \frac{r_{k}}{2}.
\end{equation}
\end{itemize}

By repeatedly applying this fact, we arrive at the bound

\begin{equation}
    r_{2k} < \frac{r_{0}}{2^{k}}.
\end{equation}

\noindent
We recall that we always have $r_{N}\ge 1$ and $r_{N+1}=0$.
If $N = 2k$, then we see

\begin{equation}
    r_{N} < \frac{r_{0}}{2^{k}}.
\end{equation}

\noindent
The fact that $r_{N}\ge 1$ allows us to write

\begin{equation}
    1 < \frac{r_{0}}{2^{k}}.
\end{equation}

\noindent
Because $N = 2k$, it follows that

\begin{equation}
    N < 2\log_{2}r_{0}.
\end{equation}

\noindent
If $N = 2k+1$, then we have

\begin{equation}
    r_{N} < r_{2k} < \frac{r_{0}}{2^{k}}.
\end{equation}

\noindent
We then see

\begin{equation}
    N < 2\log_{2}r_{0} + 1.
\end{equation}

\noindent
Thus, we always have

\begin{equation}
    N \le 1 + 2\log_{2}r_{0}.
\end{equation}

\noindent
This shows that we need $N$ steps for $N \le 1 + 2\log_{2}a$
when $a\ge b\ge 0$.
\end{proof}

From our discussion of computational complexity,
we have just shown that Alg.~\ref{alg:euclidean_alg}
is a polynomial time algorithm to compute the gcd.

\begin{example}[Using the Euclidean Algorithm]
\label{example:app_math_euclidean_alg}
\exampleCodeReference{examples/app\_math/euclidean\_alg.py}

We now use Alg.~\ref{alg:euclidean_alg}
to work through Example~\ref{example:bnt_gcd}:
Compute $\gcd\parens{1080,1872}$.
We have

\begin{align}
    r_{0} &= 1872 \nonumber\\
    r_{1} &= 1080.
\end{align}

\noindent
We see

\begin{align}
    1872 &= 1\cdot1080 + 792
        \nonumber\\
    q_{1} &= 1
        \nonumber\\
    r_{2} &= 792.
\end{align}

\noindent
In the next step, we have

\begin{align}
    1080 &= 1\cdot792 + 288
        \nonumber\\
    q_{2} &= 1
        \nonumber\\
    r_{3} &= 288.
\end{align}

\noindent
The following step gives

\begin{align}
    792 &= 2\cdot288 + 216.
        \nonumber\\
    q_{3} &= 2
        \nonumber\\
    r_{4} &= 216.
\end{align}

\noindent
Continuing on, we find

\begin{align}
    288 &= 1\cdot216 + 72
        \nonumber\\
    q_{4} &= 1
        \nonumber\\
    r_{5} &= 72.
\end{align}

\noindent
At last we have

\begin{align}
    216 &= 3\cdot 72
        \nonumber\\
    q_{5} &= 3
        \nonumber\\
    r_{6} &= 0.
\end{align}

\noindent
The process has terminated and we have determined that

\begin{equation}
    \gcd(1080,1872) = 72.
\end{equation}
\end{example}


\subsection{Extended Euclidean Algorithm}

We also include the Extended Euclidean Algorithm;
see Alg.~\ref{alg:extended_euclidean_alg}.
This algorithm produces the gcd $d$ as well as $x$ and $y$ such that

\begin{equation}
    ax + by = d.
    \label{eq:bezout_identity}
\end{equation}

\begin{algorithm}[t]
\caption[Extended Euclidean Algorithm]{Solve $ax + by = \gcd(a,b)$ for $x$ and $y$}
\label{alg:extended_euclidean_alg}
\begin{algorithmic}[1]
\Require $a\ge b\ge 0$
\Procedure{ExtendedEuclideanAlgorithm}{$a$, $b$}
    \State $r_{0} \algAssign{} a$
    \State $s_{0} \algAssign{} 1$
    \State $t_{0} \algAssign{} 0$
    \State $r_{1} \algAssign{} b$
    \State $s_{1} \algAssign{} 0$
    \State $t_{1} \algAssign{} 1$
    \State $q_{1}$, $r_{2} \algAssign{} \textsc{EuclideanDivision}(r_{0},r_{1})$
    \State $s_{2} \algAssign{} s_{0} - q_{1}s_{1}$
    \State $t_{2} \algAssign{} t_{0} - q_{1}t_{1}$
    \State $k \algAssign{} 2$
    \While {$r_{k}\ne0$}
        \State $q_{k}$, $r_{k+1} \algAssign{}
            \textsc{EuclideanDivision}(r_{k-1},r_{k})$
        \State $s_{k+1} \algAssign{} s_{k-1} - q_{k}s_{k}$
        \State $t_{k+1} \algAssign{} t_{k-1} - q_{k}t_{k}$
        \State $k \algAssign{} k+1$
    \EndWhile
    \State $d \algAssign{} r_{k-1}$
    \Comment{Greatest Common Divisor}
    \State $x \algAssign{} s_{k-1}$
    \State $y \algAssign{} t_{k-1}$
    \State \Return $d$, $x$, $y$
\EndProcedure
\end{algorithmic}
\end{algorithm}


\noindent
In this case, $x$ and $y$ are called \emph{B\'{e}zout coefficients},
and Eq.~\eqref{eq:bezout_identity} is called the \emph{B\'{e}zout identity}.

\begin{proof}[Proof that Alg.~\ref{alg:extended_euclidean_alg} produces
    $ax + by = \gcd(a,b)$]

We see that this is an extension of Alg.~\ref{alg:euclidean_alg};
thus, we know that $d = \gcd(a,b)$.

We begin by showing

\begin{equation}
    as_{k} + bt_{k} = r_{k}
\end{equation}

\noindent
for all $k$.
First, we see that it holds for $k=0$ and $k=1$:

\begin{align}
    a\cdot1 + b\cdot0 &= a \nonumber\\
    a\cdot0 + b\cdot1 &= b.
\end{align}

\noindent
We suppose that the equation is valid for $r_{k}$ and $r_{k-1}$.
Then, we see

\begin{align}
    r_{k+1} &= r_{k-1} - q_{k}r_{k} \nonumber\\
        &= \parens{as_{k-1} + bt_{k-1}} - q_{k}\parens{as_{k} + bt_{k}}
            \nonumber\\
        &= a\parens{s_{k-1} - q_{k}s_{k}} + b\parens{t_{k-1} - q_{k}t_{k}}
            \nonumber\\
        &= as_{k+1} + bt_{k+1}.
\end{align}

\noindent
Thus, it also holds for $r_{k+1}$.
Continuing this argument, we have

\begin{equation}
    as_{k} + bt_{k} = r_{k}
\end{equation}

\noindent
for all $k\in\braces{0,1,\cdots,N}$.

We now focus on the final case:

\begin{equation}
    as_{N} + bt_{N} = r_{N}.
\end{equation}

\noindent
We already know that $d = r_{N} = \gcd(a,b)$.
With $x = s_{N}$ and $y = t_{N}$, we have

\begin{equation}
    ax + by = d,
\end{equation}

\noindent
as desired.
\end{proof}

\begin{example}[Using the Extended Euclidean Algorithm]
\label{example:app_math_extended_euclidean_alg}
\exampleCodeReference{examples/app\_math/extended\_euclidean\_alg.py}

We now use Alg.~\ref{alg:extended_euclidean_alg}.
to work through Example~\ref{example:bnt_extended_gcd}:
Compute $x$ and $y$ such that

\begin{equation}
    1080x + 1872y = \gcd\parens{1080,1872}.
\end{equation}

\noindent
We will reuse information we already computed in
Example~\ref{example:app_math_euclidean_alg}.

We start by listing the initial values:

\begin{align}
    r_{0} &= 1872
        \nonumber\\
    s_{0} &= 1
        \nonumber\\
    t_{0} &= 0
        \nonumber\\
    r_{1} &= 1080
        \nonumber\\
    s_{1} &= 0
        \nonumber\\
    t_{1} &= 1.
\end{align}

\noindent
We see

\begin{align}
    1872 &= 1\cdot1080 + 792
        \nonumber\\
    q_{1} &= 1
        \nonumber\\
    r_{2} &= 792
        \nonumber\\
    s_{2} &= s_{0} - q_{1}s_{1}
            \nonumber\\
        &= 1
            \nonumber\\
    t_{2} &= t_{0} - q_{1}t_{1}
            \nonumber\\
        &= -1.
\end{align}

\noindent
In the next step, we have

\begin{align}
    1080 &= 1\cdot792 + 288
        \nonumber\\
    q_{2} &= 1
        \nonumber\\
    r_{3} &= 288
        \nonumber\\
    s_{3} &= s_{1} - q_{2}s_{2}
            \nonumber\\
        &= -1
            \nonumber\\
    t_{3} &= t_{1} - q_{2}t_{2}
            \nonumber\\
        &= 2.
\end{align}

\noindent
The following step gives

\begin{align}
    792 &= 2\cdot288 + 216.
        \nonumber\\
    q_{3} &= 2
        \nonumber\\
    r_{4} &= 216
        \nonumber\\
    s_{4} &= s_{2} - q_{3}s_{3}
            \nonumber\\
        &= 3
            \nonumber\\
    t_{4} &= t_{2} - q_{3}t_{3}
            \nonumber\\
        &= -5.
\end{align}

\noindent
Continuing on, we find

\begin{align}
    288 &= 1\cdot216 + 72
        \nonumber\\
    q_{4} &= 1
        \nonumber\\
    r_{5} &= 72
        \nonumber\\
    s_{5} &= s_{3} - q_{4}s_{4}
            \nonumber\\
        &= -4
            \nonumber\\
    t_{5} &= t_{3} - q_{4}t_{4}
            \nonumber\\
        &= 7.
\end{align}

\noindent
At last we have

\begin{align}
    216 &= 3\cdot 72
        \nonumber\\
    q_{5} &= 3
        \nonumber\\
    r_{6} &= 0.
\end{align}

\noindent
This concludes Alg.~\ref{alg:extended_euclidean_alg}.

The process has terminated and we have determined that

\begin{equation}
    \gcd(1872,1080) = 72
\end{equation}

\noindent
and

\begin{equation}
    1872\cdot\parens{-4} + 1080\cdot 7 = \gcd(1872,1080).
\end{equation}
\end{example}

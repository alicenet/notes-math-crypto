\section{Threshold Signatures}
\label{sec:ss_threshold}

\subsection{Deriving the Master Secret Key from the Group Secret Keys}

Now that we have constructed the master public key,
we will focus on computing group signatures.
We let $\mathcal{R}\subseteq\mathcal{P}$ be subset such that
$\abs{\mathcal{R}} = t+1$.

We recall the definition of $\gsk{}_{j}$ in Eq.~\eqref{eq:dkg_gskj_def}:

\begin{equation}
    \gsk{}_{j} = \sum_{P_{i}\in\mathcal{P}} s_{i\to j}.
\end{equation}

\noindent
We remember that $s_{i\to j}$ is the secret share
from $P_{i}$ to $P_{j}$.
This secret share $s_{i\to j}$ is an evaluation of $P_{i}$'s private polynomial.
Written another way, we have

\begin{equation}
    \gsk{}_{j} = \sum_{P_{i}\in\mathcal{P}} f_{i}(j).
\end{equation}

To make valid group signatures, we need to combine signatures
signed by the $\gsk{}_{j}$ keys.
In order to understand how this works, we will show how we can
derive the master secret key from the group secret keys.

We begin by recalling the definition of $\msk{}$
from Eq.~\eqref{eq:dkg_msk_def}:

\begin{equation}
    \msk{} = \sum_{P_{i}\in\mathcal{P}} s_{i}.
\end{equation}

\noindent
From Eq.~\eqref{eq:dkg_private_poly} and the surrounding discussion,
we know that

\begin{equation}
    s_{i} = f_{i}(0).
\end{equation}

\noindent
This implies that $\msk{}$ can be rewritten in terms of
evaluations of a private polynomials:

\begin{equation}
    \msk{} = \sum_{P_{i}\in\mathcal{P}} f_{i}(0).
\end{equation}

As we have stated before, during the share distribution phase
$P_{i}$ performed Shamir's Secret Sharing.
Given $\mathcal{R}$ as before, this means that the members of $\mathcal{R}$
can work together to compute

\begin{align}
    f_{i}(0) &= \sum_{P_{j}\in\mathcal{R}} f_{i}(j)R_{j} \nonumber\\
    R_{j} &= \prod_{\substack{P_{k}\in\mathcal{R} \\ k\ne j}}
        \frac{k}{k-j}.
\end{align}

\noindent
This is coming from Chapter~\ref{sec:ss_shamir};
\emph{note the indices are different.}
Rewriting this, we have

\begin{align}
    s_{i} &= \sum_{P_{j}\in\mathcal{R}} s_{i\to j}R_{j} \nonumber\\
    R_{j} &= \prod_{\substack{P_{k}\in\mathcal{R} \\ k\ne j}}
        \frac{k}{k-j}.
\end{align}

\noindent
If we sum over $P_{i}\in\mathcal{P}$, then we have

\begin{align}
    \msk{} &= \sum_{P_{i}\in\mathcal{P}} s_{i} \nonumber\\
        &= \sum_{P_{i}\in\mathcal{P}}
            \brackets{\sum_{P_{j}\in\mathcal{R}} s_{i\to j}R_{j}}
                \nonumber\\
        &= \sum_{P_{j}\in\mathcal{R}}
            \brackets{\sum_{P_{i}\in\mathcal{P}} s_{i\to j}}R_{j}
                \nonumber\\
        &= \sum_{P_{j}\in\mathcal{R}} \gsk{}_{j}R_{j}.
\end{align}

\noindent
We can use this equality to construct valid group signatures.

In more technical mathematical language, we are performing
\gls{lagrange interpolation} over a \gls{finite field}.

\subsection{Constructing Valid Group Signatures}

We can now use the work from the previous section to show
how we are able to combine partial signatures into a valid group signature.

The group wants to sign the message $m\in\braces{0,1}^{*}$.
We suppose that $\mathcal{R}\subseteq\mathcal{P}$ be as before;
that is, we have $\abs{\mathcal{R}} = t+1$.
We now suppose that they have the individual signatures

\begin{equation}
    \sigma_{j} \mathDef{} \brackets{H(m)}^{\gsk{}_{j}}.
\end{equation}

\noindent
We define the constants

\begin{equation}
    R_{j} = \prod_{\substack{P_{k}\in\mathcal{R} \\ k\ne j}}
        \frac{k}{k-j}.
\end{equation}

\noindent
We set

\begin{equation}
    \sigma \mathDef{} \prod_{P_{j}\in\mathcal{R}} \sigma_{j}^{R_{j}}.
\end{equation}

This is a valid group signature of $m$ under the master public key $\mpk{}$.
To see this, we note

\begin{align}
    e\parens{\sigma, h_{2}}
        &= e\parens{\prod_{P_{j}\in\mathcal{R}}\sigma^{R_{j}}, h_{2}}
            \nonumber\\
        &= \prod_{P_{j}\in\mathcal{R}} e\parens{\sigma^{R_{j}}, h_{2}}
            \nonumber\\
        &= \prod_{P_{j}\in\mathcal{R}} e\parens{H(m)^{\gsk{}_{j}R_{j}}, h_{2}}
            \nonumber\\
        &= \prod_{P_{j}\in\mathcal{R}} e\parens{H(m), h_{2}^{\gsk{}_{j}R_{j}}}
            \nonumber\\
        &= e\parens{H(m), \prod_{P_{j}\in\mathcal{R}} h_{2}^{\gsk{}_{j}R_{j}}}
            \nonumber\\
        &= e\parens{H(m), h_{2}^{\sum_{P_{j}\in\mathcal{R}} \gsk{}_{j}R_{j}}}
            \nonumber\\
        &= e\parens{H(m), h_{2}^{\msk{}}}
            \nonumber\\
        &= e\parens{H(m), \mpk{}}.
\end{align}

\noindent
Thus, we have a valid group signature from combining individual
signatures in a deterministic manner.
The constants only depend on which partial signatures are used
in the construction.
Additionally, they individually may be validated before combining them
because the $\gpk{}_{j}$ are public knowledge.
It may be verified by ensuring

\begin{equation}
    \PairingCheck{}\parens{\sigma, h_{2}^{-1}, H(m), \mpk{}} = \texttt{true}.
\end{equation}

In more technical mathematical language, we are performing
\gls{lagrange interpolation} over a group parameterized by elements
of a \gls{finite field}.

\section{Distributed Key Generation Overview and Setup}
\label{sec:ss_dkg}

In this section we consider the problem of
\emph{\gls{distributed key generation}}.
Here, a collection of individuals want to distribute a private key
between them in such a way that a certain portion must work
together to derive it.
This secret key will be used to digitally sign messages.
Because of the nature of the secret key,
it is \emph{imperative} that the key not actually be constructed;
rather, portions of the secret key should be used to construct
partial signatures which may then be combined in a deterministic way
to arrive at a valid \emph{group signature}.

The basis for this discussion is the Ethereum Distributed Key Generation
whitepaper~\cite{ethdkg}.
This paper uses a variation of 
Feldman's VSS scheme~\cite{feldman1987practical}
while making adjustments
based on~\cite{gennaro1999secure,gennaro2002revisiting} for security.
We seek a $\parens{t,n}$-threshold system
whereby $t+1$ participants are required to create a valid group signature.

We will focus on the actual implementation~\cite[Section 7]{ethdkg}.
While this will make our discussion specific to this particular setting,
the main ideas of \gls{distributed key generation} protocols should be clear.
Doing this will require the use of functions particular to \gls{ethereum};
see Table~\ref{table:ethereum_functions} for a description of them.

The material here will require knowledge of
\gls{pairingcrypto} from Chapter~\ref{chap:pairing} in general
and knowledge of cryptography on \gls{ethereum}~\cite{EthereumYellowpaper}
in Chapter~\ref{sec:pairing_ethereum} in particular.
Due to the fact that \glspl{smart contract} can only perform
\gls{elliptic curve}
operations in $G_{1}$, information in $G_{2}$ will be submitted
and then verified with a \PairingCheck{} call.

\begin{table}[t]
\centering
\begin{tabular}{r|l}
\hline
Function Name & Function Definition \\
\hline
\texttt{toBytes} & serialize object to its binary representation \\
\texttt{fromBytes} & deserialize object from its binary representation \\
\textsc{ECAdd} & Elliptic curve addition \\
\textsc{ECMult} & Elliptic curve scalar multiplication \\
\textsc{PairingCheck} & Validate bilinear pairing tuple \\
\hline
\end{tabular}
\caption[Ethereum Functions]{Useful functions in \gls{ethereum} for
    the \gls{distributed key generation} protocol.}
\label{table:ethereum_functions}
\end{table}


The $i$th participant will be denoted $P_{i}$
and $\mathcal{P} \mathDef{} \braces{P_{i}}_{i=1}^{n}$
will the collection of all participants.
By working together, the participants will construct
a \emph{master public key} $\mpk{}$ along with its corresponding
\emph{master secret key} $\msk{}$.
The master secret key will \emph{never} explicitly be formed;
even so, the group will be able to work together to combine partial signatures
into valid group signatures.
Threshold signatures will be discussed in Chapter~\ref{sec:ss_threshold}.

\subsection{DKG Overview}

Broadly speaking, in the \gls{distributed key generation}  protocol,
each participant is a dealer
who shares a secret key under a threshold.
Thus, participants perform \emph{simultaneous}
Shamir's Secret Sharing protocols.
The secret key for the entire group (the \emph{master secret key})
will be the sum of the individual secrets.
In order to protect against misbehavior,
this protocol contains opportunities to prove that other
participants are malicious.

\subsection{DKG Setup}

During the protocol,
we will extensively use the \gls{bilinear} $e:G_{1}\times G_{2}\to G_{T}$.
Here, we assume $\abs{G_{1}} = \abs{G_{2}} = \abs{G_{T}} = q$ for prime $q$.
We are assuming that $g_{1}, h_{1}\in G_{1}$ are independent generators;
that is, $\dlog_{g_{1}}h_{1}$ is unknown.
We also assume that $h_{2}\in G_{2}$ is a generator.
Furthermore, although not used here, we assume $H:\braces{0,1}^{*}\to G_{1}$
is a \gls{hash function} onto $G_{1}$;
this hash-to-curve function will be used when making \glspl{signature}.
These signatures will be BLS signatures~\cite{BLSSignatures}.



\section{Distributed Key Generation Protocol}
\label{sec:ss_dkg_protocol}

The \gls{distributed key generation}  protocol is split into a number
of different phases.
We begin by giving an overview of each phase before discussing them
in detail.

We note that there are variations of communication models
for different DKG protocols.
Some assume secure communication between each pair of participants
while others broadcast encrypted messages.
The particulars matter in that the communication model determines how
proofs of security are performed, but that is not our primary focus;
the goal here is to give an \emph{introduction} to DKG protocols.
In our case, we will assume the ability to broadcast messages
to all participants.

Additionally, we will assume the \gls{distributed key generation} protocol
restarts upon any malicious behavior.
While this is not strictly necessary, this may be useful in certain situations.

\subsection*{DKG Protocol Overview}

\begin{itemize}
\item \hyperref[ssec:secret_dkg_registration]{\textbf{\Registration{}:}}
    During this phase the participants register public keys
    to enable secure communication throughout the protocol.
\item \hyperref[ssec:secret_dkg_share_submission]{\textbf{\ShareSubmission{}:}}
    Each participant submits encrypted shares and commitments
    which are then broadcast to all participants.
\item \hyperref[ssec:secret_dkg_share_dispute]{\textbf{\ShareDispute{}:}}
    Anyone who incorrectly submitted his secret share information
    may be accused.
\item \hyperref[ssec:secret_dkg_key_share]{\textbf{\KeyShareSubmission{}:}}
    All users submit their portion of the master public key.
\item \hyperref[ssec:secret_dkg_mpk]{\textbf{\MPKSubmission{}:}}
    One user submits the master public key.
\item \hyperref[ssec:secret_dkg_gpk_submission]{\textbf{\GPKSubmission{}:}}
    All users submit their group public key.
\item \hyperref[ssec:secret_dkg_gpk_dispute]{\textbf{\GPKDispute{}:}}
    Anyone who incorrectly shared his group public key may be accused.
\item \hyperref[ssec:secret_dkg_completion]{\textbf{\Completion{}:}}
    The DKG process has finished.
    At this point, all information is valid and may be used to compute
    valid group signatures.
\end{itemize}

\subsection{\Registration{}}
\label{ssec:secret_dkg_registration}

Participant $P_{i}\in\mathcal{P}$ will choose a secret key
$\sk{}_{i}\chooseRandom{}\F_{q}^{*}$;
this will determine the corresponding public key

\begin{equation}
    \pk{}_{i} \mathDef{} g_{1}^{\sk{}_{i}}.
\end{equation}

\noindent
This pair $\parens{\sk{}_{i},\pk{}_{i}}$ \emph{will not}
be used for signing;
rather, it will be used for secure communication over
an \gls{insecure channel}.

The public key $\pk{}_{i}$ will be submitted by participant $P_{i}$
and broadcast to all participants.
In this way, every participant will know $P_{i}$'s public key.

\subsection{\ShareSubmission{}}
\label{ssec:secret_dkg_share_submission}

As mentioned in the overview, each participant will now perform
Shamir's Secret Sharing protocol in parallel to share a secret.

Participant $P_{i}$ will choose $s_{i}\chooseRandom{}\F_{q}$ to be his secret.
He will then choose his private polynomial
$f_{i}:\F_{q}\to\F_{q}$:

\begin{equation}
    f_{i}(x) = c_{i,0} + c_{i,1}x + c_{i,2}x^{2} + \cdots + c_{i,t}x^{t}.
    \label{eq:dkg_private_poly}
\end{equation}

\noindent
Here, $c_{i,0} = s_{i}$ and $c_{i,j}\chooseRandom{}\F_{q}$
for $j\in\braces{1,\cdots,t}$.

The secret share from $P_{i}$ to $P_{j}$ is

\begin{equation}
    s_{i\to j} \mathDef{} f_{i}(j).
\end{equation}

\noindent
In order to ensure honesty, $P_{i}$ will have the commitments

\begin{equation}
    C_{i,j} \mathDef{} g_{1}^{c_{i,j}}.
\end{equation}

\noindent
These commitments then result in the corresponding public polynomial
$F_{i}:\F_{q}\to G_{1}$:

\begin{equation}
    F_{i}(x) = C_{i,0}C_{i,1}^{x}C_{i,2}^{x^{2}} \cdots C_{i,t}^{x^{t}}.
\end{equation}

\noindent
These public polynomials make it possible to prove that secrets
were correctly shared.
This is because we have

\begin{equation}
    F_{i}(j) = g_{1}^{f_{i}(j)}.
    \label{eq:dkg_valid_secret}
\end{equation}

In order to ensure the secret shares stay secret,
the shares are encrypted under a \gls{symmetric key encryption} scheme
based on the \gls{dhke};
the algorithms for encryption and decryption will be denoted
\Enc{} and \Dec{}.
Full algorithmic details may be found
\hyperref[sssec:secret_dkg_share_encryption]{here}.

The encrypted form of the secret share from $P_{i}$ to $P_{j}$
is given by

\begin{equation}
    \overline{\texttt{s}}_{i\to j} \mathDef{}
        \Enc{}\parens{\sk{}_{i},\pk{}_{j},s_{i\to j}, j}.
\end{equation}

\noindent
Using these encrypted shares, participant $P_{i}$ broadcasts
the following message:

\begin{equation}
    \braces{
    \overline{\texttt{s}}_{i\to 1},
    \overline{\texttt{s}}_{i\to 2},
    \cdots,
    \overline{\texttt{s}}_{i\to i-1},
    \overline{\texttt{s}}_{i\to i+1},
    \cdots,
    \overline{\texttt{s}}_{i\to n},
    C_{i,0}, C_{i,1}, \cdots, C_{i,t}}.
\end{equation}

\noindent
More explicitly, all of the necessary encrypted shares are sent
along with the coefficients of the public polynomial.
Implicitly, $P_{i}$ will also send the secret share $s_{i\to i}$
to himself.
A cryptographic hash of the encrypted shares and commitments
will be stored should an accusation need to be made in the future;
accusations will be discussed below.

At this time, all participants are expected to have received all
broadcast messages.
If any participant who registered failed to submit shares,
it is possible for the protocol to continue;
that is the path chosen in~\cite{ethdkg}.
It may also be the case that the protocol could restart.
The particular choice depends on the use case.

In our case, we will assume that all participants submit share information.
After receiving share information, each share must be decrypted and verified.
If $P_{j}$ received $\overline{\texttt{s}}_{i\to j}$ from $P_{i}$,
he then computes

\begin{equation}
    \hat{s}_{i\to j} \mathDef{}
        \Dec{}\parens{\sk{}_{j}, \pk{}_{i}, \overline{\texttt{s}}_{i\to j}, j}.
\end{equation}

\noindent
If the share is valid, we will have the following equality:

\begin{equation}
    g_{1}^{\hat{s}_{i\to j}} = F_{i}(j).
    \label{eq:dkg_share_equality}
\end{equation}

\noindent
This follows from Eq.~\eqref{eq:dkg_valid_secret}.
If the above equality does not hold,
then $P_{i}$ incorrectly shared his secret with $P_{j}$.
An accusation must be performed in the next phase.

\subsubsection{Secret Share Encryption Algorithm}
\label{sssec:secret_dkg_share_encryption}

We now discuss the \gls{encryption scheme} used.
This uses a \gls{shared secret} to derive bit stream
for an XOR cipher.
The index of the participant who \emph{receives} the message is included
in order to ensure that each stream is unique.
It requires the use of a \gls{hash function} with output the size
of $\log_{2}(q)$;
that is, if $q$ is a $k$-bit number, we require the output
of the \gls{hash function} to be at least $k$ bits.
See Alg.~\ref{alg:dkg_encryption} for the full specification.

\begin{algorithm}[t]
\caption{\Glsentrytext{encryption scheme} in the \glsentrytext{distributed key generation} protocol}
\label{alg:dkg_encryption}
\begin{algorithmic}[1]
\Procedure{\Enc{}}{$\sk{}_{i}$,$\pk{}_{j}$,$s_{i\to j}$,$j$}
    \State $k \algAssign{} \pk{}_{j}^{\sk{}_{i}}$
    \Comment{Compute the \gls{dhke} \gls{shared secret}}
    \State $\texttt{kX} \algAssign{} \texttt{toBytes}(k_{x})$
    \Comment{Use the $x$ coordinate}
    \State $\texttt{j} \algAssign{} \texttt{toBytes}(j)$
    \State $\texttt{s} \algAssign{} \texttt{toBytes}(s_{i\to j})$
    \State $\texttt{hashKJ} \algAssign{} \textsc{Hash}(\texttt{kX}\|\texttt{j})$
    \State $\overline{\texttt{s}}_{i\to j} \algAssign{}
        \texttt{s} \oplus \texttt{HashKJ}$
    \State \Return $\overline{\texttt{s}}_{i\to j}$
\EndProcedure
\State
\Procedure{\Dec{}}{$\sk{}_{i}$,$\pk{}_{j}$,$\overline{\texttt{s}}_{i\to j}$,$j$}
    \State $k \algAssign{} \pk{}_{j}^{\sk{}_{i}}$
    \Comment{Compute the \gls{dhke} \gls{shared secret}}
    \State $\texttt{kX} \algAssign{} \texttt{toBytes}(k_{x})$
    \Comment{Use the $x$ coordinate}
    \State $\texttt{j} \algAssign{} \texttt{toBytes}(j)$
    \State $\texttt{hashKJ} \algAssign{} \textsc{Hash}(\texttt{kX}\|\texttt{j})$
    \State $\texttt{s} \algAssign{}
        \overline{\texttt{s}}_{i\to j} \oplus \texttt{HashKJ}$
    \State $s_{i\to j} \algAssign{} \texttt{fromBytes}(\texttt{s})$
    \State \Return $s_{i\to j}$
\EndProcedure
\State
\Procedure{\Dec{}SS}{$k$,$\overline{\texttt{s}}_{i\to j}$,$j$}
    \State $\texttt{kX} \algAssign{} \texttt{toBytes}(k_{x})$
    \Comment{Use the $x$ coordinate}
    \State $\texttt{j} \algAssign{} \texttt{toBytes}(j)$
    \State $\texttt{hashKJ} \algAssign{} \textsc{Hash}(\texttt{kX}\|\texttt{j})$
    \State $\texttt{s} \algAssign{}
        \overline{\texttt{s}}_{i\to j} \oplus \texttt{HashKJ}$
    \State $s_{i\to j} \algAssign{} \texttt{fromBytes}(\texttt{s})$
    \State \Return $s_{i\to j}$
\EndProcedure
\end{algorithmic}
\end{algorithm}



\subsection{\ShareDispute{}}
\label{ssec:secret_dkg_share_dispute}

We now assume that Eq.~\eqref{eq:dkg_share_equality} does not hold;
that is, $P_{i}$ sent $P_{j}$ an invalid share.
We now require that $P_{j}$ \emph{prove} that $P_{i}$ is malicious.

We know from Eq.~\eqref{eq:dkg_valid_secret} what the desired
equality should be.
Thus, to prove $P_{i}$'s share is invalid,
we must decrypt the encrypted share and then show the desired equality
of Eq.~\eqref{eq:dkg_share_equality} does not hold.
From the decryption algorithm in Alg.~\ref{alg:dkg_encryption},
we know that we must compute the Diffie-Hellman shared secret.

This Diffie-Hellman shared secret $k_{ij}$ can easily be computed.
The challenge is how can we \emph{prove} this is,
in fact, the shared secret?
We recall that we have the following equalities:

\begin{align}
    g_{1}^{\sk_{j}} &= \pk{}_{j} \nonumber\\
    \pk{}_{i}^{\sk_{j}} &= k_{ij}.
\end{align}

\noindent
Because the public keys $\pk{}_{i}$ and $\pk{}_{j}$ are public knowledge,
if we can show that the pairs $\parens{g_{1},\pk_{j}}$ and
$\parens{\pk{}_{i}, k}$ have the same \gls{discrete log},
then this \emph{proves} that $k$ is actually the shared secret $k_{ij}$.
To do this, we can use \gls{discrete log} equality proofs
from Chapter~\ref{sec:zkproofs_2_dlog};
see Alg.~\ref{alg:2_dlog_equality} for full details.

\begin{algorithm}[t]
\caption{Discrete logarithm equality proof and validation}
\label{alg:2_dlog_equality}
\begin{algorithmic}[1]
\Require Group order $q$
\Procedure{DLEQ-Proof}{$x_{1}$,$y_{1}$,$x_{2}$,$y_{2}$,$\alpha$}
    \State $w \chooseRandom{} \Z_{q}^{*}$
    \Comment{Generate proof that $y_{1}=x_{1}^{\alpha}$ and
        $y_{2}=x_{2}^{\alpha}$ for secret $\alpha$}
    \State $t_{1} \algAssign{} x_{1}^{w}$
    \State $t_{2} \algAssign{} x_{2}^{w}$
    \State $c \algAssign{} \textsc{Hash}(x_{1},y_{1},x_{2},y_{2},t_{1},t_{2})$
    \State $r \algAssign{} w - c\alpha \mod q$
    \State $\pi \algAssign{} \parens{c,r}$
    \State \Return $\pi$
\EndProcedure
\State
\Procedure{DLEQ-Verify}{$x_{1}$,$y_{1}$,$x_{2}$,$y_{2}$,$\pi$}
    \Comment{Validate that $y_{1}=x_{1}^{\alpha}$ and $y_{2}=x_{2}^{\alpha}$}
    \State $t_{1}' \algAssign{} y_{1}^{c}x_{1}^{r}$
    \State $t_{2}' \algAssign{} y_{2}^{c}x_{2}^{r}$
    \State $c' \algAssign{}\textsc{Hash}(x_{1},y_{1},x_{2},y_{2},t_{1}',t_{2}')$
    \If{$c' = c$}
        \State \Return \textsf{True}
    \Else
        \State \Return \textsf{False}
    \EndIf
\EndProcedure
\end{algorithmic}
\end{algorithm}


Thus, $P_{j}$ can broadcast the Diffie-Hellman shared secret
$k_{ij}$ along with \gls{discrete log} proof $\pi(k_{ij})$
which may be verified by everyone.
At this point, all users may decrypt $\overline{\texttt{s}}_{i\to j}$
using $\Dec{}SS$:

\begin{equation}
    \hat{s}_{i\to j} \mathDef{} 
        \Dec{}SS\parens{k_{ij},\overline{\texttt{s}}_{i\to j}, j}.
\end{equation}

\noindent
Everyone can verify that

\begin{equation}
    g_{1}^{\hat{s}_{i\to j}} \ne F_{i}(j).
\end{equation}

\noindent
This proves that $P_{i}$ shared an invalid share with $P_{j}$;
this is a malicious action.

When this is implemented on \gls{ethereum},
$P_{i}$'s encrypted shares and commitments would need to be resubmitted.
After confirming their validity by rehashing the data,
all accusation logic may be performed by a \gls{smart contract}.
If we do not have equality Eq.~\eqref{eq:dkg_share_equality},
then $P_{j}$ just proved $P_{i}$ is malicious
because he submitted an invalid share;
if we have equality Eq.~\eqref{eq:dkg_share_equality},
then $P_{j}$ is malicious for submitting an invalid accusation.


\subsection{\KeyShareSubmission{}}
\label{ssec:secret_dkg_key_share}

If no one submitted any invalid shares,
then the next step is for each participant to submit his portion
of the master public key.

To do this, $P_{i}$ must submit $h_{1}^{s_{i}}$ along with proof

\begin{equation}
    \pi(h_{1}^{s_{i}}) \mathDef{}
        \textsc{DLEQ-Proof}(g_{1},g_{1}^{s_{i}},h_{1},h_{1}^{s_{i}},s_{i})
\end{equation}

\noindent
and $h_{2}^{s_{i}}$.
This is possible because $g_{1}^{s_{i}} = C_{i,0}$ is public knowledge.

A \gls{smart contract} call will confirm the validity of $h_{1}^{s_{i}}$
given $\pi(h_{1}^{s_{i}})$.
A \PairingCheck{} will confirm the validity of $h_{2}^{s_{i}}$
by requiring

\begin{equation}
    \PairingCheck{}(h_{1}^{s_{i}},h_{2}^{-1},h_{1},h_{2}^{s_{i}})
        = \texttt{true}.
\end{equation}

\noindent
The values $h_{1}^{s_{i}}$, $\pi(h_{1}^{s_{i}})$, and $h_{2}^{s_{i}}$
are then broadcast to all users.
The values $\braces{h_{1}^{s_{i}}}_{P_{i}\in\mathcal{P}}$ are stored
for the next phase.


\subsection{\MPKSubmission{}}
\label{ssec:secret_dkg_mpk}

After all key shares have been submitted, we can compute
the \emph{master public key} $\mpk{}$:

\begin{equation}
    \mpk{} \mathDef{} \prod_{P_{i}\in\mathcal{P}} h_{2}^{s_{i}}.
\end{equation}

\noindent
The corresponding \emph{master secret key} $\msk{}$ is defined
similarly:

\begin{equation}
    \msk{} \mathDef{} \sum_{P_{i}\in\mathcal{P}} s_{i}.
    \label{eq:dkg_msk_def}
\end{equation}

The $\mpk{}$ may be validated upon submission because we assume
that $\braces{h_{1}^{s_{i}}}_{P_{i}\in\mathcal{P}}$ are stored.
Then, we can form

\begin{equation}
    \mpk{}^{*} \mathDef{} \prod_{P_{i}\in\mathcal{P}} h_{1}^{s_{i}}
\end{equation}

\noindent
and then confirm

\begin{equation}
    \PairingCheck{}(\mpk{}^{*},h_{2}^{-1},h_{1},\mpk{}) = \texttt{true}.
\end{equation}

The reason we should not use $\prod_{P_{i}\in\mathcal{P}} g_{1}^{s_{i}}$
as the master public key is because the $g_{1}^{s_{i}}$ values
are public knowledge after the initial share.
Because of this, some participants could attempt to bias the distribution
of $\prod_{P_{i}\in\mathcal{P}} g_{1}^{s_{i}}$ based on the order
of submission;
see~\cite{gennaro1999secure,gennaro2002revisiting} for further discussion.


\subsection{\GPKSubmission{}}
\label{ssec:secret_dkg_gpk_submission}

At this point, we have constructed the $\mpk{}$ and distributed
the $\msk{}$ between all participants.
We now focus on how to compute valid group signatures.
Doing this requires signing under a special private key.
These partial signatures may then be combined in a deterministic
way to form a valid group signature.

Participant $P_{j}$ has a \emph{group secret key} $\gsk{}_{j}$:

\begin{equation}
    \gsk{}_{j} \mathDef{} \sum_{P_{i}\in\mathcal{P}} s_{i\to j}.
    \label{eq:dkg_gskj_def}
\end{equation}

\noindent
There is also the corresponding \emph{group public key} $\gpk{}_{j}$:

\begin{equation}
    \gpk{}_{j} \mathDef{} h_{2}^{\gsk{}_{j}}.
\end{equation}

\noindent
Based on the definition of $\gsk{}_{j}$, we have the following:

\begin{align}
    g_{1}^{\gsk{}_{j}} &= g_{1}^{\sum_{P_{i}\in\mathcal{P}} s_{i\to j}}
        \nonumber\\
    &= \prod_{P_{i}\in\mathcal{P}} g_{1}^{s_{i\to j}}
        \nonumber\\
    &= \prod_{P_{i}\in\mathcal{P}} g_{1}^{f_{i}(j)}
        \nonumber\\
    &= \prod_{P_{i}\in\mathcal{P}} F_{i}(j).
    \label{eq:dkg_gpkj_star_derivation}
\end{align}

\noindent
In this way, $g_{1}^{\gsk{}_{j}}$ may be constructed from public information.

We now define

\begin{equation}
    \gpk{}_{j}^{*} \mathDef{} \prod_{P_{i}\in\mathcal{P}} F_{i}(j).
    \label{eq:dkg_gpkj_star}
\end{equation}

\noindent
The definition of $\gpk{}_{j}^{*}$ in Eq.~\eqref{eq:dkg_gpkj_star}
along with the derivation in Eq.~\eqref{eq:dkg_gpkj_star_derivation}
shows us that

\begin{equation}
    \gpk{}_{j}^{*} = g_{1}^{\gsk{}_{j}}.
\end{equation}

\noindent
This implies that $\gpk{}_{j}$ is a valid group public key if

\begin{equation}
    \PairingCheck{}(\gpk{}_{j}^{*},h_{2}^{-1},g_{1},\gpk{}_{j})
        = \texttt{true}.
\end{equation}

\noindent
This fact gives us the ability to verify all $\gpk{}_{j}$ submissions.


\subsection{\GPKDispute{}}
\label{ssec:secret_dkg_gpk_dispute}

Given a submission $\gpk{}_{j}$, all parties can compute $\gpk{}_{j}^{*}$
from Eq.~\eqref{eq:dkg_gpkj_star} and verify

\begin{equation}
    \PairingCheck{}(\gpk{}_{j}^{*},h_{2}^{-1},g_{1},\gpk{}_{j})
        = \texttt{true}.
\end{equation}

\noindent
This logic may also be performed by a \gls{smart contract},
although care must be taken to ensure efficient computation.


\subsection{\Completion{}}
\label{ssec:secret_dkg_completion}

After allowing sufficient time for accusations of malicious
$\gpk{}_{j}$ submissions,
the \gls{distributed key generation} process is complete.

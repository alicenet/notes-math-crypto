\chapter{Computational Hardness Assumptions}
\label{chap:hardness}

In this chapter we discuss some of the hardness assumptions inherent
to \gls{publiccrypto}.
We will focus on those related to \gls{discrete log} systems
because they are important to \gls{ecc}.

When talking about computational infeasibility, we assume that it is
not possible to do $2^{128}$ operations.
Table~\ref{table:key_lengths} describes key length requirements
and has been included from~\cite[Page 381]{IntroModernCrypto};
the notation has been slightly modified to match ours.

\begin{table}[t]
\centering
\begin{tabular}{c|c|c}
\hline
\multirowthead{3}{Effective \\ Key Length} &
    \multicolumn{2}{c}{\thead{Discrete Logarithm}} \\
\cline{2-3}
& \thead{Order-$q$ \\ Subgroup of $\F_{p}^{*}$} &
    \thead{Elliptic Curve \\ Group Order $q$} \\
\hline
112 & $p$: 2048,  $q$: 224 & 224 \\
128 & $p$: 3072,  $q$: 256 & 256 \\
192 & $p$: 7680,  $q$: 384 & 384 \\
256 & $p$: 15360, $q$: 512 & 512 \\
\hline
\end{tabular}
\caption[Key Length Estimates]{Estimated key lengths for specified
    security levels~\cite[Page 381]{IntroModernCrypto}.
    The effective key length determines the required number of operations.
    An effective key length of 128 bits means that
    it will take approximately $2^{128}$ operations to break the cryptosystem.
    As we can see, the order of \gls{elliptic curve} \glspl{group}
    are \emph{significantly smaller} than those required for
    \glspl{subgroup} of $\F_{p}$.}
\label{table:key_lengths}
\end{table}




\section{Discrete Logarithm Problem}
\label{sec:hardness_dlp}

\subsection{Formal Definition}

\begin{defn}[Discrete Logarithm Problem]
Given a \gls{finite cyclic group} $G = \angles{g}$ and $h\chooseRandom{}G$,
compute $x$ such that

\begin{equation}
    h = g^{x}.
\end{equation}
\end{defn}

\subsection{General Methods for Solving DLP}

We assume that $\abs{G} = n$ is a $k$-bit number.
There are some well known general methods for solving the \gls{dlp}.
These include

\begin{itemize}
\item Baby Step, Giant Step:
    the total cost is $O(2^{k/2})$ space and $O(2^{k/2})$ time.
\item Pollard's Rho:
    the total cost is $O(1)$ space and expected $O(2^{k/2})$ time;
    this is a probabilistic algorithm.
\end{itemize}

\noindent
Both of these algorithms are \emph{exponential} algorithms;
the constants hidden in the $O(\cdot)$ are small.
They are also called \emph{square root} algorithms because
the cost is approximately the square root of $\abs{G}$.
This implies that if we want $k$-bit security,
then $\abs{G}$ must be a $2k$-bit number.
These methods are discussed in~\cite[Chapter 10]{IntroModernCrypto}.

If $n$ is not prime, it is sometimes possible to speed up
these attacks even further.
Because of this, secure algorithms should use \glspl{cyclic group} whose size
is prime or which contains a \emph{large} prime factor.

\subsection{Discussion}

When we are designing a cryptographic system which relies
on the \gls{dlp} for security,
we must ensure that it is impractical to solve using all known methods.
Thus, the generic methods described ensure that $\abs{G}$ must be
at least $2^{2k}$ for $k$-bit security.
This estimate only holds when these generic algorithms are the best
algorithms.

When solving the \gls{dlp} in $\F_{p}$, though, better methods exist.
One such method is the \emph{index calculus algorithm}
and its complexity is
\emph{subexponential} in $\log_{2}(p)$~\cite[Chapter 10]{IntroModernCrypto}.
Because of this, key lengths in $\F_{p}$ must be \emph{significantly}
larger.
From Table~\ref{table:key_lengths}, we see that
$p$ must be a 3072-bit prime for 128-bit security.

We can contrast this with solving the \gls{dlp} on \glspl{subgroup}
of \glspl{elliptic curve}.
There are no known algorithms which beat the generic algorithms
when solving DLP over \glspl{elliptic curve}
(with some well-known exceptions).
From Table~\ref{table:key_lengths}, we see that
we need to choose a 256-bit prime $p$ when we want 128-bit security.
This means that public keys will be much smaller in cryptosystems
built using \glspl{elliptic curve} than $\F_{p}$.



\section{Diffie-Hellman Problems}

\subsection{Formal Definitions}

\begin{defn}[Computational Diffie-Hellman Problem]
Fix a \gls{finite cyclic group} $G = \angles{g}$ of order $q$
and $a,b\chooseRandom{}\Z_{q}$.
Let $A = g^{a}$ and $B = g^{b}$.
Given $A$ and $B$, compute $g^{ab}$.
\end{defn}

\begin{defn}[Decisional Diffie-Hellman Problem]
Fix a \gls{finite cyclic group} $G = \angles{g}$ of order $q$
and $a,b,c\chooseRandom{}Z_{q}$.
Let $A = g^{a}$, $B = g^{b}$, and $C = g^{c}$.
Distinguish $g^{ab}$ and $C$.
\end{defn}

\subsection{General Methods for Solving CDH and DDH}

It is clear that \emph{if} we can solve DLP, then we can solve CDH.
This follows because if we can solve DLP, then we could solve for $a$ in

\begin{equation}
    A = g^{a}
\end{equation}

\noindent
and then compute

\begin{align}
    g^{ab} &= \parens{g^{b}}^{a} \nonumber\\
        &= B^{a}.
\end{align}

\noindent
It is not clear that if we can solve CDH then we can solve DLP;
this would imply that DLP and CDH are equivalent.
Here is a paper which discusses a variant of
CDH and its relationship to DLP~\cite{cryptoeprint:2004:306}

Similarly, we can see that if we can solve CDH then we can solve DDH.
The reverse is not thought
to be true~\cite[Pages 340--1]{IntroModernCrypto}.
Additionally, we note that DDH is easy when given a \gls{bilinear}.
A survey paper of DDH may be found in~\cite{boneh1998decision}.

\subsection{Additional Variations}

We mention that these two \glspl{dhp} (CDH and DDH)
are the primary DHPs.
There are variations on these which also show up in the literature,
but they are usually variations on these two problems.

\section{Fields}
\label{sec:math_fields}

\subsection{Why do we care about Fields?}
\Glspl{field} arise in many different areas.
In cryptography, we use \glspl{field} in \gls{dhke} and \glspl{signature}.
Additionally, \glspl{field} are used when working with \glspl{elliptic curve}.

\subsection{Intuition and Examples}
\emph{\Gls{field}} are common mathematical objects.
All \glspl{field} are \glspl{ring} but have additional properties.
Informally, a \gls{field} is a generalization of the rational numbers
with addition and multiplication.
In particular, every nonzero element has a multiplicative inverse.
We begin with some examples.

\begin{example}[The Rational Numbers]
The rational numbers are a standard example of a \gls{field}.
For $a\in\Q\setminus\braces{0}$, we can write

\begin{equation}
    a = \frac{m}{n}
\end{equation}

\noindent
for $m,n\in\Z\setminus\braces{0}$.
Then $\frac{n}{m}\in\Q$ and

\begin{equation}
    \frac{m}{n} \cdot \frac{n}{m} = 1.
\end{equation}

\noindent
Thus, $a$ has multiplicative inverse.
The rational numbers $\Q$ have an infinite number of elements.
\end{example}

\begin{example}[The Field $(\Z_{11},+,\cdot)$]
\exampleCodeReference{examples/math\_review/finite\_field\_inverses.py}

Let us look at $\Z_{11}$.
We have the following multiplication facts:

\begin{align}
    1\cdot 1 &= 1 \mod 11
        &
    6\cdot 2 &= 1 \mod 11 \nonumber\\
    2\cdot 6 &= 1 \mod 11
        &
    7\cdot 8 &= 1 \mod 11 \nonumber\\
    3\cdot 4 &= 1 \mod 11
        &
    8\cdot 7 &= 1 \mod 11 \nonumber\\
    4\cdot 3 &= 1 \mod 11
        &
    9\cdot 5 &= 1 \mod 11 \nonumber\\
    5\cdot 9 &= 1 \mod 11
        &
    10\cdot 10 &= 1 \mod 11.
\end{align}

\noindent
This shows that every nonzero element in $\Z_{11}$ has a multiplicative
inverse.
This, along with the other \gls{ring} properties, makes $\Z_{11}$ a \gls{field}.
In this case, we write it as $\F_{11}$ to emphasize the \gls{field} nature.
The \gls{field} $\F_{11}$ has a finite number of elements.
\end{example}

\subsection{Formal Definition}
\begin{defn}[Field]
A \gls{field} is a \gls{set} $F$ together with two binary operations
addition $+$ and multiplication $\cdot$ which have the following properties:

\begin{itemize}
\item $\parens{F,+}$ is an \gls{abelian group} with additive identity $0$.

\item $\parens{F^{*},\cdot}$ is an \gls{abelian group} with multiplicative
    identity $1$.
    Here, $F^{*} \mathDef{} F\setminus\braces{0}$.

\item We have $0\ne 1$.

\item We have the following distribution law between multiplication
    and addition.
    For all $a,b,c\in F$, we have

\begin{equation}
    a\cdot\parens{b + c} = \parens{a\cdot b} + \parens{a\cdot c}.
\end{equation}
\end{itemize}

\noindent
We formally say that $\parens{F,+,\cdot}$ is a \gls{field}.
\end{defn}

\subsection{Continued Discussion}

\subsubsection{Difference Between Rings and Fields}

We note that a \gls{field} is a \gls{commutative ring} with no zero divisors
and where every nonzero element has a multiplicative inverse.

\subsubsection{Finite Fields}

While the \glspl{field} of the rationals $\Q$, the reals $\R$,
and complex numbers $\C$ are more familiar,
in cryptography we will be particularly interested
in \emph{\glspl{finite field}}.
As one may guess, a \gls{field} $\F$ is \emph{finite} if the number of
elements is finite; that is, $\abs{\F}<\infty$.

The \glspl{finite field} we will focus on are $\F_{p}$;
here $p$ is a prime number.
We will look at more examples of these \glspl{field} below.
There are additional types of \glspl{finite field},
but the \glspl{finite field} $\F_{p}$ are our primary focus.
Additional material on \glspl{field} may be found in
Appendix~\ref{app:math_finite_fields}.

\subsection{More Examples}

\begin{example}[Integers modulo a prime]
We now show that $\parens{\F_{p},+,\cdot}$ is a \gls{field},
where

\begin{equation}
    \F_{p} \mathDef{} \braces{0, 1, 2, \ldots, p-1}
\end{equation}

\noindent
and $p$ is prime.
We let

\begin{equation}
    \F_{p}^{*} \mathDef{} \F_{p}\setminus\braces{0},
\end{equation}

\noindent
so $\F_{p}^{*}$ are the nonzero elements of $\F_{p}$.

Let $a\in\F_{p}^{*}$ for prime $p>2$.
Because $p$ is prime and $1\le a\le p-1$,
$\gcd(a,p) = 1$.
Thus, there exists $x$ and $y$ such that

\begin{equation}
    ax + py = 1.
\end{equation}

\noindent
If we look at the above modulo $p$, we see

\begin{equation}
    ax = 1 \mod p.
\end{equation}

\noindent
Therefore, $a$ has a multiplicative inverse $x\mod p$.
This holds for all $a\in\F_{p}^{*}$, so $\parens{\F_{p},+,\cdot}$
is a \gls{field}.
\end{example}

\begin{example}[More examples with $\F_{p}$]
There are some unusual properties that we will discuss about $\F_{p}$.
Throughout this example, all arithmetic will be performed modulo $p$.

For all $x\in\F_{p}$, we have

\begin{equation}
    p\cdot x = 0.
\end{equation}

\noindent
By this, we mean

\begin{equation}
    \underbrace{x + x + \cdots + x}_{\text{$p$ times}} = 0.
\end{equation}

\noindent
In particular, we have

\begin{equation}
    p\cdot 1 = 0;
\end{equation}

\noindent
that is,

\begin{equation}
    \underbrace{1 + 1 + \cdots + 1}_{\text{$p$ times}} = 0.
\end{equation}

\noindent
This is not the case in the more familiar \glspl{field}
of $\Q$, $\R$, and $\C$.
In particular,

\begin{equation}
    \underbrace{1 + 1 + \cdots + 1}_{\text{$n$ times}} = n
\end{equation}

\noindent
for every $n\in\N$.
It is not possible to add $1$ to itself and arrive at $0$.

Additionally, for $x,y\in\F_{p}$, we have

\begin{equation}
    \parens{x+y}^{p} = x^{p} + y^{p}.
\end{equation}

\noindent
This follows from the previous property and the Binomial Theorem%
\footnote{\url{https://en.wikipedia.org/wiki/Binomial_theorem}}.
Although this is fascinating, we will not discuss it further.
\end{example}

\begin{example}[Even more examples with $\F_{p}$]
\label{example:math_field_p_more}
\exampleCodeReference{examples/math\_review/finite\_field\_powers.py}

We now fix a value of $p$ in order to work on some examples.
We let

\begin{align}
    p &= 65537 \nonumber\\
        &= 2^{16} + 1.
\end{align}

Addition modulo $p$ is well understood.
We will take some time to look at examples of multiplication.
In particular, we will look at exponentiation.

We see

\begin{align}
    3^{0}  &= 1
        &
    3^{32}  &= 61869 \nonumber\\
    3^{1}  &= 3
        &
    3^{64}  &= 19139 \nonumber\\
    3^{2}  &= 9
        &
    3^{128}  &= 15028 \nonumber\\
    3^{4}  &= 81
        &
    3^{256}  &= 282 \nonumber\\
    3^{8}  &= 6561
        &
    3^{512}  &= 13987 \nonumber\\
    3^{16} &= 54449
        &
    3^{1024}  &= 8224.
\end{align}

\noindent
Once we reach a large enough exponent, say 16,
there does not appear to be any relationship between the exponent
$x$ and the resulting number $3^{x}\mod p$.

To get a \gls{group} from the \gls{field} $\F_{p}$,
we can look at the \gls{subgroup}
generated by 3 under multiplication;
that is, we can look at

\begin{equation}
    H \mathDef{} \braces{3^{x} \mod p \mid x\in\Z}.
\end{equation}

\noindent
Then $H\le \F_{p}^{*}$; that is, $H$ is a \gls{subgroup} of the \gls{group}
$\parens{\F_{p}^{*},\cdot}$.
In general, if we choose prime $p$ large enough,
we can use $H$ as a \gls{group} for the \gls{dhke};
this will be discussed in Chapter~\ref{sec:public_diffie_hellman}.
\end{example}

\subsection{Encoding Fields}

When working with the \gls{field} $\F_{p}$, we can encode $k\in\F_{p}$
by its binary representation.
Other \glspl{field} have different representations.

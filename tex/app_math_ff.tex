\section{Finite Fields}
\label{app:math_finite_fields}

Throughout this section, we will be looking at the \gls{finite field} $\F_{p}$.
We have the following theorems:

\begin{thm}[Fermat's Little Theorem~{\cite[Lemma 2.1.16]{mullen2013handbook}}]
\label{thm:app_math_Fermat_Little}
For all $a\in\F_{p}^{*}$, we have

\begin{equation}
    a^{p-1} = 1 \mod p.
\end{equation}
\end{thm}

\begin{thm}[$\F_{p}^{*}$ is cyclic~{\cite[Theorem 2.1.37]{mullen2013handbook}}]
\label{thm:app_math_Fp_cyclic}
Let $p$ be prime.
Then $\parens{\F_{p}^{*},\cdot}$ is a \gls{cyclic group} of order $p-1$.
\end{thm}

\subsection{Generating Primitive Elements}
\label{app:math_finite_fields_primitive}

Given the fact that $\F_{p}^{*}$ is a \gls{cyclic group},
we have the following definition:

\begin{defn}[Primitive Element]
We say $g\in\F_{p}^{*}$ is a \emph{primitive element}
if $g$ is a generator of $\F_{p}^{*}$.
\end{defn}

Theorem~\ref{thm:app_math_Fp_cyclic} does not \emph{specify}
a generator for $\F_{p}^{*}$;
it merely states that one \emph{exists}.
In Chapter~\ref{sec:zkproofs_group}, we stated without proof that
elements of $\F_{p}$ were primitive elements.
We will now give an algorithm to construct such generators;
see Alg.~\ref{alg:finite_field_generator}.

\begin{algorithm}[t]
\caption{Compute a primitive element of a \glsentrytext{finite field}}
\label{alg:finite_field_generator}
\begin{algorithmic}[1]
\Require $p$ prime
\Procedure{ComputeFiniteFieldGenerator}{$p$}
    \State $q \algAssign{} p-1$
    \State $p_{1}^{n_{1}}p_{2}^{n_{2}}\cdots p_{\ell}^{n_{\ell}}
        \algAssign{} \textsc{Factor}(q)$
    \Comment{Compute the prime factorization of $q=p-1$}
    \While{}
        \State $\alpha \chooseRandom{} \F_{p}^{*}$
        \Comment{Choose a random element in $\F_{p}^{*}$}
        \State $\texttt{IsGenerator} \algAssign{} \texttt{true}$
        \For {$k\algAssign{}1; k\le \ell; k{+}{+}$}
            \State $q_{k} \algAssign{} q/p_{k}$
            \State $\beta \algAssign{} \alpha^{q_{k}}$
            \If{$\beta = 1$}
                \State $\texttt{IsGenerator} \algAssign{} \texttt{false}$
                \Break
            \EndIf
        \EndFor
        \If{$\texttt{IsGenerator}$}
            \Break
        \EndIf
    \EndWhile
    \State \Return $\alpha$
\EndProcedure
\end{algorithmic}
\end{algorithm}


Proving that Alg.~\ref{alg:finite_field_generator} produces
a generator for $\F_{p}^{*}$ relies on this proposition:

\begin{prop}[Generators of Finite Cyclic Groups]
\label{prop:math_finite_cyclic_generator}
Let $G$ a \gls{finite cyclic group} of order $q$ with

\begin{equation}
    q = p_{1}^{n_{1}}p_{2}^{n_{2}}\cdots p_{\ell}^{n_{\ell}}.
\end{equation}

\noindent
Then $h\in G$ is a generator if and only if

\begin{equation}
    h^{q/p_{k}} \ne 1, \quad k\in\braces{1,\cdots,\ell}.
\end{equation}
\end{prop}

To use Alg.~\ref{alg:finite_field_generator}, we must compute
the prime factorization of $p-1$.
From here, we choose random elements until one satisfies
the necessary constraints.

\begin{example}[Computing Finite Field Generators 1]
\label{example:app_math_finite_generator_7919}
\exampleCodeReference{examples/app\_math/finite\_field\_generator\_1.py}

For the first example, we will compute some generators of $\F_{7919}^{*}$;
we have used this \gls{group} with generator $7$ many times.

We start with

\begin{equation}
    p = 7919.
\end{equation}

\noindent
We also see that

\begin{align}
    p - 1 &= 7918 \nonumber\\
        &= 2\cdot 37\cdot 107.
\end{align}

\noindent
We set

\begin{align}
    p_{1} &= 2 \nonumber\\
    p_{2} &= 37 \nonumber\\
    p_{3} &= 107.
\end{align}

\noindent
We know that $2$, $37$, and $107$ are prime numbers.

Instead of randomly generating elements, we will start at $2$
and proceed forward.
For ease of notation, we set

\begin{align}
    q_{1} &= \frac{\parens{p-1}}{p_{1}} \nonumber\\
        &= 3959 \nonumber\\
    q_{2} &= \frac{\parens{p-1}}{p_{2}} \nonumber\\
        &= 214 \nonumber\\
    q_{3} &= \frac{\parens{p-1}}{p_{3}} \nonumber\\
        &= 74.
\end{align}

\begin{itemize}
\item $2$: We see

\begin{align}
    2^{q_{1}} \mod p &= 1
        \nonumber\\
    2^{q_{2}} \mod p &= 5823
        \nonumber\\
    2^{q_{3}} \mod p &= 6451.
\end{align}

\noindent
Because $2^{q_{1}} = 1$, $2$ is not a generator of $\F_{p}^{*}$.

\item $3$: We see

\begin{align}
    3^{q_{1}} \mod p &= 1
        \nonumber\\
    3^{q_{2}} \mod p &= 4930
        \nonumber\\
    3^{q_{3}} \mod p &= 4828.
\end{align}

\noindent
Because $3^{q_{1}} = 1$, $3$ is not a generator of $\F_{p}^{*}$.

\item $4$: We see

\begin{align}
    4^{q_{1}} \mod p &= 1
        \nonumber\\
    4^{q_{2}} \mod p &= 6090
        \nonumber\\
    4^{q_{3}} \mod p &= 1056.
\end{align}

\noindent
Because $4^{q_{1}} = 1$, $4$ is not a generator of $\F_{p}^{*}$.

\item $5$: We see

\begin{align}
    5^{q_{1}} \mod p &= 1
        \nonumber\\
    5^{q_{2}} \mod p &= 6
        \nonumber\\
    5^{q_{3}} \mod p &= 48.
\end{align}

\noindent
Because $5^{q_{1}} = 1$, $5$ is not a generator of $\F_{p}^{*}$.

\item $6$: We see

\begin{align}
    6^{q_{1}} \mod p &= 1
        \nonumber\\
    6^{q_{2}} \mod p &= 1015
        \nonumber\\
    6^{q_{3}} \mod p &= 1.
\end{align}

\noindent
Because $6^{q_{1}} = 1$, $6$ is not a generator of $\F_{p}^{*}$.

\item $7$: We see

\begin{align}
    7^{q_{1}} \mod p &= 7918
        \nonumber\\
    7^{q_{2}} \mod p &= 755
        \nonumber\\
    7^{q_{3}} \mod p &= 5549.
\end{align}

\noindent
Because $7^{q_{i}} \ne 1$ for all $i$, $7$ is a generator of $\F_{p}^{*}$.
\end{itemize}

Thus, we were able to find 1 generator in the first 6 possibilities.
We note that the next generator is 13.
\end{example}


\begin{example}[Computing Finite Field Generators 2]
\label{example:app_math_finite_generator_zkproofs_group}
\exampleCodeReference{examples/app\_math/finite\_field\_generator\_2.py}

We now compute generators for the \gls{group}
from Chapter~\ref{sec:zkproofs_group}:

\begin{align}
    p &= rq + 1 \nonumber\\
    q &= 2^{32} + 15 \nonumber\\
    r &= 12.
\end{align}

\noindent
We see that we have the factorization

\begin{align}
    p-1 &= rq \nonumber\\
        &= \parens{12}\parens{2^{32} + 15} \nonumber\\
        &= 2^{2}\cdot 3 \cdot \parens{2^{32} + 15}.
\end{align}

\noindent
We accept the assertion that $2^{32}+15$ is prime
and set

\begin{align}
    p_{1} &= 2 \nonumber\\
    p_{2} &= 3 \nonumber\\
    p_{3} &= 2^{32} + 15.
\end{align}

Instead of randomly generating elements, we will start at $2$
and proceed forward.
For ease of notation, we set

\begin{align}
    q_{1} &= \frac{\parens{p-1}}{p_{1}} \nonumber\\
        &= 25769803866
            \nonumber\\
    q_{2} &= \frac{\parens{p-1}}{p_{2}} \nonumber\\
        &= 17179869244
            \nonumber\\
    q_{3} &= \frac{\parens{p-1}}{p_{3}} \nonumber\\
        &= 12.
\end{align}

\begin{itemize}
\item $2$: We see

\begin{align}
    2^{q_{1}} \mod p &= 51539607732
        \nonumber\\
    2^{q_{2}} \mod p &= 1
        \nonumber\\
    2^{q_{3}} \mod p &= 4096.
\end{align}

\noindent
Because $2^{q_{2}} = 1$, $2$ is not a generator of $\F_{p}^{*}$.

\item $3$: We see

\begin{align}
    3^{q_{1}} \mod p &= 1
        \nonumber\\
    3^{q_{2}} \mod p &= 5944319764
        \nonumber\\
    3^{q_{3}} \mod p &= 531441.
\end{align}

\noindent
Because $3^{q_{1}} = 1$, $3$ is not a generator of $\F_{p}^{*}$.

\item $4$: We see

\begin{align}
    4^{q_{1}} \mod p &= 1
        \nonumber\\
    4^{q_{2}} \mod p &= 1
        \nonumber\\
    4^{q_{3}} \mod p &= 16777216.
\end{align}

\noindent
Because $4^{q_{1}} = 1$, $4$ is not a generator of $\F_{p}^{*}$.

\item $5$: We see

\begin{align}
    5^{q_{1}} \mod p &= 51539607732
        \nonumber\\
    5^{q_{2}} \mod p &= 5944319764
        \nonumber\\
    5^{q_{3}} \mod p &= 244140625.
\end{align}

\noindent
Because $5^{q_{i}} \ne 1$ for all $i$, $5$ is a generator of $\F_{p}^{*}$.

\item $6$: We see

\begin{align}
    6^{q_{1}} \mod p &= 51539607732
        \nonumber\\
    6^{q_{2}} \mod p &= 5944319764
        \nonumber\\
    6^{q_{3}} \mod p &= 2176782336.
\end{align}

\noindent
Because $6^{q_{i}} \ne 1$ for all $i$, $6$ is a generator of $\F_{p}^{*}$.

\item $7$: We see

\begin{align}
    7^{q_{1}} \mod p &= 51539607732
        \nonumber\\
    7^{q_{2}} \mod p &= 45595287968
        \nonumber\\
    7^{q_{3}} \mod p &= 13841287201.
\end{align}

\noindent
Because $7^{q_{i}} \ne 1$ for all $i$, $7$ is a generator of $\F_{p}^{*}$.
\end{itemize}

Thus, we were able to find 3 generators in the first 6 possibilities.
\end{example}

We note that Proposition~\ref{prop:math_finite_cyclic_generator}
\emph{critically} relies on the factorization of the group order.
It is an open problem to quickly determine a generator of a \gls{cyclic group};
that is, it is an open problem to determine a polynomial time algorithm
which computes a generator for a \gls{cyclic group} without knowing
the factorization of the group order~\cite[Remark 11.1.25]{mullen2013handbook}


\subsection{Quadratic Residues and the Legendre Symbol}
\label{app:math_finite_fields_legendre}

When working within $\F_{p}$, we want to know if $a\in\F_{p}$
has a square root;
that is, we want to know if there is an $x\in\F_{p}$ such that

\begin{equation}
    x^{2} = a \mod p.
\end{equation}

To this end, we have the following definitions:

\begin{defn}[Quadratic Residues and Quadratic Nonresidues]
Given $a\in\F_{p}^{*}$, we say that $a$ is a \emph{quadratic residue}
if there is some $x\in\F_{p}^{*}$ such that

\begin{equation}
    x^{2} = a \mod p.
\end{equation}

\noindent
If this equation has no solution, then we say that $a$
is a \emph{quadratic nonresidue}.
\end{defn}

\noindent
We will sometimes shorten \emph{quadratic residue} to \emph{residue}
and \emph{quadratic nonresidue} to \emph{nonresidue}.

We note the following interesting property:

\begin{align}
    R\cdot R &= R \nonumber\\
    R\cdot N &= N \nonumber\\
    N\cdot N &= R.
\end{align}

\noindent
Here, $R$ denotes a residue and $N$ denotes a nonresidue.
While this may seem strange at first, we recall the following
fact about the multiplication of real numbers:

\begin{align}
    \parens{+}\cdot\parens{+} &= \parens{+} \nonumber\\
    \parens{+}\cdot\parens{-} &= \parens{-} \nonumber\\
    \parens{-}\cdot\parens{-} &= \parens{+}.
\end{align}

\noindent
We also note this fact about adding integers:

\begin{align}
    E + E &= E \nonumber\\
    E + O &= O \nonumber\\
    O + O &= E.
\end{align}

\noindent
Here, $E$ denotes an even integer and $O$ denotes an odd integer.
In some ways, this property about quadratic residues and nonresidues
may not be so strange.

From this definition, we can define another:

\begin{defn}[Legendre Symbol]
Given $a\in\F_{p}$, we can define the \emph{Legendre symbol} as

\begin{equation}
    \parens{\frac{a}{p}} \mathDef{} \begin{cases}
        1, &\quad \text{$a$ is a quadratic residue} \\
        -1, &\quad \text{$a$ is a quadratic nonresidue} \\
        0, &\quad \text{$a=0$} \\
    \end{cases}.
\end{equation}
\end{defn}

One of the important properties of the Legendre symbol
is that is it multiplicative:
for all $a,b\in\F_{p}$, we have

\begin{equation}
    \parens{\frac{ab}{p}} = 
    \parens{\frac{a}{p}}
    \parens{\frac{b}{p}}.
\end{equation}

\noindent
Additionally, it is not too difficult to show that
there are equal numbers of residues and nonresidues.

\begin{example}[Computing Legendre Symbols by Squaring]
\exampleCodeReference{examples/app\_math/legendre\_squaring.py}

We will work with prime $p = 13$.
We see the following:

\begin{align}
    1^{2} &= 1
        \nonumber\\
    2^{2} &= 4
        \nonumber\\
    3^{2} &= 9
        \nonumber\\
    4^{2} &= 3
        \nonumber\\
    5^{2} &= 12
        \nonumber\\
    6^{2} &= 10
        \nonumber\\
    7^{2} &= 10
        \nonumber\\
    8^{2} &= 12
        \nonumber\\
    9^{2} &= 3
        \nonumber\\
    10^{2} &= 9
        \nonumber\\
    11^{2} &= 4
        \nonumber\\
    12^{2} &= 1.
\end{align}

\noindent
The numbers on the right-hand side are residues,
while the numbers that do not show up are nonresidues.
This implies the following:

\begin{align}
    \text{Residues of $p=13$:}
        &\quad 1, 3, 4, 9, 10, 12 \nonumber\\
    \text{Nonresidues of $p=13$:}
        &\quad 2, 5, 6, 7, 8, 11.
\end{align}
\end{example}

To determine the residues and nonresidues for the previous example,
we explicitly squared each element in $\F_{p}^{*}$.
A more efficient method is this:

\begin{equation}
    \parens{\frac{a}{p}} = a^{\frac{p-1}{2}}.
    \label{eq:app_math_legendre_compute}
\end{equation}

\begin{example}[Computing Legendre Symbols by Exponentiation]
\exampleCodeReference{examples/app\_math/legendre\_exponentiation.py}

We will work with prime $p = 7919$.
We see the following:

\begin{align}
    1^{\frac{p-1}{2}} &= 1
        \nonumber\\
    2^{\frac{p-1}{2}} &= 1
        \nonumber\\
    3^{\frac{p-1}{2}} &= 1
        \nonumber\\
    4^{\frac{p-1}{2}} &= 1
        \nonumber\\
    5^{\frac{p-1}{2}} &= 1
        \nonumber\\
    6^{\frac{p-1}{2}} &= 1
        \nonumber\\
    7^{\frac{p-1}{2}} &= -1.
\end{align}

\noindent
Thus, we see that all of these numbers have modular square roots in $p$
except for $7$.
\end{example}


\subsection{Square Roots}
\label{app:math_finite_fields_sqrt}

In Eq.~\eqref{eq:app_math_legendre_compute}, we were given an efficient
method for computing the Legendre symbol;
however, knowing a square root \emph{exists} is different from knowing
its \emph{value}.

Suppose that $a\in\F_{p}^{*}$ is a residue and $p = 3 \mod 4$.
In this case, we know

\begin{equation}
    a^{\frac{p-1}{2}} = 1.
    \label{eq:app_math_qr_p3m4}
\end{equation}

\noindent
We note the following:

\begin{align}
    \parens{a^{\frac{p+1}{4}}}^{2} &= a^{\frac{p+1}{2}}
        \nonumber\\
    &= a\cdot a^{\frac{p-1}{2}}
        \nonumber\\
    &= a.
\end{align}

\noindent
We used Eq.~\eqref{eq:app_math_qr_p3m4} going from line 2 to line 3.

This is a simple method for computing square roots
in $\F_{p}^{*}$ when $p = 3 \mod 4$:
Given $a\in\F_{p}^{*}$, we have

\begin{equation}
    \sqrt{a} \mathDef{} a^{\frac{p+1}{4}}.
\end{equation}

\noindent
We note that no deterministic algorithm
is known when $p = 1 \mod 4$;
even so, square roots may still be computed using the
Tonelli--Shanks algorithm%
\footnote{\url{https://en.wikipedia.org/wiki/Tonelli\%E2\%80\%93Shanks_algorithm}}.

\begin{example}[Computing Square Roots by Exponentiation]
\exampleCodeReference{examples/app\_math/square\_root.py}

We will work with prime $p = 7919$;
we note that $p = 3 \mod 4$.
We see the following:

\begin{align}
    1^{\frac{p+1}{4}} \mod p &= 1
        \nonumber\\
    2^{\frac{p+1}{4}} \mod p &= 89
        \nonumber\\
    3^{\frac{p+1}{4}} \mod p &= 3023
        \nonumber\\
    4^{\frac{p+1}{4}} \mod p &= 2
        \nonumber\\
    5^{\frac{p+1}{4}} \mod p &= 1782
        \nonumber\\
    6^{\frac{p+1}{4}} \mod p &= 7720.
\end{align}

\noindent
We now verify the square roots:

\begin{align}
    1^{2} \mod p &= 1
        \nonumber\\
    89^{2} \mod p &= 2
        \nonumber\\
    3023^{2} \mod p &= 3
        \nonumber\\
    2^{2} \mod p &= 4
        \nonumber\\
    1782^{2} \mod p &= 5
        \nonumber\\
    7720^{2} \mod p &= 6.
\end{align}

We know that $7$ does not have a square root.
If we ignore this and compute

\begin{equation}
    7^{\frac{p+1}{4}} \mod p = 4769,
\end{equation}

\noindent
then we find

\begin{align}
    4769^{2} \mod p &= 7912 \mod p \nonumber\\
        &= -7 \mod p.
\end{align}

\noindent
Thus, we \emph{do} need to first compute the Legendre Symbol
before attempting to compute its square root.
\end{example}

\section{Rings}
\label{sec:math_rings}

\subsection{Why do we care about Rings?}
\Glspl{ring} arise in many different areas.
We use \glspl{ring} to ease our way into the discussion about
\glspl{field} in Chapter~\ref{sec:math_fields}.

\subsection{Intuition and Examples}
\emph{\Glspl{ring}} are mathematical objects which we will sometimes encounter.
They are a generalization of the integers when we look
at both addition \emph{and} multiplication of integers.
We begin with some examples.

\begin{example}[Integers under addition and multiplication]
We know that we can add and multiply integers and remain
in the set of integers.
In this way, the integers are closed under addition
and closed under multiplication.
We also note that both addition and multiplication are \gls{associative}.
We know that both addition is \gls{commutative};
multiplication is also \gls{commutative},
but here we emphasize the \gls{commutative} nature of addition.

In the case of addition, we know that $0\in\Z$ is the additive inverse:
for every $a\in\Z$, we have

\begin{equation}
    0 + a = a + 0 = a.
\end{equation}

\noindent
Similarly, we have that $1\in\Z$ is the multiplicative inverse:
we have

\begin{equation}
    1\cdot a = a\cdot1 = a
\end{equation}

\noindent
for all $a\in\Z$.

We also know that addition and multiplication follow certain rules.
In particular, given $a,b,c\in\Z$, we have

\begin{equation}
    a\cdot\parens{b+c} = \parens{a\cdot b} + \parens{a\cdot c};
\end{equation}

\noindent
that is, we can \emph{distribute} multiplication over addition.
We also have

\begin{equation}
    \parens{a+b}\cdot c = \parens{a\cdot c} + \parens{b\cdot c}.
\end{equation}
\end{example}

\begin{example}[Integers modulo $n$]
We now look at modular arithmetic.
To be concrete, we look at $\Z_{6}$.
This satisfies all of the properties of the integers under addition
and multiplication.

One interesting property of $\Z_{6}$ but not $\Z$ is that $2,3\in\Z_{6}$
with $2\ne0$ and $3\ne0$ but

\begin{equation}
    2\cdot 3 = 0 \mod 6.
\end{equation}

\noindent
This is an example where two nonzero elements may be multiplied
together to equal zero.
Thus, we can see that there is something distinctly different
between $\Z$ and $\Z_{6}$.
\end{example}

\subsection{Formal Definition}

\begin{defn}[Ring]
A \gls{ring} is a \gls{set} $R$ together with two binary operations addition $+$
and multiplication $\cdot$.
First, $R$ is closed under addition and multiplication.
Furthermore, we have the following properties:

\begin{itemize}
\item $\parens{R,+}$ is an \gls{abelian group} with additive identity $0$.
    This implies that addition is \gls{associative}.
\item Multiplication is \gls{associative} on $R$; that is,
    for all $a,b,c \in R$, we have

\begin{equation}
    \parens{a\cdot b}\cdot c = a\cdot\parens{b\cdot c}.
\end{equation}
\item There is multiplicative identity $1\in R$; that is, 
    for all $a\in R$, we have

\begin{equation}
    1\cdot a = a\cdot 1 = a.
\end{equation}

\noindent
Furthermore, $0\ne 1$.

\item We have the following distribution laws between multiplication
    and addition.
    For all $a,b,c\in R$, we have

\begin{align}
    a\cdot\parens{b + c} &= \parens{a\cdot b} + \parens{a\cdot c}
        \nonumber\\
    \parens{b + c}\cdot a &= \parens{b\cdot a} + \parens{c\cdot a}
\end{align}
\end{itemize}

\noindent
We formally write $\parens{R,+,\cdot}$ is a ring.
\end{defn}

The above definition holds for all rings.
A \emph{\gls{commutative ring}} is one where multiplication
is \gls{commutative};
that is $a\cdot b = b\cdot a$ for all $a,b\in R$.
We will focus on \glspl{commutative ring} here because those will arise
in our work moving forward.
Thus, our distribution law reduces to 

\begin{equation}
    a\cdot\parens{b + c} = \parens{a\cdot b} + \parens{a\cdot c}.
\end{equation}

\subsection{Continued Discussion}

We say that  $a\in R\setminus\braces{0}$ is a \emph{zero divisor}
if there exists $b\in R\setminus\braces{0}$ such that $ab = 0$.

\begin{example}[Example of Zero Divisors]
We continue to look at the \gls{ring} $\Z_{6}$.
We have $2,3\in\Z_{6}$ with $2\ne0$ and $3\ne0$, yet we know

\begin{equation}
    2\cdot 3 = 6 \equiv 0 \mod 6.
\end{equation}

\noindent
This implies that $2$ and $3$ are zero divisors in $\Z_{6}$.

More generally, let $n = pq$ for distinct primes $p$ and $q$.
Then $p,q\in\Z_{n}$ and we know

\begin{equation}
    p\cdot q = n \equiv 0 \mod n.
\end{equation}

\noindent
This shows that $p$ and $q$ are zero divisors in $\Z_{n}$.
\end{example}

\begin{example}[Non-example of Zero Divisors]
While we may not have used this language before,
$\Z$ has no zero divisors.
We know (although we have not shown) that for $a,b\in\Z$ with
$ab = 0$ implies that $a=0$ or $b=0$.

This shows that the \gls{ring} $\parens{\Z,+,\cdot}$
is very different from the \glspl{ring} $\parens{\Z_{n},+,\cdot}$
when $n$ is composite.
\end{example}

\subsection{Encoding Rings}

When working with the \gls{ring} $\Z_{n}$, we can encode $k\in\Z_{n}$
by its binary representation.
Other rings have different representations.

\section{Applications of Cryptographic Hash Functions}

\subsection{Hash-based Message Authentication Code (HMAC)}
\label{app:crypto_hmac}

We now spend time discussing \glspl{hmac} (HMAC)~\cite{HMAC1996,rfc2104}.

As mentioned in Chapter~\ref{sec:symmetric_mac},
a \emph{\gls{mac}} ensures message integrity.
This results from using a secret key $k$ together with a message $m$
to produce a tag $t$.
For \glspl{mac} to be useful, it should be impractical for adversary Eve
to produce a valid tag even if she has seen multiple valid
message-tag pairs.

Throughout this section we will work with a \gls{hash function}
$H:\braces{0,1}^{*}\to\braces{0,1}^{b}$.
This specific construction is designed for \glspl{hash function}
based the \MD{} construction
discussed in Appendix~\ref{app:crypto_md_construction}.
A \gls{mac} based on the sponge construction in
Appendix~\ref{app:crypto_sponge_construction}
will be discussed in Appendix~\ref{app:crypto_kmac}.

\subsubsection{Initial (Incorrect) Guess for Hash-based MAC}

We let $k\in\braces{0,1}^{\ell}$ be our secret key;
in practice, we frequently have $\text{len}(k) = b$.

One potential method for constructing a \gls{mac} based on a
\gls{hash function} would be the following:
Given a message $m$ and secret key $k$, compute

\begin{equation}
    t \mathDef{} H(k||m).
\end{equation}

\noindent
While this could be used, the challenge with this method is that,
if $H$ is based on the \MD{} construction,
this is susceptible to length extension attacks.
Given a valid $\parens{m,t}$, Eve would be able to construct
valid MAC tags for all messages of the form $\textsf{pad}(m)||y$.
\emph{This should not be possible.}

\subsubsection{HMAC Definition}

The \gls{hmac} (HMAC)~\cite{HMAC1996,rfc2104}
was constructed so that length extension attacks are not possible.
Given a key $k$ and message $m$,
we compute the following:

\begin{align}
    k_{1} &\mathDef{} k\oplus\textsf{ipad} \nonumber\\
    k_{2} &\mathDef{} k\oplus\textsf{opad} \nonumber\\
    t' &\mathDef{} H(k_{1}||m) \nonumber\\
    t &\mathDef{} H(k_{2}||t').
\end{align}

\noindent
Here, \textsf{ipad} and \textsf{opad} are two different public
byte constants which ensure that $k$ produces to different
keys for each application of the \gls{hash function}:
one key for the inner application of the \gls{hash function},
and a second key for the outer application.
The two applications of the \gls{hash function} ensure that
it is not possible to perform length extension attacks on the protocol.
The HMAC definition can be combined to

\begin{align}
    t &\mathDef{} \textsf{HMAC}(k, m) \nonumber\\
    \textsf{HMAC}(k, m) &\mathDef{}
    H\parens{\brackets{k\oplus\textsf{opad}}||
        H\parens{\brackets{k\oplus\textsf{ipad}}||m}}.
\end{align}

This scheme has been proven secure~\cite{cryptoeprint:2014:578}.
Because the \MDFive{} and \ShaOne{} \glspl{hash function} are compromised,
HMAC-\MDFive{} and HMAC-\ShaOne{}
\emph{should not be used}~\cite{cryptoeprint:2006:187,rfc6151};
they should \emph{only} be used when \emph{required} for legacy systems.
Although it may be the case that HMAC-\ShaOne{}
is still secure~\cite[Section 3.3]{rfc6194},
the fact remains that \ShaOne{} is a \emph{compromised} \gls{hash function}.


\subsection{Keccak Message Authentication Code (KMAC)}
\label{app:crypto_kmac}

\Keccak{}-MAC is a \gls{mac} based on the \Keccak{} sponge function.
While it would be possible to also use the HMAC construction
from Appendix~\ref{app:crypto_hmac}, the KMAC construction is more efficient
because it requires only one hash operation instead of two.

We now define KMAC.
As before, we let $k$ be the secret key and $m$ be the message;
we let $H$ be a \gls{hash function} based on the \Keccak{} sponge function
with rate $r$.
The tag is then

\begin{align}
    t &\mathDef{} \textsf{KMAC}(k,m) \nonumber\\
        &= H\parens{\textsf{pad}(k,r)||m}.
\end{align}

\noindent
Here, $\textsf{pad}(k,r)$ means that we pad $k$ to a multiple of $r$ bits.
KMAC is formally specified as a derived \ShaThree{}
function~\cite{NIST-SP-800-185}.
In~\cite{NIST-SP-800-185},
additional padding is used for domain separation.
By padding to a multiple of the rate, it is possible to precompute
the internal state so that the first iteration has already been processed.
This would allow for faster computation
if many tags need to be calculated for the same key.


\subsection{HMAC-based Key Derivation Function (HKDF)}
\label{app:crypto_hkdf}

Here we will discuss the \gls{hkdf} (HKDF) protocol,
attempting to condense the main ideas from
the primary sources~\cite{HKDF2010,rfc5869}.

The purpose of \emph{\glspl{kdf}}
is to convert \emph{nonuniform} randomness into \emph{uniform} randomness.
Nonuniform randomness could come from the \gls{shared secret}
of a \gls{dhke};
in this case, the \gls{shared secret} is a random group element
but not a \emph{uniformly} random bit string.
Most \gls{symmetric key encryption} protocols assume a
\emph{uniform} random key,
so merely using the \gls{shared secret} would violate the assumptions
of the protocol;
violating cryptographic protocol security assumptions is
\emph{always} a bad idea.

The \gls{hkdf} protocol is based on an
\emph{extract-then-expand} principle:
nonuniform randomness is first compressed into a uniformly random key;
this uniformly random key is then used to derive additional
uniformly random bits.
From~\cite{HKDF2010}, we have

\begin{equation}
    \textsf{HKDF}\parens{\textsf{salt}, \textsf{skm}, \textsf{ctx}, L}
        \mathDef{} \textsf{trunc}\parens{T(1) || T(2) || \cdots || T(t), L},
\end{equation}

\noindent
where \textsf{salt} is a \gls{salt},
\textsf{skm} is the source key material,
\textsf{ctx} is the context information, and $L$ is the output in bits.
We recall that a \glsfirst{salt} is used to ensure
that different source key materials produce different pseudorandom keys.
\Glspl{salt} are closely related to \glspl{nonce}.
From here, we define $T(i)$ by

\begin{align}
    \textsf{prk} &\mathDef{} \textsf{HMAC}\parens{\textsf{salt},\textsf{skm}}
            \nonumber\\
    T(1) &\mathDef{} \textsf{HMAC}\parens{\textsf{prk}, \textsf{ctx}||1}
            \nonumber\\
    T(i) &\mathDef{} \textsf{HMAC}\parens{\textsf{prk},T(i-1)||
        \textsf{ctx}||i} \quad i\in\braces{2,\cdots,t}.
\end{align}

\noindent
The source key material (possibly with a \gls{salt}) is concentrated
into a pseudorandom key \textsf{prk}.
From there, this pseudorandom key is not output directly
but rather is used to derive additional pseudorandom bits.
In order to help resist any correlation affects between repeated
\textsf{HMAC} calls,
the pseudorandom bits from one iteration are fed into
the message for the next iteration;
this helps ensure that the \textsf{HMAC} instantiations
are independent.
The context information enables the same source key material
to be reused in different settings while ensuring
independent keys.

We note that all of the operations here which use \textsf{HMAC}
could also be replaced with calls to \textsf{KMAC}.


\subsection{PBKDF2}
\label{app:crypto_pbkdf2}

Although we have mentioned that
Password-Based Key Derivation Function 2~\cite{rfc8018}
should not be used~\cite{blocki2018economics},
we \emph{will} describe it here in order to help the reader understand
how it works and how it attempts to increase the difficulty
over standard \glsfirstplural{hash function}.
Even so, \emph{do not use this method to store passwords!}
It was originally designed in 2000~\cite{rfc2898}.

One of the difficulties of protecting passwords with standard
\glsfirstplural{hash function} is that specialized hardware
can be designed to quickly compute them.
To slow down specialized hardware like
application-specific integrated circuits (ASICs)%
\footnote{\url{https://en.wikipedia.org/wiki/Application-specific_integrated_circuit}},
it is necessary to add computations
which \emph{must} be performed sequentially.

We now give the specification from~\cite{rfc8018};
we use \textsf{HMAC} as the required pseudorandom function:

\begin{equation}
    \textsf{PBKDF2}\parens{\textsf{pw}, \textsf{salt}, c, L}
        \mathDef{} T_{1} || T_{2} || \cdots || T_{k}.
\end{equation}

\noindent
Here, $\textsf{pw}$ is the password,
$\textsf{salt}$ is the \gls{salt},
$c$ is the number of iterations,
and $L$ is the derived key output length.

We have

\begin{equation}
    T_{i} \mathDef{} U_{1,i} \oplus U_{2,i} \oplus \cdots \oplus U_{c,i}
\end{equation}

\noindent
where

\begin{align}
    U_{1,i} &\mathDef{} \textsf{HMAC}\parens{\textsf{pw}, \textsf{salt} || i}
        \nonumber\\
    U_{2,i} &\mathDef{} \textsf{HMAC}\parens{\textsf{pw}, U_{1,i}}
        \nonumber\\
    U_{3,i} &\mathDef{} \textsf{HMAC}\parens{\textsf{pw}, U_{2,i}}
        \nonumber\\
        &\quad \vdots \nonumber\\
    U_{c,i} &\mathDef{} \textsf{HMAC}\parens{\textsf{pw}, U_{c-1,i}}.
\end{align}

\noindent
This feed-forward scheme is similar to what we saw in the \gls{hkdf}
scheme in Appendix~\ref{app:crypto_hkdf}.
Greater computation can be required by having a large $c$ value,
as these require many \emph{sequential} \textsf{HMAC} iterations.
This causes a slowdown in the total operation.

We note that although this is a good starting place,
recent work has shown that the protection PBKDF2 offers
is insufficient against dedicated attackers~\cite{blocki2018economics}.
This is because it is possible to build dedicated hardware
to efficiently compute PBKDF2.
In particular, we note that the $T_{i}$ values could be built up
iteratively, so all the $U_{j,i}$ values do not need to be stored.
As such, the algorithm does not have large memory requirements.
\emph{Do not use this algorithm to store passwords!}


\subsection{Hash to Finite Field}
\label{app:crypto_hash-to-finite-field}

Here we will work out the details proving the bounds for hashing
to \gls{finite field}.

Suppose we have a \gls{random oracle} $\widetilde{H}:\braces{0,1}^{*}\to\Z_{N}$.
We are interested in knowing how much $\widetilde{H}(m) \mod p$
deviates from the uniform distribution.

We suppose that $N>p$ and $p\nmid N$;
that is, $p$ is not a divisor of $N$.
In this case, find $q$ and $r$ such that

\begin{equation}
    N = qp + r.
\end{equation}

\noindent
Here, we have $q\ge1$ and $0\le r < p$.
We have $r\ge1$ because $p\nmid N$.

We let $X$ be a random variable uniformly distributed on $\Z_{N}$
and $X_{p} \mathDef{} X \mod p$.
Then we have the following:

\begin{align}
    \mathcal{P}\parens{X_{p} = k} &= \frac{q+1}{N},
        \quad k\in \braces{0,\cdots,r-1} \nonumber\\
    \mathcal{P}\parens{X_{p} = k} &= \frac{q}{N},
        \quad k \in \braces{r,\cdots,N-1}.
\end{align}

\noindent
Here, $\mathcal{P}$ is the probability of an event occurring.
Thus, $\mathcal{P}(X_{p} = k)$ denotes the probability
that the random variable $X_{p}$ equals the value $k$.

We now determine how far $X_{p}$ deviates from the uniform
distribution $U_{p}$:

\begin{align}
    \Delta(X_{p},U_{p}) &\mathDef{}
    \sum_{k=0}^{p-1} \abs{\mathcal{P}(X_{p}=k) - \mathcal{P}(U_{p}=k)}
        \nonumber\\
    &= \sum_{k=0}^{r-1} \abs{\frac{q+1}{N} - \frac{1}{p}} +
        \sum_{k=r}^{p-1} \abs{\frac{q}{N} - \frac{1}{p}} \nonumber\\
    &= r\frac{qp + p - N}{pN} + \parens{p-r}\frac{N-qp}{pN}
        \nonumber\\
    &\le \frac{p}{N}.
\end{align}

\noindent
Thus, the relative size of $p$ and $N$ controls the deviation from uniformity.
In this way, if the $p$ is $n$-bit prime number and we desire
$k$ bits of security,
then we need $N\ge 2^{n+k}$.
After hashing into $\Z_{N}$, performing a reduction modulo $p$
will result in a sufficiently uniform hash distribution.

In some sense, we are free in how we choose $N$ provided it is
sufficiently large.
Because we have \glspl{hash function} which output bits,
it would be easy to hash to some power of 2.
In practice, our prime $p$ will be somewhat large;
generally, at least 200 bits, and usually 256 bits or more.
If we want at least 128 bits of security, then a standard 256-bit
\gls{hash function} will not be sufficient.

Because of this, using the \gls{hkdf} previously discussed in
Appendix~\ref{app:crypto_hkdf} appears to be a reasonable decision.
Although we are not interested in keys \emph{per se},
we are looking to map random inputs
to random outputs.
These outputs will then be deterministically mapped to points
on an \gls{elliptic curve}.

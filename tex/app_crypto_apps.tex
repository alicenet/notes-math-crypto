\section{Applications of Cryptographic Hash Functions}

\subsection{Hash-based Message Authentication Code (HMAC)}
\label{app:crypto_hmac}

We now spend time discussing \glspl{hmac} (HMAC)~\cite{HMAC1996,rfc2104}.

As mentioned in Chapter~\ref{sec:symmetric_mac},
a \emph{\gls{mac}} ensures message integrity.
This results from using a secret key $k$ together with a message $m$
to produce a tag $t$.
For \glspl{mac} to be useful, it should be impractical for adversary Eve
to produce a valid tag even if she has seen multiple valid
message-tag pairs.

Throughout this section we will work with a \gls{hash function}
$H:\braces{0,1}^{*}\to\braces{0,1}^{b}$.
This specific construction is designed for \glspl{hash function}
based the \MD{} construction
discussed in Appendix~\ref{app:crypto_md_construction}.
A \gls{mac} based on the sponge construction in
Appendix~\ref{app:crypto_sponge_construction}
will be discussed in Appendix~\ref{app:crypto_kmac}.

\subsubsection{Initial (Incorrect) Guess for Hash-based MAC}

We let $k\in\braces{0,1}^{\ell}$ be our secret key;
in practice, we frequently have $\text{len}(k) = b$.

One potential method for constructing a \gls{mac} based on a
\gls{hash function} would be the following:
Given a message $m$ and secret key $k$, compute

\begin{equation}
    t \mathDef{} H(k\|m).
\end{equation}

\noindent
While this could be used, the challenge with this method is that,
if $H$ is based on the \MD{} construction,
this is susceptible to length extension attacks.
Given a valid $\parens{m,t}$, Eve would be able to construct
valid MAC tags for all messages of the form $\textsf{pad}(m)\|y$.
\emph{This should not be possible.}

\subsubsection{HMAC Definition}

The \gls{hmac} (HMAC)~\cite{HMAC1996,rfc2104}
was constructed so that length extension attacks are not possible.
Given a key $k$ and message $m$,
we compute the following:

\begin{align}
    k_{1} &\mathDef{} k\oplus\textsf{ipad} \nonumber\\
    k_{2} &\mathDef{} k\oplus\textsf{opad} \nonumber\\
    t' &\mathDef{} H(k_{1}\|m) \nonumber\\
    t &\mathDef{} H(k_{2}\|t').
\end{align}

\begin{algorithm}[t]
\caption{Hash-based Message Authentication Code}
\label{alg:hmac}
\begin{algorithmic}[1]
\Require \textsf{Hash} is a hash function with internal block size $B$
\Procedure{HMAC}{$k$, $m$}
    \State $\textsf{ipad} \algAssign{} \texttt{0x3636}\cdots\texttt{36}$
    \Comment{The byte \texttt{36} is repeated $B$ times}
    \State $\textsf{opad} \algAssign{} \texttt{0x5c5c}\cdots\texttt{5c}$
    \Comment{The byte \texttt{5c} is repeated $B$ times}
    \If {$\textsc{Length}($k$) > B$}
        \State $k \algAssign{} \textsc{Hash}(k)$
    \EndIf
    \State $k' \algAssign{} k\|\texttt{0000}\cdots\texttt{00}$
    \Comment{Concatenate $k$ with byte \texttt{00} until $k'$ has length $B$}
    \State $k_{1} \algAssign{} k' \oplus \textsf{ipad}$
    \State $k_{2} \algAssign{} k' \oplus \textsf{opad}$
    \State $t' \algAssign{} \textsf{Hash}(k_{1}, m)$
    \State $t \algAssign{} \textsf{Hash}(k_{2}, t')$
    \State \Return $t$
\EndProcedure
\end{algorithmic}
\end{algorithm}


\noindent
Here, \textsf{ipad} and \textsf{opad} are two different public
byte constants which ensure that $k$ produces to different
keys for each application of the \gls{hash function}:
one key for the inner application of the \gls{hash function},
and a second key for the outer application.
The two applications of the \gls{hash function} ensure that
it is not possible to perform length extension attacks on the protocol.
The HMAC definition can be combined to

\begin{align}
    t &\mathDef{} \textsf{HMAC}(k, m) \nonumber\\
    \textsf{HMAC}(k, m) &\mathDef{}
    H\parens{\brackets{k\oplus\textsf{opad}}\|
        H\parens{\brackets{k\oplus\textsf{ipad}}\|m}}.
\end{align}

\noindent
The formal specification may be found in
Alg.~\ref{alg:hmac}.
The minimum recommended key size is length of the hash function
output~\cite{rfc2104}.

This scheme has been proven secure~\cite{cryptoeprint:2014:578}.
Because the \MDFive{} and \ShaOne{} \glspl{hash function} are compromised,
HMAC-\MDFive{} and HMAC-\ShaOne{}
\emph{should not be used}~\cite{cryptoeprint:2006:187,rfc6151};
they should \emph{only} be used when \emph{required} for legacy systems.
Although it may be the case that HMAC-\ShaOne{}
is still secure~\cite[Section 3.3]{rfc6194},
the fact remains that \ShaOne{} is a \emph{compromised} \gls{hash function}.

\begin{example}[HMAC-\MDFive{}]
\exampleCodeReference{examples/app\_crypto/hmac\_md5.py}

We compute the output of HMAC using \MDFive{};
see Listing~\ref{list:hmac_md5}.
We have the following key and message pair:

\begin{align}
    \textsf{key} &=
            \textsf{\MDFive{}}(\texttt{13131313131313131313131313131313})
            \nonumber\\
        &= \texttt{e885a118b42df3823f4999d4135ce08f} \nonumber\\
    \textsf{msg} &= 
    \texttt{4242424242424242424242424242424242424242424242424242424242424242}.
\end{align}

\noindent
The resulting value is

\begin{equation}
    \textsf{HMAC-\MDFive{}}\parens{\textsf{key},\textsf{msg}} =
        \texttt{7fc82df5f026fab8f8f880bf73a8bb08}.
\end{equation}

\lstset{
    showstringspaces=false,
    frame=single
}
\lstinputlisting[
    label=list:hmac_md5,
    float,
    language=Python,
    caption={HMAC using \MDFive{}}]{code/app_crypto/hmac_md5.py}

\end{example}


\subsection{Keccak Message Authentication Code (KMAC)}
\label{app:crypto_kmac}

\Keccak{}-MAC is a \gls{mac} based on the \Keccak{} sponge function.
While it would be possible to also use the HMAC construction
from Appendix~\ref{app:crypto_hmac}, the KMAC construction is more efficient
because it requires only one hash operation instead of two.

We now define KMAC.
As before, we let $k$ be the secret key and $m$ be the message;
we let $H$ be a \gls{hash function} based on the \Keccak{} sponge function
with rate $r$.
The tag is then

\begin{algorithm}[t]
\caption{Keccak Message Authentication Code}
\label{alg:kmac}
\begin{algorithmic}[1]
\Require \textsf{Hash} is a hash function based on \textsc{Keccak} with internal rate $r$
\Procedure{KMAC}{$k$, $m$}
    \State $k' \algAssign{} \textsf{pad}(k, r)$
    \Comment{Pad $k$ to a multiple of $r$ bytes with zero bytes}
    \State $t \algAssign{} \textsf{Hash}(k', m)$
    \State \Return $t$
\EndProcedure
\end{algorithmic}
\end{algorithm}


\begin{align}
    t &\mathDef{} \textsf{KMAC}(k,m) \nonumber\\
        &= H\parens{\textsf{pad}(k,r)\|m}.
\end{align}

\noindent
Here, $\textsf{pad}(k,r)$ means that we pad $k$ to a multiple of $r$ bits.
KMAC is formally specified as a derived \ShaThree{}
function~\cite{NIST-SP-800-185}.
In~\cite{NIST-SP-800-185},
additional padding is used for domain separation.
By padding to a multiple of the rate, it is possible to precompute
the internal state so that the first iteration has already been processed.
This would allow for faster computation
if many tags need to be calculated for the same key.
The formal specification may be found in
Alg.~\ref{alg:kmac}.

\begin{example}[KMAC using \ShaThree{}]
\exampleCodeReference{examples/app\_crypto/kmac\_sha3.py}

We compute the output of KMAC using \ShaThree{};
see Listing~\ref{list:kmac_sha3}.
We have the following key and message pair:

\begin{align}
    \textsf{key} &=
    \textsf{\ShaThree{}}(
    \texttt{1313131313131313131313131313131313131313131313131313131313131313})
        \nonumber\\
    &= \texttt{0554a7d12c7a1bfcb47088e98851c036ca90c3e2d2b0ef6d88b341d86ec6bd4a}
        \nonumber\\
    \textsf{msg} &= 
    \texttt{4242424242424242424242424242424242424242424242424242424242424242}.
\end{align}

\noindent
The resulting value is

\begin{align}
    &\textsf{KMAC-\ShaThree{}}\parens{\textsf{key},\textsf{msg}} = \nonumber\\
    &\texttt{b4033d41cef0553003b8e1b189fae0c44283a4b1870f59cdd8a8133a7d31c585}.
\end{align}

\lstset{
    showstringspaces=false,
    frame=single
}
\lstinputlisting[
    label=list:kmac_sha3,
    float,
    language=Python,
    caption={KMAC using \ShaThree{}}]{code/app_crypto/kmac_sha3.py}

\end{example}


\subsection{HMAC-based Key Derivation Function (HKDF)}
\label{app:crypto_hkdf}

Here we will discuss the \gls{hkdf} (HKDF) protocol,
attempting to condense the main ideas from
the primary sources~\cite{HKDF2010,rfc5869}.

\subsubsection{Purpose of Key Derivation Functions}

The purpose of a \emph{\gls{kdf}}
is to convert \emph{nonuniform} randomness into \emph{uniform} randomness.
Nonuniform randomness may arise in a variety of places.
One example is the \gls{shared secret} of a \gls{dhke}:
the \gls{shared secret} is a random group element
but not a \emph{uniformly} random bit string.
Most \gls{symmetric key encryption} protocols assume a
\emph{uniform} random key,
so merely using the \gls{shared secret} would violate the assumptions
of the protocol;
violating cryptographic protocol security assumptions is
\emph{always} a bad idea.

\subsubsection{Traditional KDF Protocol}

As stated in~\cite[Section 8]{HKDF2010},
the traditional KDF scheme is given as

\begin{equation}
    T \mathDef{} \textsf{Hash}\parens{\textsf{skm}\|1\|\textsf{ctx}}
                 \textsf{Hash}\parens{\textsf{skm}\|2\|\textsf{ctx}}
                 \textsf{Hash}\parens{\textsf{skm}\|3\|\textsf{ctx}}\cdots
\end{equation}

\noindent
Here, \textsf{skm} is the source key material,
\textsf{ctx} is the context information, and $\ell$ is the output in bits;
The result is truncated to the appropriate length.
A formal specification may be found in Alg.~\ref{alg:trad_kdf}.

\begin{algorithm}[t]
\caption{Traditional Key Derivation Function}
\label{alg:trad_kdf}
\begin{algorithmic}[1]
\Require \textsf{Hash} is a hash function with output size $n$
\Procedure{TraditionalKDF}{\textsf{skm}, \textsf{ctx}, $\ell$}
    \State $t \algAssign{} \ceil{\ell/n}$
    \State $T \algAssign{} \text{``''}$
    \Comment{Empty byte string}
    \For{$i=1$; $i\le t$; $i{+}{+}$}
        \State $T \algAssign{} T\|\textsf{Hash}(\textsf{skm}\|i\|\textsf{ctx})$
    \EndFor
    \State \Return $\textsf{trunc}(T, \ell)$
\EndProcedure
\end{algorithmic}
\end{algorithm}


The NIST KDF scheme slightly modifies the traditional scheme
by switching the order of the counter and the source key material:

\begin{equation}
    T \mathDef{} \textsf{Hash}\parens{1\|\textsf{skm}\|\textsf{ctx}}
                 \textsf{Hash}\parens{2\|\textsf{skm}\|\textsf{ctx}}
                 \textsf{Hash}\parens{3\|\textsf{skm}\|\textsf{ctx}}\cdots
\end{equation}

\noindent
A formal specification may be found in Alg.~\ref{alg:nist_kdf}.
This version (One-Step Key Derivation) is still approved by
NIST~\cite[Section 4]{NIST-SP-800-56Cr2}.
In both the Traditional and NIST KDF specification,
the counter is 4 bytes (32 bits),
and the maximum number of iterations is $2^{32}$.

\begin{algorithm}[t]
\caption{NIST Key Derivation Function
    (One-Step Key Derivation~\cite[Section 4]{NIST-SP-800-56Cr2})}
\label{alg:nist_kdf}
\begin{algorithmic}[1]
\Require \textsf{Hash} is a hash function with output size $n$
\Procedure{NIST-KDF}{\textsf{skm}, \textsf{ctx}, $\ell$}
    \State $t \algAssign{} \ceil{\ell/n}$
    \State $T \algAssign{} \text{``''}$
    \Comment{Empty byte string}
    \For{$i=1$; $i\le t$; $i{+}{+}$}
        \State $T \algAssign{} T\|\textsf{Hash}(i\|\textsf{skm}\|\textsf{ctx})$
    \EndFor
    \State \Return $\textsf{trunc}(T, \ell)$
\EndProcedure
\end{algorithmic}
\end{algorithm}


It is noted that both instances,
the inputs of the hash function are highly correlated;
there are only a few bits which differ between hash calls.
While this is fine when using a \gls{random oracle},
the \glspl{hash function} used in practice are not perfect.
\Gls{hkdf} was designed to counter these problems~\cite{HKDF2010}.


\subsubsection{HKDF Protocol}

The \gls{hkdf} protocol is based on an
\emph{extract-then-expand} principle:
nonuniform randomness is first compressed into a uniformly random key;
this uniformly random key is then used to derive additional
uniformly random bits.
From~\cite{HKDF2010}, we have

\begin{equation}
    \textsf{HKDF}\parens{\textsf{salt}, \textsf{skm}, \textsf{ctx}, \ell}
        \mathDef{} \textsf{trunc}\parens{T(1) \| T(2) \| \cdots \| T(t), \ell},
\end{equation}

\noindent
where \textsf{salt} is a \gls{salt},
\textsf{skm} is the source key material,
\textsf{ctx} is the context information,
and $\ell$ is the length of output in bits.
We recall that a \glsfirst{salt} is used to ensure
that different source key materials produce different pseudorandom keys.
\Glspl{salt} are closely related to \glspl{nonce}.
From here, we define $T(i)$ by

\begin{align}
    \textsf{prk} &\mathDef{} \textsf{HMAC}\parens{\textsf{salt},\textsf{skm}}
            \nonumber\\
    T(1) &\mathDef{} \textsf{HMAC}\parens{\textsf{prk}, \textsf{ctx}\|1}
            \nonumber\\
    T(i) &\mathDef{} \textsf{HMAC}\parens{\textsf{prk},T(i-1)\|
        \textsf{ctx}\|i} \quad i\in\braces{2,\cdots,t}.
\end{align}

\noindent
The source key material (possibly with a \gls{salt}) is concentrated
into a pseudorandom key \textsf{prk}.
From there, this pseudorandom key is not output directly
but rather is used to derive additional pseudorandom bits.
In order to help resist any correlation affects between repeated
\textsf{HMAC} calls,
the pseudorandom bits from one iteration are fed into
the message for the next iteration;
this helps ensure that the \textsf{HMAC} instantiations
are independent.
The context information enables the same source key material
to be reused in different settings while ensuring
independent keys.
A formal specification may be found in Alg.~\ref{alg:hkdf}.
In~\cite{rfc5869}, the counter is 1 byte
and the number of iterations are limited to 255.

\begin{algorithm}[t]
\caption{HMAC-based Key Derivation Function}
\label{alg:hkdf}
\begin{algorithmic}[1]
\Require \textsf{HMAC} is a Hash-based MAC
    which uses a hash function with output size $n$
\Procedure{HKDF}{\textsf{salt}, \textsf{skm}, \textsf{ctx}, $\ell$}
    \State $t \algAssign{} \ceil{\ell/n}$
    \State $T \algAssign{} \text{``''}$
    \Comment{Empty byte string}
    \State $\textsf{prk} \algAssign{}
        \textsf{HMAC}(\textsf{salt}, \textsf{skm})$
    \State $\textsf{out} \algAssign{} 
        \textsf{HMAC}(\textsf{prk}, \textsf{ctx}\|1)$
    \State $T \algAssign{} T\|\textsf{out}$
    \For{$i=2$; $i\le t$; $i{+}{+}$}
        \State $\textsf{out} \algAssign{} 
            \textsf{HMAC}(\textsf{prk}, \textsf{out}\|\textsf{ctx}\|i)$
        \State $T \algAssign{} T\|\textsf{out}$
    \EndFor
    \State \Return $\textsf{trunc}(T, \ell)$
\EndProcedure
\end{algorithmic}
\end{algorithm}


We note that all of the operations here which use \textsf{HMAC}
could also be replaced with calls to \textsf{KMAC},
and this is done in Example~\ref{example:hkdf-kmac-sha3} below.

As seen in Examples~\ref{example:hkdf-hmac-sha2} and
\ref{example:hkdf-kmac-sha3},
calling \textsf{HKDF} with the same
\textsf{skm}, \textsf{salt}, and \textsf{ctx} values
but varying lengths will cause the shorter outputs to be
contained within longer outputs.
Although within the \gls{hkdf} specification
there is no strict requirement that the length of the output
be included within the \textsf{ctx} information,
the author of this work feels this should be encouraged
if there is the slightest chance for variable-length outputs
to occur in practice.

\begin{example}[HKDF using HMAC-\ShaTwo{}-256]
\label{example:hkdf-hmac-sha2}
\exampleCodeReference{examples/app\_crypto/hkdf\_hmac\_sha2.py}

We compute the output of HKDF using HMAC-\ShaTwo{}-256;
see Listing~\ref{list:hkdf_hmac_sha2}.
We have the following source key material, salt, and context:

\begin{align}
    \textsf{skm} &=
    \textsf{\MDFive{}}(
    \texttt{01010101010101010101010101010101})\|\texttt{00\dots00}
        \nonumber\\
    &= \texttt{24311d9abc4077123c2c9a167afbe75400000000000000000000000000000000}
        \nonumber\\
    \textsf{salt} &= 
    \texttt{0202020202020202020202020202020202020202020202020202020202020202}.
\end{align}

\noindent
We let \textsf{ctx} be empty.
In this instance, we are modeling the situation where
some of the source key material has entropy
while other parts of it do not has as much entropy.
We have the following results:

\begin{align}
    &\textsf{HKDF-HMAC-\ShaTwo{}}\parens{
        \textsf{skm},\textsf{salt},\textsf{ctx}, 16} = \nonumber\\
    &\texttt{209a24fb9c313a19d8ece110a922a75c}
        \nonumber\\
    &\textsf{HKDF-HMAC-\ShaTwo{}}\parens{
        \textsf{skm},\textsf{salt},\textsf{ctx}, 32} = \nonumber\\
    &\texttt{209a24fb9c313a19d8ece110a922a75c7fdccb0ea20fff64eacf0bc621e688fe}
        \nonumber\\
    &\textsf{HKDF-HMAC-\ShaTwo{}}\parens{
        \textsf{skm},\textsf{salt},\textsf{ctx}, 64} = \nonumber\\
    &\texttt{209a24fb9c313a19d8ece110a922a75c7fdccb0ea20fff64eacf0bc621e688fe\textbackslash}
        \nonumber\\
    &\texttt{9d905f252edb3b56a13609eaab6c13cd446a073ea08423c32b2fc009271d828c}.
\end{align}

\noindent
As we can see, the output looks pretty random.

\lstset{
    showstringspaces=false,
    frame=single
}
\lstinputlisting[
    label=list:hkdf_hmac_sha2,
    float,
    language=Python,
    linerange={1-12,29-99999},
    caption={HKDF using HMAC-\ShaTwo{}}]{code/app_crypto/hkdf_hmac_sha2.py}

\end{example}

\begin{example}[HKDF using KMAC-\ShaThree{}-256]
\label{example:hkdf-kmac-sha3}
\exampleCodeReference{examples/app\_crypto/hkdf\_kmac\_sha3.py}

We compute the output of HKDF using KMAC-\ShaThree{}-256;
see Listing~\ref{list:hkdf_kmac_sha3}.
We use the same values for \textsf{skm}, \textsf{salt}, and \textsf{ctx}
as in Example~\ref{example:hkdf-hmac-sha2}.
We have the following results:

\begin{align}
    &\textsf{HKDF-KMAC-\ShaThree{}}\parens{
        \textsf{skm},\textsf{salt},\textsf{ctx}, 16} = \nonumber\\
    &\texttt{f9b8dbec9e6ba6c6862cc45cfc408e0b}
        \nonumber\\
    &\textsf{HKDF-KMAC-\ShaThree{}}\parens{
        \textsf{skm},\textsf{salt},\textsf{ctx}, 32} = \nonumber\\
    &\texttt{f9b8dbec9e6ba6c6862cc45cfc408e0b33b125aecff2705650f191b7cb599b74}
        \nonumber\\
    &\textsf{HKDF-KMAC-\ShaThree{}}\parens{
        \textsf{skm},\textsf{salt},\textsf{ctx}, 64} = \nonumber\\
    &\texttt{f9b8dbec9e6ba6c6862cc45cfc408e0b33b125aecff2705650f191b7cb599b74\textbackslash}
        \nonumber\\
    &\texttt{02c32a28defa349720c44c449319f9cff77b211fa5262de491334409115c712d}.
\end{align}

\lstset{
    showstringspaces=false,
    frame=single
}
\lstinputlisting[
    label=list:hkdf_kmac_sha3,
    float,
    language=Python,
    caption={HKDF using KMAC-\ShaThree{}}]{code/app_crypto/hkdf_kmac_sha3.py}

\end{example}


\subsection{Mask Generation Function and Extendable Output Function}
\label{app:crypto_mgf_xof}

We now discuss \glspl{mgf} and \glspl{xof}.

\Glspl{hash function} are very useful,
but when a random output is needed with length greater than
the message digest length, something else must be done.
This was needed in the case of the RSA
encryption or signature schemes~\cite{cryptoeprint:2006/223,rfc8017}.
This lead to the development of a \gls{mgf}.
The particular scheme in~\cite{rfc8017} is given here:

\begin{equation}
    \textsf{MGF}(\textsf{seed}, \ell) \mathDef{}
        \textsf{Hash}(\textsf{seed}\|0)\,
        \textsf{Hash}(\textsf{seed}\|1)\,
        \textsf{Hash}(\textsf{seed}\|2)\,\cdots
\end{equation}

\noindent
In this case, \textsf{Hash} is a hash function,
$\textsf{seed}$ is the data to be hashed,
and $\ell$ is the specified output length.
The numbers are encoded as 32-bit unsigned integers,
and the result is truncated to the desired length;
thus, there is a  maximum of $2^{32}$ iterations.
One reference defining a padding scheme which uses
a \gls{mgf} is from 1998~\cite{rfc2437}.

This definition is for ``MGF1'';
no additional versions appear to have been specified.
A formal specification may be found in Alg.~\ref{alg:mgf1}.
This is very similar to the definition
of the Traditional KDF from Alg.~\ref{alg:trad_kdf}
(the only difference being a lack of notion for context),
and this also means that it has similar problems.

\begin{algorithm}[t]
\caption{Mask Generation Function 1}
\label{alg:mgf1}
\begin{algorithmic}[1]
\Require \textsf{Hash} is a hash function with output size $n$
\Procedure{MGF1}{\textsf{seed}, $\ell$}
    \State $t \algAssign{} \ceil{\ell/n}$
    \State $T \algAssign{} \text{``''}$
    \Comment{Empty byte string}
    \For{$i=0$; $i<t$; $i{+}{+}$}
        \State $T \algAssign{} T\|\textsf{Hash}(\textsf{seed}\|i)$
    \EndFor
    \State \Return $\textsf{trunc}(T, \ell)$
\EndProcedure
\end{algorithmic}
\end{algorithm}


Later, when defining the \ShaThree{} standard~\cite{FIPS-202},
additional functions, called \glspl{xof},
were specified which were like \glspl{hash function}
but with the ability to produce
arbitrarily long output~\cite[Section 6.2]{FIPS-202}.
This is due in part to the sponge function structure of \ShaThree{}.

The difference between \glspl{mgf} and \glspl{xof}
may be that a \gls{mgf} is designed to use a \gls{hash function}
to produce arbitrarily long output
while an \gls{xof} is a ``hash function'' designed
to produce arbitrarily long output.

In summary, the \gls{mgf} as defined in~\cite{rfc8017}
\emph{should not be used except for backward compatibility}
due to the weakness of its definition.
If possible, use an \gls{xof} such as
SHAKE128 or SHAKE256~\cite{FIPS-202,NIST-SP-800-185};
otherwise, use \gls{hkdf} with \textsf{HMAC} or \textsf{KMAC}.


\subsection{PBKDF2}
\label{app:crypto_pbkdf2}

Although we have mentioned that
Password-Based Key Derivation Function 2~\cite{rfc8018}
should not be used~\cite{blocki2018economics},
we \emph{will} describe it here in order to help the reader understand
how it works and how it attempts to increase the difficulty
over standard \glsfirstplural{hash function}.
Even so, \emph{do not use this method to store passwords!}
It was originally designed in 2000~\cite{rfc2898}.

One of the difficulties of protecting passwords with standard
\glsfirstplural{hash function} is that specialized hardware
can be designed to quickly compute them.
To slow down specialized hardware like
application-specific integrated circuits (ASICs)%
\footnote{\url{https://en.wikipedia.org/wiki/Application-specific_integrated_circuit}},
it is necessary to add computations
which \emph{must} be performed sequentially.

\begin{algorithm}[t]
\caption{Password-Based Key Derivation Function Version 2 with \textsf{HMAC}}
\label{alg:pbkdf2}
\begin{algorithmic}[1]
\Require \textsf{HMAC} is a Hash-based MAC
    which uses a hash function with output size $n$
\Procedure{PBKDF2}{\textsf{pw}, \textsf{salt}, $c$, $\ell$}
    \State $t \algAssign{} \ceil{\ell/n}$
    \State $T \algAssign{} \text{``''}$
    \Comment{Empty byte string}
    \For{$i=1$; $i\le t$; $i{+}{+}$}
        \State $U_{1} \algAssign{} 
            \textsf{HMAC}(\textsf{pw}, \textsf{salt}\|i)$
        \For{$k=2$; $k\le c$; $k{+}{+}$}
            \State $U_{k} \algAssign{} 
                \textsf{HMAC}(\textsf{pw}, U_{k-1})$
        \EndFor
        \State $\textsf{out} \algAssign{}
            U_{1} \oplus U_{2} \oplus \cdots \oplus U_{c}$
        \State $T \algAssign{} T\|\textsf{out}$
    \EndFor
    \State \Return $\textsf{trunc}(T, \ell)$
\EndProcedure
\end{algorithmic}
\end{algorithm}


We now give the specification from~\cite{rfc8018};
we use \textsf{HMAC} as the required pseudorandom function:

\begin{equation}
    \textsf{PBKDF2}\parens{\textsf{pw}, \textsf{salt}, c, \ell}
        \mathDef{} T_{1} \| T_{2} \| \cdots \| T_{k}.
\end{equation}

\noindent
Here, $\textsf{pw}$ is the password,
$\textsf{salt}$ is the \gls{salt},
$c$ is the number of iterations,
and $\ell$ is the derived key output length.

We have

\begin{equation}
    T_{i} \mathDef{} U_{1,i} \oplus U_{2,i} \oplus \cdots \oplus U_{c,i}
\end{equation}

\noindent
where

\begin{align}
    U_{1,i} &\mathDef{} \textsf{HMAC}\parens{\textsf{pw}, \textsf{salt} \| i}
        \nonumber\\
    U_{2,i} &\mathDef{} \textsf{HMAC}\parens{\textsf{pw}, U_{1,i}}
        \nonumber\\
    U_{3,i} &\mathDef{} \textsf{HMAC}\parens{\textsf{pw}, U_{2,i}}
        \nonumber\\
        &\quad \vdots \nonumber\\
    U_{c,i} &\mathDef{} \textsf{HMAC}\parens{\textsf{pw}, U_{c-1,i}}.
\end{align}

\noindent
This feed-forward scheme is similar to what we saw in the \gls{hkdf}
scheme in Appendix~\ref{app:crypto_hkdf}.
Greater computation can be required by having a large $c$ value,
as these require many \emph{sequential} \textsf{HMAC} iterations.
This causes a slowdown in the total operation.
A formal specification may be found in Alg.~\ref{alg:pbkdf2}.

We note that although this is a good starting place for password storage,
recent work has shown that the protection PBKDF2 offers
is insufficient against dedicated attackers~\cite{blocki2018economics}.
This is because it is possible to build dedicated hardware
to efficiently compute PBKDF2.
In particular, we note that the $T_{i}$ values could be built up
iteratively, so all the $U_{j,i}$ values do not need to be stored.
As such, the algorithm does not have large memory requirements.
\emph{Do not use this algorithm to store passwords!}


\subsection{Hash to Finite Field}
\label{app:crypto_hash-to-finite-field}

Here we will work out the details proving the bounds for hashing
to \gls{finite field}.

Suppose we have a \gls{random oracle} $H:\braces{0,1}^{*}\to\Z_{N}$.
We are interested in knowing how much $H(m) \mod p$
deviates from the uniform distribution.

We suppose that $N>p$ and $p\nmid N$;
that is, $p$ is not a divisor of $N$.
In this case, find $q$ and $r$ such that

\begin{equation}
    N = qp + r.
\end{equation}

\noindent
Here, we have $q\ge1$ and $0\le r < p$.
We have $r\ge1$ because $p\nmid N$.

We let $X$ be a random variable uniformly distributed on $\Z_{N}$
and $X_{p} \mathDef{} X \mod p$.
Then we have the following:

\begin{align}
    \mathcal{P}\parens{X_{p} = \ell} &= \frac{q+1}{N},
        \quad \ell\in \braces{0,\cdots,r-1} \nonumber\\
    \mathcal{P}\parens{X_{p} = \ell} &= \frac{q}{N},
        \quad \ell\in \braces{r,\cdots,N-1}.
\end{align}

\noindent
Here, $\mathcal{P}$ is the probability of an event occurring.
Thus, $\mathcal{P}(X_{p} = \ell)$ denotes the probability
that the random variable $X_{p}$ equals the value $\ell$.

We now determine how far $X_{p}$ deviates from the uniform
distribution $U_{p}$:

\begin{align}
    \Delta(X_{p},U_{p}) &\mathDef{}
    \sum_{\ell=0}^{p-1} \abs{\mathcal{P}(X_{p}=\ell) - \mathcal{P}(U_{p}=\ell)}
        \nonumber\\
    &= \sum_{\ell=0}^{r-1} \abs{\frac{q+1}{N} - \frac{1}{p}} +
        \sum_{\ell=r}^{p-1} \abs{\frac{q}{N} - \frac{1}{p}} \nonumber\\
    &= r\frac{qp + p - N}{pN} + \parens{p-r}\frac{N-qp}{pN}
        \nonumber\\
    &= \frac{2r(p-r)}{pN}
        \nonumber\\
    &\le \frac{p}{2N}.
\end{align}

\noindent
Thus, the relative size of $p$ and $N$ controls the deviation from uniformity.
In this way, if the $p$ is $n$-bit prime number and we desire
$k$ bits of security,
then we need $N\ge 2^{n+k}$.
After hashing into $\Z_{N}$, performing a reduction modulo $p$
will result in a sufficiently uniform hash distribution.

In some sense, we are free in how we choose $N$ provided it is
sufficiently large.
Because we have \glspl{hash function} which output bits,
it would be easy to hash to some power of 2.
In practice, our prime $p$ will be somewhat large;
generally, at least 200 bits, and usually 256 bits or more.
If we want at least 128 bits of security, then a standard 256-bit
\gls{hash function} will not be sufficient.

Because of this, using the \gls{hkdf} previously discussed in
Appendix~\ref{app:crypto_hkdf} appears to be a reasonable decision.
Although we are not interested in keys \emph{per se},
we are looking to map random inputs
to random outputs.
These outputs will then be deterministically mapped to points
on an \gls{elliptic curve}.

%%%%%%%%%%%%%%%%%%%%%%%%%%%%%%%%%%%%%%%%%%%%%%%%%%%%%%%%%%%%%%%%%%%%%%%%
%%% Preamble: New Commands

\newcommand*{\norm}[1]{\left| \left| #1 \right| \right|}		
\newcommand*{\angles}[1]{\left\langle #1 \right\rangle}
\newcommand*{\abs}[1]{\left| #1 \right|}

\newcommand*{\parens}[1]{\left( #1 \right)}
\newcommand*{\brackets}[1]{\left[ #1 \right]}	
\newcommand*{\braces}[1]{\left\{ #1 \right\}}
\newcommand*{\ceil}[1]{\left\lceil #1 \right\rceil}
\newcommand*{\floor}[1]{\left\lfloor #1 \right\rfloor}

\DeclareMathOperator{\sgn}{sgn}
\DeclareMathOperator{\sign}{sign}
\DeclareMathOperator{\dlog}{dlog}

\newcommand*{\del}[0]{\delta}
\newcommand*{\eps}[0]{\varepsilon}
\newcommand*{\vphi}[0]{\varphi}
\newcommand*{\N}[0]{\mathbb{N}}
\newcommand*{\R}[0]{\mathbb{R}}
\newcommand*{\Q}[0]{\mathbb{Q}}
\newcommand*{\Z}[0]{\mathbb{Z}}
\newcommand*{\C}[0]{\mathbb{C}}
\newcommand*{\F}[0]{\mathbb{F}}
\renewcommand{\P}[0]{\mathbb{P}} % Overwrites Paragraph symbol

\newcommand*{\equalQ}[0]{\stackrel{?}{=}}

% Make Footnotes on Titlepage better
\thanksmarkseries{fnsymbol}

% Set DateTime2 information
\DTMsavedate{PublicRelease}{2022-06-28}

% Comment section
\newcommand*{\chgcomment}[1]{\todo[author=CHG,color=green,inline]{#1}}

% New commands
\DeclareRobustCommand*{\MDFive}[0]{MD5}
\DeclareRobustCommand*{\ShaOne}[0]{SHA\nobreakdash-1}
\DeclareRobustCommand*{\ShaTwo}[0]{SHA\nobreakdash-2}
\DeclareRobustCommand*{\ShaThree}[0]{SHA\nobreakdash-3}
\DeclareRobustCommand*{\Keccak}[0]{\textsc{Keccak}}
\DeclareRobustCommand*{\MD}[0]{Merkle-Damg\r{a}rd}

\newcommand*{\exampleCodeReference}[1]{Code may be found at~\texttt{#1}.}

\newcommand*{\chooseRandom}[0]{\stackrel{\$}{\leftarrow}}
\newcommand*{\mathDef}[0]{\coloneqq}

\newcommand*{\algAssign}[0]{\coloneqq}
\newcommand*{\Break}[0]{\State \textbf{break}}
\newcommand*{\Continue}[0]{\State \textbf{continue}}

\newcommand*{\PairingCheck}[0]{\textsc{PairingCheck}}

% DKG Phases and Commands
\newcommand*{\sk}[0]{\textrm{sk}}
\newcommand*{\pk}[0]{\textrm{pk}}
\newcommand*{\mpk}[0]{\textrm{mpk}}
\newcommand*{\msk}[0]{\textrm{msk}}
\newcommand*{\gpk}[0]{\textrm{gpk}}
\newcommand*{\gsk}[0]{\textrm{gsk}}

\newcommand*{\Enc}[0]{\textsc{Encrypt}}
\newcommand*{\Dec}[0]{\textsc{Decrypt}}

\newcommand*{\Registration}[0]{Registration Phase}
\newcommand*{\ShareSubmission}[0]{Share Submission Phase}
\newcommand*{\ShareDispute}[0]{Share Dispute Phase}
\newcommand*{\KeyShareSubmission}[0]{Key Share Submission Phase}
\newcommand*{\MPKSubmission}[0]{MPK Submission Phase}
\newcommand*{\GPKSubmission}[0]{GPK Submission Phase}
\newcommand*{\GPKDispute}[0]{GPK Dispute Phase}
\newcommand*{\Completion}[0]{Completion Phase}

% Fancyhdr
\pagestyle{fancy}
\renewcommand{\chaptermark}[1]{\markboth{\chaptername\ \thechapter.\ #1}{}}
%\fancyhead[L]{\nouppercase{\leftmark}}
%\fancyhead[L]{\textsl{\MakeUppercase{\leftmark}}}
\fancyhead[L]{\textsl{\nouppercase{\leftmark}}}
\fancyhead[C]{}
\fancyhead[R]{\thepage}
\fancyfoot[L]{}
\fancyfoot[C]{}
\fancyfoot[R]{}
\renewcommand{\headrulewidth}{0.4pt}
%\renewcommand{\footrulewidth}{0pt}



% New definition of square root:
% it renames \sqrt as \oldsqrt
\let\oldsqrt\sqrt
% it defines the new \sqrt in terms of the old one
\def\sqrt{\mathpalette\DHLhksqrt}
\def\DHLhksqrt#1#2{%
\setbox0=\hbox{$#1\oldsqrt{#2\,}$}\dimen0=\ht0
\advance\dimen0-0.2\ht0
\setbox2=\hbox{\vrule height\ht0 depth -\dimen0}%
{\box0\lower0.4pt\box2}}



%\newtheoremstyle{mythm}% name of the style to be used
%  {parskip}% measure of space to leave above the theorem. E.g.: 3pt
%  %{spaceabove}% measure of space to leave above the theorem. E.g.: 3pt
%  {parskip}% measure of space to leave below the theorem. E.g.: 3pt
%  %{spacebelow}% measure of space to leave below the theorem. E.g.: 3pt
%  {}% name of font to use in the body of the theorem
%  %{bodyfont}% name of font to use in the body of the theorem
%  {\quad}% measure of space to indent
%  %{indent}% measure of space to indent
%  {}% name of head font
%  %{headfont}% name of head font
%  {}% punctuation between head and body
%  %{headpunctuation}% punctuation between head and body
%  {}% space after theorem head; " " = normal interword space
%  %{headspace}% space after theorem head; " " = normal interword space
%  {}
%  %{\thmname{#1}\thmnumber{ #2}:\thmnote{ #3}}
%  %{headspec}% Manually specify head

\newtheoremstyle{mythm}% name
  {12pt}%Space above
  %{3pt}%Space above
  {12pt}%Space below
  {\normalfont}%Body font
  {0pt}%Indent amount
  {\bf}% Theorem head font
  %{\itshape}% Theorem head font
  {}%Punctuation after theorem head
  %{.}%Punctuation after theorem head
  {\newline}%Space after theorem head 2
  {}%Theorem head spec (can be left empty, meaning ‘normal’)

\theoremstyle{mythm}
\newtheorem{thm}{Theorem}[chapter] % number by chapter; can replace with section
\newtheorem{cor}[thm]{Corollary}
\newtheorem{lem}[thm]{Lemma}
\newtheorem{prop}[thm]{Proposition}
\newtheorem{ax}{Axiom}
\newtheorem{problem}[thm]{Problem}
\newtheorem{example}[thm]{Example}
%
\theoremstyle{definition}
\newtheorem{defn}{Definition}[chapter] % number by chapter
%
\theoremstyle{remark}
\newtheorem{rem}{Remark}[chapter]
\newtheorem*{notation}{Notation}
%
\newtheorem{hw}[thm]{Homework}

% Allow equations to break over pages (?); from Kyle
%\allowdisplaybreaks

\lstset{language=Python,
    keywordstyle=\color{blue}\bfseries,
    commentstyle=\color{green!50!black},
    stringstyle=\ttfamily\color{red!50!brown},
    showstringspaces=false}

% Replace "Listings" with "List of Listings"
\renewcommand*{\lstlistlistingname}{List of Listings}

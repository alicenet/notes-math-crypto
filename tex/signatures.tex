\chapter{Digital Signatures}
\label{chap:signatures}

This chapter is devoted to \glspl{signature}.
We will describe what they are as well as go through examples.



\section{The Need for Digital Signatures}

\Glspl{signature} allow for \emph{nonrepudiation};
that is, if Alice sends Bob a message $m$ with a valid \gls{signature}
$\sigma$ of $m$, Bob can be certain that Alice did indeed
send him that message.
Only if Eve stole Alice's signing (private) key would Eve be able
to forge a signature.
Thus, if Alice's signing key remains secure, it is not possible
for Alice to later claim that she did not sign the message $m$.
In particular, we want to ensure that Bob can present $\parens{m,\sigma}$
to Charlie as proof that Alice sent message $m$ to Bob.

We note that \glspl{signature} are different from \glspl{mac};
see Chapter~\ref{sec:symmetric_mac}.
A \gls{mac} \emph{does not} allow for nonrepudiation.
Because of this, Bob is not able to use a MAC to prove to Charlie
that Alice sent him a message because both Bob and Alice
have the private key used in a MAC.
Even though MACs do not allow for nonrepudiation,
the fact remains that they require \emph{orders of magnitude}
less computation.



\section{Clearing Up Potential Confusion about Digital Signatures}

\Glspl{signature} are sometimes explained as
``performing \gls{public key encryption} backwards.''
Although from a historical perspective this is understandable,
this is \emph{completely incorrect}.
\Glspl{signature} \emph{should not} be thought of as the ``reverse''
of encryption.
The \emph{purpose} of a \gls{signature} is enable Bob to prove to Charlie
that Alice sent him a message;
encryption \emph{does not} play any role.

Now, it is understandable how this confusion arose.
In the original RSA paper~\cite{RSApaper},
\gls{public key encryption} and \glspl{signature} are both discussed.
The authors describe a digital signature scheme which is merely
their \gls{public key encryption} scheme backwards.
Of course, the original method described in~\cite{RSApaper}
\emph{should not be used} because it is \emph{not secure}.
Additionally, it should be noted that Ralph Merkle in his
dissertation~\cite[Chapter~1.4]{merkle1979secrecy}
makes a similar statement.
As modern cryptography developed, it became clear that it was best
to separate the notions of an \gls{encryption scheme}
and \gls{signature} scheme due to the different security objectives.

This history explains why some may incorrectly describe \glspl{signature}
as ``reverse encryption.''



\section{Elgamal Signatures}
\label{sec:signatures_elgamal}

Elgamal signatures~\cite{elgamal1985public} are related to
Elgamal encryption previously discussed
in Chapter~\ref{sec:public_elgamal_encryption}.

\subsection{Formal Definition}

\begin{defn}
We fix a prime $p$ and let $g\in\F_{p}^{*}$ generate $\F_{p}^{*}$.
Also, let $H:\braces{0,1}^{*}\to\Z_{p-1}$ be a \gls{hash function}.

\paragraph{Key Generation}
Choose $x\chooseRandom{}\Z_{p-1}^{*}$.
Set $y = g^{x} \mod p$.
Then $x$ is the private key and $y$ is the public key.

\paragraph{Signature Generation}
Choose a \gls{nonce} $k\chooseRandom{}\Z_{p-1}^{*}$
such that $\gcd(k,p-1) = 1$.
Set

\begin{align}
    r &= g^{k} \mod p \nonumber\\
    s &= \brackets{H(m) - xr}k^{-1} \mod p-1.
\end{align}

\noindent
If $s=0$, choose another $k$ and repeat the process.
The signature is $\parens{r,s}$.

\paragraph{Signature Verification}
The signature $\parens{r,s}$ is valid if $r,s\in\F_{p}^{*}$ and

\begin{equation}
    g^{H(m)} = y^{r}r^{s} \mod p.
\end{equation}
\end{defn}

\subsection{Verification}

Suppose we receive $\parens{r,s}$ as previously defined.
In this case, we see

\begin{equation}
    H(m) = ks + xr \mod (p-1).
\end{equation}

\noindent
This allows the following equalities:

\begin{align}
    g^{H(m)} &= g^{ks + xr} \nonumber\\
        &= \parens{g^{k}}^{s}\parens{g^{x}}^{r} \nonumber\\
        &= r^{s}y^{r}.
\end{align}

\noindent
Thus, we have the desired equality.

\subsubsection{The Necessary Requirement of $s\ne0$}.

We briefly describe why we must have $s\ne0$.
Suppose $s = 0$.
Then we have

\begin{equation}
    H(m) = xr \mod (p-1).
\end{equation}

\noindent
This implies that

\begin{equation}
    x = H(m)r^{-1} \mod (p-1).
\end{equation}

\noindent
Thus, if $s=0$, then Alice leaks her private key.
\emph{This should never happen.}

\subsection{Security Considerations}

If the \gls{nonce} $k$ is reused, the security of the scheme breaks
because the private key can be computed;
the reader should work out the details.
A similar derivation can be found in Chapter~\ref{sig:schnorr_security}.

\subsection{Example}

\begin{example}[Elgamal Signature]
\exampleCodeReference{examples/signatures/elgamal\_signature.py}

We will again work within $\F_{p}$ with $p = 7919$
and base point $g = 7$.
Alice chooses her private key $x=5654$;
her public key is

\begin{align}
    y &= g^{x} \mod p \nonumber\\
        &= 5581.
\end{align}

\noindent
Alice has a message $m$ she wishes to sign.
We have $H(m) = 2689$.

\paragraph{Signature Generation}
Alice chooses her \gls{nonce} $k=3331$;
we note $\gcd(7918, 3331) = 1$.
Additionally, we see
In this case, we have

\begin{align}
    r &= g^{k} \mod p \nonumber\\
        &= 3521.
\end{align}

\noindent
We then have

\begin{align}
    s &= \brackets{H(m) - xr}k^{-1} \mod \parens{p-1} \nonumber\\
        &= 585.
\end{align}

\noindent
The signature is $\parens{3521, 585}$.

\paragraph{Signature Verification}
We now want to verify the signature $\parens{3521, 585}$
from Alice with public key $y = 5581$.
The message $m$ has the hash $H(m) = 2689$.

We look at the left-hand side:

\begin{equation}
    g^{H(m)} \mod p = 367.
\end{equation}

\noindent
We look at the right-hand side:

\begin{equation}
    y^{r}r^{s} \mod p = 367.
\end{equation}

\noindent
Thus, our signature is valid.
\end{example}



\section{Schnorr Signatures}
\label{sec:signatures_schnorr}

One difficulty with Elgamal signatures is that they are large.
Schnorr signatures produce smaller signatures.

\subsection{Formal Definition}

\begin{defn}
We let $G = \angles{g}$ be a \gls{cyclic group} with prime order $q$.
We fix a \gls{hash function} $H:\braces{0,1}^{*}\to\Z_{q}$.
Here, $m\in\braces{0,1}^{*}$ is the message to be signed.

\paragraph{Key Generation}
Let $x\chooseRandom{}\Z_{q}^{*}$ be the signing key.
Then $y = g^{x}$ is the public key.

\paragraph{Signature Generation}
Choose a \gls{nonce} $k\chooseRandom{}\Z_{q}^{*}$.
Set $r = g^{k}$.
Let

\begin{equation}
    e = H(r||m);
\end{equation}

\noindent
here, we have a deterministic binary representation for $r$.
Set

\begin{equation}
    s = k - xe \mod q.
\end{equation}

\noindent
The signature is $\parens{s,e}$.

\paragraph{Signature Verification}
We receive the message $m$ along with $\parens{s,e}$.
Let

\begin{align}
    r_{v} &= g^{s}y^{e}
        \nonumber\\
    e_{v} &= H(r_{v}||m).
\end{align}

\noindent
The signature is valid if $e_{v} = e$.
\end{defn}

\subsection{Verification}

We now verify the result.
In this case, if we use the values as defined, we see

\begin{align}
    r_{v} &= g^{s}y^{e} \nonumber\\
        &= g^{k-xe}g^{xe} \nonumber\\
        &= g^{k}.
\end{align}

\noindent
Therefore, $r_{v} = r$ and

\begin{align}
    e_{v} &= H(r_{v}||m) \nonumber\\
        &= H(r||m) \nonumber\\
        &= e.
\end{align}

\noindent
Thus, the signature is valid.

\subsection{Group Realization}

Here, we presented the Schnorr signature in a general \gls{cyclic group}.
In practice, this was originally designed to be implemented
in a \gls{subgroup} of order $q$ in $\F_{p}$ for large prime $p$.
This is similar to the situation in DSA
in Chapter~\ref{sec:signatures_dsa}.
More explicitly, let $p$ be a large prime with

\begin{equation}
    p = rq + 1.
\end{equation}

\noindent
Furthermore, let $h\in\F_{p}^{*}$ be a primitive element;
that is, let $h$ be a generator so that $\angles{h} = \F_{p}^{*}$.
Then set

\begin{equation}
    g = h^{r} \mod p.
\end{equation}

\noindent
In this case, $\abs{\angles{g}} = q$ and so $\angles{g}$
is the desired \gls{cyclic group} of prime order $q$.

\subsection{Security Considerations}
\label{sig:schnorr_security}

Care must be taken to ensure the \gls{nonce} is not reused.
\Gls{nonce} reuse easily leads to recovery of the private key.
In fact, mere \emph{bias} in the \gls{nonce} distribution
can lead to private key recovery;
see~\cite{cryptoeprint:2019:023}.
The use of deterministic \glspl{nonce} have been recommended
for some time~\cite{rfc6979}.

\subsubsection{Private Key Recovery from Nonce Reuse}

We show what happens when the \gls{nonce} is reused.

Suppose Alice signs distinct messages $m$ and $m'$
but reuses the \gls{nonce} $k$ to save computation.
That is, we have $r = g^{k}$ with

\begin{align}
    e &= H(r||m) \nonumber\\
    s &= k - xe \nonumber\\
    e' &= H(r||m') \nonumber\\
    s' &= k - xe'.
\end{align}

\noindent
Somehow, the adversary Eve knows that $k$ is reused;
this could be determined by verifying the signatures
and then noticing the computed $r$ values are the same.
Using this information, we see

\begin{align}
    s - s' &= \parens{k - xe} - \parens{k - xe'} \nonumber\\
        &= x\parens{e'-e},
\end{align}

\noindent
so it follows that

\begin{equation}
    x = \frac{s-s'}{e'-e} \mod q.
\end{equation}

\noindent
Therefore, Eve has now learned Alice's private key.

\subsection{Example}

\begin{example}[Schnorr Signature]
\label{example:schnorr_signature}
\exampleCodeReference{examples/signatures/schnorr\_signature.py}

We have the following:

\begin{align}
    p &= cq + 1 \nonumber\\
    q &= 2^{32} + 15 \nonumber\\
    c &= 12.
\end{align}

\noindent
It can be shown that $p$ and $q$ are prime numbers
and that

\begin{equation}
    h = 7
\end{equation}

\noindent
is a generator for $\F_{p}^{*}$.
Our $q$-order \gls{subgroup} is generated by

\begin{align}
    g &= h^{c} \mod p \nonumber\\
        &= 13841287201.
\end{align}

\noindent
See Chapter~\ref{sec:zkproofs_group} for more information.

Alice chooses her private key and public key so that

\begin{align}
    x &= 2981464014 \nonumber\\
    y &= g^{x} \mod p \nonumber\\
        &= 2966837275.
\end{align}

\noindent
Alice wants to sign the message

\begin{equation}
    m = \texttt{00010203}.
\end{equation}

\paragraph{Signature Generation}
Alice chooses

\begin{align}
    k &= 458074281 \nonumber\\
    r &= g^{k} \mod p \nonumber\\
        &= 2699985659.
\end{align}

\noindent
She then computes

\begin{align}
    e &= \text{\MDFive{}}\parens{r||m} \mod q \nonumber\\
        &= 1756904634.
\end{align}

\noindent
We then have

\begin{align}
    s &= \parens{k - xe} \mod q \nonumber\\
        &= 1743605188.
\end{align}

\noindent
The signature is $\parens{1743605188, 1756904634}$.

\paragraph{Signature Verification}
Bob receives $\parens{1743605188, 1756904634}$
along with the message $m = \texttt{00010203}$.
Alice has the public key

\begin{equation}
    y = 2966837275.
\end{equation}

\noindent
Bob computes

\begin{align}
    r_{v} &= g^{s}y^{e} \mod p \nonumber\\
        &= 2699985659.
\end{align}

\noindent
He then sees

\begin{align}
    e_{v} &= \text{\MDFive{}}\parens{r_{v}||m} \mod q \nonumber\\
        &= 1756904634.
\end{align}

\noindent
Bob determines the signature is valid because $e = e_{v}$.
\end{example}

\subsection{Schnorr Signature Patent}

Claus Schnorr held the patent on Schnorr signatures until it expired in 2008.
This led to the development of the Digital Signature Algorithm
(discussed in Chapter~\ref{sec:signatures_dsa}).



\section{Digital Signature Algorithm (DSA)}
\label{sec:signatures_dsa}

The Digital Signature Algorithm (DSA) is another signature
algorithm.
It is more complicated than a Schnorr signature and based
on the Elgamal signature scheme in Chapter~\ref{sec:signatures_elgamal}.

\subsection{Formal Definition}

\begin{defn}
\label{def:dsa}
Let $p$ be a prime.
Choose $g\in\F_{p}$ such that $\abs{\angles{g}} = q$, a prime.
Fix a \gls{hash function} $H:\braces{0,1}^{*}\to\Z_{q}$.
Here, $m\in\braces{0,1}^{*}$ is the message to be signed.

\paragraph{Key Generation}
Choose $x\chooseRandom{}\Z_{q}^{*}$ and let $y = g^{x}\mod p$.
We have public key $y$ and private key $x$.

\paragraph{Signature Generation}
Choose a \gls{nonce} $k\chooseRandom{}\Z_{q}^{*}$.
Set

\begin{equation}
    r = \brackets{g^{k}\mod p} \mod q.
\end{equation}

\noindent
If $r = 0$, restart and choose another $k$.
Set

\begin{equation}
    s = k^{-1}\brackets{H(m) + xr} \mod q.
\end{equation}

\noindent
If $s = 0$, restart and choose another $k$.
The signature is $\parens{r,s}$.

\paragraph{Signature Verification}
Let $m$ be a message with signature $\parens{r,s}$.
First, verify $0<r<q$ and $0<s<q$.
Set

\begin{align}
    w &= s^{-1} \mod q \nonumber\\
    u_{1} &= H(m)\cdot w \mod q \nonumber\\
    u_{2} &= r\cdot w \mod q \nonumber\\
    v &= \brackets{g^{u_{1}}y^{u_{2}} \mod p} \mod q.
\end{align}

\noindent
The signature is valid if $v = r$.
\end{defn}

\subsection{Verification}

Suppose we have message $m$ and signature $\parens{r,s}$ as defined above.
Then we have

\begin{align}
    u_{1} &= H(m)\cdot s^{-1} \mod q \nonumber\\
    u_{2} &= r\cdot s^{-1} \mod q.
\end{align}

\noindent
It follows that

\begin{align}
    g^{u_{1}}y^{u_{2}} &= g^{H(m)s^{-1}} y^{rs^{-1}} \nonumber\\
        &= g^{H(m)s^{-1}} g^{xrs^{-1}} \nonumber\\
        &= g^{\brackets{H(m) + xr}s^{-1}} \nonumber\\
        &= g^{k}.
\end{align}

\noindent
Therefore, we see

\begin{align}
    v &= \brackets{g^{u_{1}}y^{u_{2}} \mod p} \mod q \nonumber\\
        &= \brackets{g^{k} \mod p} \mod q \nonumber\\
        &= r,
\end{align}

\noindent
as desired.

\subsubsection{The Necessary Requirement of $r\ne0$}
If $r=0$, then we also have $u_{2}=0$.
This implies that we have a valid signature for \emph{any}
public key; additionally, $k$ can also be recovered.
Thus, we cannot let $r=0$.

\subsubsection{The Necessary Requirement of $s\ne0$}
If $s=0$, then we cannot compute the modular inverse in verification.
Thus, we cannot verify the signature.

\subsection{Security Considerations}

Reusing the \gls{nonce} $k$ can lead to recovery of the private key.
The reader should work out the details.

\subsection{Example}

\begin{example}[DSA Signature]
\exampleCodeReference{examples/signatures/dsa\_signature.py}

We use the same \gls{group} from Example~\ref{example:schnorr_signature};
namely, we have

\begin{align}
    p &= cq + 1 \nonumber\\
    q &= 2^{32} + 15 \nonumber\\
    c &= 12 \nonumber\\
    h &= 7 \nonumber\\
    g &= h^{c} \mod p \nonumber\\
        &= 13841287201.
\end{align}

Alice chooses her private key and public key so that

\begin{align}
    x &= 3713319676 \nonumber\\
    y &= g^{x} \mod p \nonumber\\
        &= 43016495308.
\end{align}

\noindent
Alice wants to sign the same message

\begin{equation}
    m = \texttt{00010203}
\end{equation}

\noindent
as before; this time, she wants to use DSA.

\paragraph{Signature Generation}
Alice chooses

\begin{align}
    k &= 2073840447 \nonumber\\
    r &= \brackets{g^{k} \mod p} \mod q \nonumber\\
        &= 1886505440.
\end{align}

\noindent
She then computes

\begin{align}
    s &= k^{-1}\brackets{H(m) + xr} \mod q \nonumber\\
        &= 2123944771.
\end{align}

\noindent
The signature is $\parens{1886505440, 2123944771}$.

\paragraph{Signature Verification}
Bob receives $\parens{1886505440, 2123944771}$
along with the message $m = \texttt{00010203}$.
Alice has the public key

\begin{equation}
    y = 43016495308.
\end{equation}

\noindent
Bob computes

\begin{align}
    w &= s^{-1} \mod q \nonumber\\
        &= 1156096597 \nonumber\\
    u_{1} &= H(m)\cdot w \mod q \nonumber\\
        &= 938259556 \nonumber\\
    u_{2} &= r\cdot w \mod q \nonumber\\
        &= 75869898 \nonumber\\
    v &= \parens{g^{u_{1}}y^{u_{2}} \mod p} \mod q \nonumber\\
        &= 1886505440.
\end{align}

\noindent
Bob determines the signature is valid because $r = v$.
\end{example}



\section{Problems with DSA}

There are some critical problems with DSA.

It is critical that the \gls{nonce} $k$ be sufficiently random.
As noted above, \gls{nonce} reuse allows for the recovery of the signing key.
Furthermore, if $k$ is not chosen uniformly
but comes from a biased distribution as a result of
poor random number generation,
then information about the signing key can be leaked.
This information may lead to signing key recovery;
see~\cite{PartialNonces2002}.
Because of this, it is \emph{critical} that $k$ be chosen carefully.
It is \emph{recommended} to deterministically choose $k$~\cite{rfc6979}.

The fact there are subexponential algorithms for factoring
integers leads to subexponential algorithms for breaking DSA.
This is discussed more in Chapter~\ref{sec:Fp_problems}.



\section{Comparison of Digital Signature Schemes}
\label{sec:signatures_comparison}

We now perform a comparison between the different signature
algorithms and why one might prefer one to the others.

Given the standard security requirements
(discussed in Chapter~\ref{chap:hardness}; see Table~\ref{table:key_lengths}),
the above signature algorithms require $p$ to be a 3072-bit prime.
In the case of DSA, $q$ is required to be a 256-bit prime.

\subsection{Public Key Length}

We begin by discussing the public key length:

\begin{itemize}
    \item Elgamal: 3072 bits; 384 bytes
    \item Schnorr: 3072 bits; 384 bytes
    \item DSA: 3072 bits; 384 bytes
\end{itemize}

\noindent
In each case, the public key length is the same.

\subsection{Signature Length}

We now discuss the signature length:

\begin{itemize}
    \item Elgamal: $2\cdot3072$ bits; 768 bytes
    \item Schnorr: $3072 + 256$ bits; 416 bytes
    \item DSA: $2\cdot 256$ bits; 64 bytes
\end{itemize}

\noindent
In each case, DSA clearly has the smallest signature.
Schnorr also produces a small signature.

\subsection{Computational Complexity}

Finally, we discuss the computational complexity for signature generation:

\begin{itemize}
    \item Elgamal: 1 exponentiation, 1 modular inverse, 1 hash
    \item Schnorr: 1 exponentiation, 1 hash
    \item DSA: 1 exponentiation, 1 modular inverse, 1 hash
\end{itemize}

\noindent
Here, we see that DSA and Elgamal have essentially the same complexity
for signing.
We remember that when working in $\F_{p}$, exponentiation
and modular inversion have similar complexity.
Furthermore, hashing is cheap.
Thus, Schnorr signatures have the smallest complexity.

\subsection{Conclusion}

After comparing major aspects of these \glspl{signature},
we see that Schnorr signatures can be computed the fastest
while having signatures that are not too large.
One challenge, as noted above, is that Schnorr signatures
were patented.
For this reason, DSA was developed by NIST to be algorithm
that everyone could use.
It results in a smaller signature but larger computational
cost when compared with Schnorr.

As noted, \emph{all} of these algorithms have problems when
$k$ is not uniformly random.
Some attacks can be found~\cite{cryptoeprint:2019:023,PartialNonces2002}.
The use of deterministic \glspl{nonce} is
\emph{strongly encouraged}~\cite{rfc6979}.



\section{Malleable Signatures}

Similar to \glspl{encryption scheme}, \gls{signature} schemes may also
be malleable.
We mention that the original RSA signature algorithm~\cite{RSApaper}
is malleable.

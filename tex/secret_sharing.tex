\chapter{Secret Sharing Protocols}
\label{chap:secret_sharing}

We now consider the situation where individuals wish to share
secret information.
Perhaps a collection of individuals wish to split a private key
between multiple parties;
this way, a specified threshold will need to work together
to use the private key.

If the reader is not familiar with \gls{lagrange interpolation},
it would be a good idea for him to review the material
in Chapter~\ref{sec:math_lagrange}, as it will be used extensively
in this chapter.
Furthermore, the reader is assumed to also be familiar with
\glspl{bilinear} and BLS signatures as discussed in Chapter~\ref{chap:pairing}.

\emph{Notation Convention:}
Throughout this chapter we will be looking at
$\parens{t,n}$-threshold schemes.
By this we mean that $t+1$ participants are required to work together
in order to learn the secret
and that $t$ individuals working together gain no additional information.
We point this out because there are conflicting conventions
throughout the literature.

\section{The Need for Secret Sharing Protocols}

Suppose that Alice, Bob, and Charlie start a business.
Being equals, they want a majority to be able to withdraw
money for business expenses at any time.
At the same time, they do not fully trust each other.
They would like to split the private key between the three of them
so that 2 out of 3 participants will be required to open the vault.
Alice, Bob, and Charlie will need to use a form of secret sharing
to share their private vault key.
The challenge is that Alice should know
\emph{nothing} about the private key without assistance from either
Bob or Charlie.

We will start by discussing the original secret sharing
protocol (Shamir's Secret Sharing)~\cite{shamir1979share}
in Chapter~\ref{sec:ss_shamir}.
From here, we will allow for the possibility of malicious
actors in Chapter~\ref{sec:ss_verifiable}.
Finally, we discuss the solution to Alice, Bob, and Charlie's
problem when we give an overview of \gls{distributed key generation} (DKG)
in Chapter~\ref{sec:ss_dkg};
the DKG protocol is discussed in Chapter~\ref{sec:ss_dkg_protocol}.
As a follow up to DKG, we discuss how to make group signatures
in Chapter~\ref{sec:ss_threshold}.

\section{Shamir's Secret Sharing}
\label{sec:ss_shamir}

\subsection{Setup}

We now discuss Shamir's Secret Sharing protocol~\cite{shamir1979share}.

We want a $\parens{t,n}$-sharing protocol,
where $n$ is the total number of participants
and we require $t+1$ secret shares to recover the original secret.
To that end, let $P_{i}$ be the $i$th participant
and let $\mathcal{P} = \braces{P_{i}}_{i=1}^{n}$ be the set
of all participants.
\emph{Note well:} this protocol requires \emph{trusted setup};
that is, whoever is producing the secret shares must be honest.

To do this, let $q\in\N$ be a prime sufficiently large.
Also, let $s\chooseRandom{}\F_{q}$ be the secret that we wish to distribute.
Additionally, we choose random coefficients
$c_{1}, c_{2}, \cdots, c_{t} \chooseRandom{} \F_{q}$;
we set $c_{0} = s$.
At this point, we have the secret polynomial

\begin{equation}
    p(x) = c_{0} + c_{1}x + c_{2}x^{2} + \cdots + c_{t}x^{t}.
\end{equation}

We use this secret polynomial to derive the secret shares.
We let $\parens{i, p(i)}$ be the share for participant $P_{i}$.
Because $\braces{c_{i}}_{i=0}^{n}$ are chosen uniformly at random in $\F_{q}$,
$p(i)$ will appear to be randomly distributed in $\F_{q}$.

\subsection{Secret Reconstruction}

We now show how to reconstruct the secret $s = c_{0}$.

We let $\mathcal{R}\subseteq\mathcal{P}$
be a valid subset of participants who are able to reconstruct the secret;
that is, we have $\abs{\mathcal{R}} = t+1$.
We now denote the secret share for participant $P_{k}$ as

\begin{equation}
    s_{k}\mathDef{} p(k).
\end{equation}

\noindent
This is meant to simplify the equations below.

We review some information related to \gls{lagrange interpolation}
from Chapter~\ref{sec:math_lagrange}.
The Lagrange interpolating polynomial for
$\braces{\parens{x_{k},y_{k}}}_{k=1}^{n}$ is

\begin{align}
    L(x) &\mathDef{} \sum_{k=1}^{n} y_{k}\ell_{k}(x) \nonumber\\
    \ell_{k}(x) &\mathDef{}
        \prod_{\substack{1\le j \le n \\ j\ne k}} \frac{x-x_{j}}{x_{k}-x_{j}}.
\end{align}

\noindent
These equations were previously presented in Eq.~\eqref{eq:math_lagrange_def}.
In this setting, our data is
$\braces{\parens{k, s_{k}}}_{P_{k}\in\mathcal{R}}$.
Thus, we have the following interpolating polynomial:

\begin{align}
    L(x) &= \sum_{P_{k}\in\mathcal{R}} s_{k}\ell_{k}(x) \nonumber\\
    \ell_{k}(x) &=
        \prod_{\substack{P_{j}\in\mathcal{R} \\ j\ne k}}
            \frac{x-j}{k-j} \nonumber\\
    &= \prod_{\substack{P_{j}\in\mathcal{R} \\ j\ne k}}
            \frac{j-x}{j-k}.
\end{align}

\noindent
In order to compute the secret $s$, we need to evaluate $L(0)$.
This gives us

\begin{align}
    s &= \sum_{P_{k}\in\mathcal{R}} s_{k} R_{k} \nonumber\\
    R_{k} &= \prod_{\substack{P_{j}\in\mathcal{R} \\ j\ne k}}
        \frac{j}{j-k}.
\end{align}

\subsection{Examples}

We work through some examples to understand the entire process.

\begin{example}[Alice, Bob, and Charlie share a secret]
\exampleCodeReference{examples/secret\_sharing/secret\_sharing\_1.py}

As mentioned previously, Alice, Bob, and Charlie formed a business
and want to share a secret key to their bank account whereby
2 out of 3 must sign off on any withdrawals.
Thus, they want a $\parens{1,3}$-threshold system.

We begin by choosing $q = 7919$.
We have the secret $s = 2310$ and polynomial

\begin{equation}
    p(x) = 2310 + 4673x.
\end{equation}

\noindent
In this case, we see Alice, Bob, and Charlie receive the shares

\begin{align}
    \text{Alice} &= \parens{1, p(1)} \nonumber\\
        &= \parens{1, 6983} \nonumber\\
    \text{Bob} &= \parens{2, p(2)} \nonumber\\
        &= \parens{2, 3737} \nonumber\\
    \text{Charlie} &= \parens{3, p(3)} \nonumber\\
        &= \parens{3, 491}.
\end{align}

We now look at all three possible share combinations:
Alice and Bob, Alice and Charlie, and Bob and Charlie.

\begin{itemize}
\item Alice and Bob:

In this case, we have that

\begin{align}
    s &= s_{1}R_{1} + s_{2}R_{2} \nonumber\\
    s_{1} &= 6983 \nonumber\\
    s_{2} &= 3737 \nonumber\\
    R_{1} &= \frac{2}{2-1} \mod 7919 \nonumber\\
    R_{2} &= \frac{1}{1-2} \mod 7919.
\end{align}

\noindent
We reduce these modulo $q$ to find

\begin{align}
    R_{1} &= \frac{2}{2-1} \mod 7919 \nonumber\\
        &= 2 \nonumber\\
    R_{2} &= \frac{1}{1-2} \mod 7919 \nonumber\\
        &= -1 \mod 7919 \nonumber\\
        &= 7918.
\end{align}

\noindent
This reduces to

\begin{align}
    s &= 6983\cdot2 + 3737\cdot7918 \mod 7919 \nonumber\\
        &= 2310.
\end{align}

\noindent
Thus, we have arrived at the \gls{shared secret} $s$.

\item Alice and Charlie:

In this case, we have that

\begin{align}
    s &= s_{1}R_{1} + s_{3}R_{3} \nonumber\\
    s_{1} &= 6983 \nonumber\\
    s_{3} &= 491 \nonumber\\
    R_{1} &= \frac{3}{3-1} \mod 7919 \nonumber\\
    R_{3} &= \frac{1}{1-3} \mod 7919.
\end{align}

\noindent
In this case, we actually have to do some operations modulo $q$.
We see that

\begin{align}
    R_{1} &= \frac{3}{2} \mod 7919 \nonumber\\
        &= 3961 \nonumber\\
    R_{3} &= -\frac{1}{2} \mod 7919 \nonumber\\
        &= 3959.
\end{align}

\noindent
This reduces to

\begin{align}
    s &= 6983\cdot3961 + 491\cdot3959 \mod 7919 \nonumber\\
        &= 2310.
\end{align}

\noindent
Thus, we have arrived at the \gls{shared secret} $s$.

\item Bob and Charlie:

In this case, we have that

\begin{align}
    s &= s_{2}R_{2} + s_{3}R_{3} \nonumber\\
    s_{2} &= 3737 \nonumber\\
    s_{3} &= 491 \nonumber\\
    R_{2} &= \frac{3}{3-2} \mod 7919 \nonumber\\
    R_{3} &= \frac{2}{2-3} \mod 7919.
\end{align}

\noindent
We see that

\begin{align}
    R_{2} &= 3 \nonumber\\
    R_{3} &= \frac{2}{2-3} \mod 7919 \nonumber\\
        &= -2 \mod 7919 \nonumber\\
        &= 7917.
\end{align}

\noindent
This reduces to

\begin{align}
    s &= 3737\cdot3 + 491\cdot7917 \mod 7919 \nonumber\\
        &= 2310.
\end{align}

\noindent
Thus, we have arrived at the \gls{shared secret} $s$.
\end{itemize}

In each case, every pair of participants are able to derive the
\gls{shared secret}.
\end{example}

\begin{example}[Alice, Bob, Charlie, and Dave share a secret]
\label{example:dkg_shamir_4}
\exampleCodeReference{examples/secret\_sharing/secret\_sharing\_2.py}

After making some good progress on their business,
Alice, Bob, and Charlie decided to add Dave as a business partner.
Again, with all treating each other as equals, they want
the ability for a strict majority of them to be required to derive
the secret key for the bank account.
Thus, they now need to create a new secret and use a $\parens{2,4}$
threshold system.

As before, they choose $q = 7919$.
This time, they have the secret $s = 1770$ and the secret polynomial

\begin{equation}
    p(x) = 1770 + 6540x + 2889x^{2}.
\end{equation}

\noindent
In this case, we see Alice, Bob, Charlie, and Dave receive the shares

\begin{align}
    \text{Alice} &= \parens{1, p(1)} \nonumber\\
        &= \parens{1, 3280} \nonumber\\
    \text{Bob} &= \parens{2, p(2)} \nonumber\\
        &= \parens{2, 2649} \nonumber\\
    \text{Charlie} &= \parens{3, p(3)} \nonumber\\
        &= \parens{3, 7796} \nonumber\\
    \text{Dave} &= \parens{4, p(4)} \nonumber\\
        &= \parens{4, 2883}.
\end{align}

We will look at two examples of deriving the \gls{shared secret}:
the triple Alice, Bob, and Dave as well as the triple Alice, Charlie, and Dave.

\begin{itemize}
\item Alice, Bob, and Dave:

In this case, we have that

\begin{align}
    s &= s_{1}R_{1} + s_{2}R_{2} + s_{4}R_{4} \nonumber\\
    s_{1} &= 3280 \nonumber\\
    s_{2} &= 2649 \nonumber\\
    s_{4} &= 2883 \nonumber\\
    R_{1} &= \frac{2}{2-1}\cdot\frac{4}{4-1} \mod 7919 \nonumber\\
    R_{2} &= \frac{1}{1-2}\cdot\frac{4}{4-2} \mod 7919 \nonumber\\
    R_{4} &= \frac{1}{1-4}\cdot\frac{2}{2-4} \mod 7919.
\end{align}

\noindent
Performing these operations we find

\begin{align}
    R_{1} &= \frac{2}{2-1}\cdot\frac{4}{4-1} \mod 7919 \nonumber\\
        &= \frac{8}{3} \mod 7919 \nonumber\\
        &= 5282 \nonumber\\
    R_{2} &= \frac{1}{1-2}\cdot\frac{4}{4-2} \mod 7919 \nonumber\\
        &= -2 \mod 7919 \nonumber\\
        &= 7917 \nonumber\\
    R_{4} &= \frac{1}{1-4}\cdot\frac{2}{2-4} \mod 7919 \nonumber\\
        &= \frac{1}{3} \mod 7919 \nonumber\\
        &= 2640.
\end{align}

\noindent
This reduces to

\begin{align}
    s &= 3280\cdot5282 + 2649\cdot7917 + 2883\cdot2640 \mod 7919
        \nonumber\\
    &= 1770.
\end{align}

\noindent
Thus, we have arrived at the \gls{shared secret} $s$.

\item Alice, Charlie, and Dave:

In this case, we have that

\begin{align}
    s &= s_{1}R_{1} + s_{3}R_{3} + s_{4}R_{4} \nonumber\\
    s_{1} &= 3280 \nonumber\\
    s_{3} &= 7796 \nonumber\\
    s_{4} &= 2883 \nonumber\\
    R_{1} &= \frac{3}{3-1}\cdot\frac{4}{4-1} \mod 7919 \nonumber\\
    R_{3} &= \frac{1}{1-3}\cdot\frac{4}{4-3} \mod 7919 \nonumber\\
    R_{4} &= \frac{1}{1-4}\cdot\frac{3}{3-4} \mod 7919.
\end{align}

\noindent
Performing these operations we find

\begin{align}
    R_{1} &= \frac{3}{3-1}\cdot\frac{4}{4-1} \mod 7919 \nonumber\\
        &= 2 \nonumber\\
    R_{3} &= \frac{1}{1-3}\cdot\frac{4}{4-3} \mod 7919 \nonumber\\
        &= -2 \mod 7919 \nonumber\\
        &= 7917 \nonumber\\
    R_{4} &= \frac{1}{1-4}\cdot\frac{3}{3-4} \mod 7919 \nonumber\\
        &= 1.
\end{align}

\noindent
This reduces to

\begin{align}
    s &= 3280\cdot2 + 7796\cdot7917 + 2883\cdot1 \mod 7919
        \nonumber\\
    &= 1770.
\end{align}

\noindent
Thus, we have arrived at the \gls{shared secret} $s$.
\end{itemize}

In both cases we were able to derive the \gls{shared secret} key.
\end{example}

\subsection{Discussion}
\label{ssec:ss_shamir_discussion}

This protocol works well but has some problems.
First, it requires \emph{trusted setup}; there has to be a trusted
party who determines the secret shares and distributes them
to each participant.
Additionally, we assume each participant \emph{correctly shares}
his portion of the secret.
Furthermore, once a participant has shared his secret,
there is nothing stopping the other participants from stealing it.

The required truthfulness of the dealer as well as the other participants
has lead to further development of secret sharing protocols
which are designed to reduce these requirements.
Even so, the protocols discussed below build upon the material here.

We have not explicitly mentioned where $q$ is used.
The size of $q$ defends the system against brute force attack.
In practice, $q$ needs to be large enough so that it is impractical
to guess the key.
Thus, choosing $q$ as a 256-bit prime number would be reasonable.
Of course, all of this depends on the desired level of security.


\section{Verifiable Secret Sharing}
\label{sec:ss_verifiable}

Verifiable Secret Sharing protocols seek to combat the shortcomings
discussed in Chapter~\ref{ssec:ss_shamir_discussion}
of Shamir's Secret Sharing protocol.
Feldman's scheme~\cite{feldman1987practical}
is frequently used but by itself is not
secure~\cite{gennaro1999secure,gennaro2002revisiting}.

Due to the fact that VSS protocols are used when attempting to distribute
a key between individuals,
we do not mention VSS protocols any further but rather
proceed to discuss \gls{distributed key generation} protocols.

\section{Distributed Key Generation Overview and Setup}
\label{sec:ss_dkg}

In this section we consider the problem of
\emph{\gls{distributed key generation}}.
Here, a collection of individuals want to distribute a private key
between them in such a way that a certain portion must work
together to derive it.
This secret key will be used to digitally sign messages.
Because of the nature of the secret key,
it is \emph{imperative} that the key not actually be constructed;
rather, portions of the secret key should be used to construct
partial signatures which may then be combined in a deterministic way
to arrive at a valid \emph{group signature}.

The basis for this discussion is the Ethereum Distributed Key Generation
whitepaper~\cite{ethdkg}.
This paper uses a variation of 
Feldman's VSS scheme~\cite{feldman1987practical}
while making adjustments
based on~\cite{gennaro1999secure,gennaro2002revisiting} for security.
We seek a $\parens{t,n}$-threshold system
whereby $t+1$ participants are required to create a valid group signature.

We will focus on the actual implementation~\cite[Section 7]{ethdkg}.
While this will make our discussion specific to this particular setting,
the main ideas of \gls{distributed key generation} protocols should be clear.
Doing this will require the use of functions particular to \gls{ethereum};
see Table~\ref{table:ethereum_functions} for a description of them.

The material here will require knowledge of
\gls{pairingcrypto} from Chapter~\ref{chap:pairing} in general
and knowledge of cryptography on \gls{ethereum}~\cite{EthereumYellowpaper}
in Chapter~\ref{sec:pairing_ethereum} in particular.
Due to the fact that \glspl{smart contract} can only perform
\gls{elliptic curve}
operations in $G_{1}$, information in $G_{2}$ will be submitted
and then verified with a \PairingCheck{} call.

\begin{table}[t]
\centering
\begin{tabular}{r|l}
\hline
Function Name & Function Definition \\
\hline
\texttt{toBytes} & serialize object to its binary representation \\
\texttt{fromBytes} & deserialize object from its binary representation \\
\textsc{ECAdd} & Elliptic curve addition \\
\textsc{ECMult} & Elliptic curve scalar multiplication \\
\textsc{PairingCheck} & Validate bilinear pairing tuple \\
\hline
\end{tabular}
\caption[Ethereum Functions]{Useful functions in \gls{ethereum} for
    the \gls{distributed key generation} protocol.}
\label{table:ethereum_functions}
\end{table}


The $i$th participant will be denoted $P_{i}$
and $\mathcal{P} \mathDef{} \braces{P_{i}}_{i=1}^{n}$
will the collection of all participants.
By working together, the participants will construct
a \emph{master public key} $\mpk{}$ along with its corresponding
\emph{master secret key} $\msk{}$.
The master secret key will \emph{never} explicitly be formed;
even so, the group will be able to work together to combine partial signatures
into valid group signatures.
Threshold signatures will be discussed in Chapter~\ref{sec:ss_threshold}.

\subsection{DKG Overview}

Broadly speaking, in the \gls{distributed key generation}  protocol,
each participant is a dealer
who shares a secret key under a threshold.
Thus, participants perform \emph{simultaneous}
Shamir's Secret Sharing protocols.
The secret key for the entire group (the \emph{master secret key})
will be the sum of the individual secrets.
In order to protect against misbehavior,
this protocol contains opportunities to prove that other
participants are malicious.

\subsection{DKG Setup}

During the protocol,
we will extensively use the \gls{bilinear} $e:G_{1}\times G_{2}\to G_{T}$.
Here, we assume $\abs{G_{1}} = \abs{G_{2}} = \abs{G_{T}} = q$ for prime $q$.
We are assuming that $g_{1}, h_{1}\in G_{1}$ are independent generators;
that is, $\dlog_{g_{1}}h_{1}$ is unknown.
We also assume that $h_{2}\in G_{2}$ is a generator.
Furthermore, although not used here, we assume $H:\braces{0,1}^{*}\to G_{1}$
is a \gls{hash function} onto $G_{1}$;
this hash-to-curve function will be used when making \glspl{signature}.
These signatures will be BLS signatures~\cite{BLSSignatures}.



\section{Distributed Key Generation Protocol}
\label{sec:ss_dkg_protocol}

The \gls{distributed key generation}  protocol is split into a number
of different phases.
We begin by giving an overview of each phase before discussing them
in detail.

We note that there are variations of communication models
for different DKG protocols.
Some assume secure communication between each pair of participants
while others broadcast encrypted messages.
The particulars matter in that the communication model determines how
proofs of security are performed, but that is not our primary focus;
the goal here is to give an \emph{introduction} to DKG protocols.
In our case, we will assume the ability to broadcast messages
to all participants.

Additionally, we will assume the \gls{distributed key generation} protocol
restarts upon any malicious behavior.
While this is not strictly necessary, this may be useful in certain situations.

\subsection*{DKG Protocol Overview}

\begin{itemize}
\item \hyperref[ssec:secret_dkg_registration]{\textbf{\Registration{}:}}
    During this phase the participants register public keys
    to enable secure communication throughout the protocol.
\item \hyperref[ssec:secret_dkg_share_submission]{\textbf{\ShareSubmission{}:}}
    Each participant submits encrypted shares and commitments
    which are then broadcast to all participants.
\item \hyperref[ssec:secret_dkg_share_dispute]{\textbf{\ShareDispute{}:}}
    Anyone who incorrectly submitted his secret share information
    may be accused.
\item \hyperref[ssec:secret_dkg_key_share]{\textbf{\KeyShareSubmission{}:}}
    All users submit their portion of the master public key.
\item \hyperref[ssec:secret_dkg_mpk]{\textbf{\MPKSubmission{}:}}
    One user submits the master public key.
\item \hyperref[ssec:secret_dkg_gpk_submission]{\textbf{\GPKSubmission{}:}}
    All users submit their group public key.
\item \hyperref[ssec:secret_dkg_gpk_dispute]{\textbf{\GPKDispute{}:}}
    Anyone who incorrectly shared his group public key may be accused.
\item \hyperref[ssec:secret_dkg_completion]{\textbf{\Completion{}:}}
    The DKG process has finished.
    At this point, all information is valid and may be used to compute
    valid group signatures.
\end{itemize}

\subsection{\Registration{}}
\label{ssec:secret_dkg_registration}

Participant $P_{i}\in\mathcal{P}$ will choose a secret key
$\sk{}_{i}\chooseRandom{}\F_{q}^{*}$;
this will determine the corresponding public key

\begin{equation}
    \pk{}_{i} \mathDef{} g_{1}^{\sk{}_{i}}.
\end{equation}

\noindent
This pair $\parens{\sk{}_{i},\pk{}_{i}}$ \emph{will not}
be used for signing;
rather, it will be used for secure communication over
an \gls{insecure channel}.

The public key $\pk{}_{i}$ will be submitted by participant $P_{i}$
and broadcast to all participants.
In this way, every participant will know $P_{i}$'s public key.

\subsection{\ShareSubmission{}}
\label{ssec:secret_dkg_share_submission}

As mentioned in the overview, each participant will now perform
Shamir's Secret Sharing protocol in parallel to share a secret.

Participant $P_{i}$ will choose $s_{i}\chooseRandom{}\F_{q}$ to be his secret.
He will then choose his private polynomial
$f_{i}:\F_{q}\to\F_{q}$:

\begin{equation}
    f_{i}(x) = c_{i,0} + c_{i,1}x + c_{i,2}x^{2} + \cdots + c_{i,t}x^{t}.
    \label{eq:dkg_private_poly}
\end{equation}

\noindent
Here, $c_{i,0} = s_{i}$ and $c_{i,j}\chooseRandom{}\F_{q}$
for $j\in\braces{1,\cdots,t}$.

The secret share from $P_{i}$ to $P_{j}$ is

\begin{equation}
    s_{i\to j} \mathDef{} f_{i}(j).
\end{equation}

\noindent
In order to ensure honesty, $P_{i}$ will have the commitments

\begin{equation}
    C_{i,j} \mathDef{} g_{1}^{c_{i,j}}.
\end{equation}

\noindent
These commitments then result in the corresponding public polynomial
$F_{i}:\F_{q}\to G_{1}$:

\begin{equation}
    F_{i}(x) = C_{i,0}C_{i,1}^{x}C_{i,2}^{x^{2}} \cdots C_{i,t}^{x^{t}}.
\end{equation}

\noindent
These public polynomials make it possible to prove that secrets
were correctly shared.
This is because we have

\begin{equation}
    F_{i}(j) = g_{1}^{f_{i}(j)}.
    \label{eq:dkg_valid_secret}
\end{equation}

In order to ensure the secret shares stay secret,
the shares are encrypted under a \gls{symmetric key encryption} scheme
based on the \gls{dhke};
the algorithms for encryption and decryption will be denoted
\Enc{} and \Dec{}.
Full algorithmic details may be found
\hyperref[sssec:secret_dkg_share_encryption]{here}.

The encrypted form of the secret share from $P_{i}$ to $P_{j}$
is given by

\begin{equation}
    \overline{\texttt{s}}_{i\to j} \mathDef{}
        \Enc{}\parens{\sk{}_{i},\pk{}_{j},s_{i\to j}, j}.
\end{equation}

\noindent
Using these encrypted shares, participant $P_{i}$ broadcasts
the following message:

\begin{equation}
    \braces{
    \overline{\texttt{s}}_{i\to 1},
    \overline{\texttt{s}}_{i\to 2},
    \cdots,
    \overline{\texttt{s}}_{i\to i-1},
    \overline{\texttt{s}}_{i\to i+1},
    \cdots,
    \overline{\texttt{s}}_{i\to n},
    C_{i,0}, C_{i,1}, \cdots, C_{i,t}}.
\end{equation}

\noindent
More explicitly, all of the necessary encrypted shares are sent
along with the coefficients of the public polynomial.
Implicitly, $P_{i}$ will also send the secret share $s_{i\to i}$
to himself.
A cryptographic hash of the encrypted shares and commitments
will be stored should an accusation need to be made in the future;
accusations will be discussed below.

At this time, all participants are expected to have received all
broadcast messages.
If any participant who registered failed to submit shares,
it is possible for the protocol to continue;
that is the path chosen in~\cite{ethdkg}.
It may also be the case that the protocol could restart.
The particular choice depends on the use case.

In our case, we will assume that all participants submit share information.
After receiving share information, each share must be decrypted and verified.
If $P_{j}$ received $\overline{\texttt{s}}_{i\to j}$ from $P_{i}$,
he then computes

\begin{equation}
    \hat{s}_{i\to j} \mathDef{}
        \Dec{}\parens{\sk{}_{j}, \pk{}_{i}, \overline{\texttt{s}}_{i\to j}, j}.
\end{equation}

\noindent
If the share is valid, we will have the following equality:

\begin{equation}
    g_{1}^{\hat{s}_{i\to j}} = F_{i}(j).
    \label{eq:dkg_share_equality}
\end{equation}

\noindent
This follows from Eq.~\eqref{eq:dkg_valid_secret}.
If the above equality does not hold,
then $P_{i}$ incorrectly shared his secret with $P_{j}$.
An accusation must be performed in the next phase.

\subsubsection{Secret Share Encryption Algorithm}
\label{sssec:secret_dkg_share_encryption}

We now discuss the \gls{encryption scheme} used.
This uses a \gls{shared secret} to derive bit stream
for an XOR cipher.
The index of the participant who \emph{receives} the message is included
in order to ensure that each stream is unique.
It requires the use of a \gls{hash function} with output the size
of $\log_{2}(q)$;
that is, if $q$ is a $k$-bit number, we require the output
of the \gls{hash function} to be at least $k$ bits.
See Alg.~\ref{alg:dkg_encryption} for the full specification.

\begin{algorithm}[t]
\caption{\Glsentrytext{encryption scheme} in the \glsentrytext{distributed key generation} protocol}
\label{alg:dkg_encryption}
\begin{algorithmic}[1]
\Procedure{\Enc{}}{$\sk{}_{i}$,$\pk{}_{j}$,$s_{i\to j}$,$j$}
    \State $k \algAssign{} \pk{}_{j}^{\sk{}_{i}}$
    \Comment{Compute the \gls{dhke} \gls{shared secret}}
    \State $\texttt{kX} \algAssign{} \texttt{toBytes}(k_{x})$
    \Comment{Use the $x$ coordinate}
    \State $\texttt{j} \algAssign{} \texttt{toBytes}(j)$
    \State $\texttt{s} \algAssign{} \texttt{toBytes}(s_{i\to j})$
    \State $\texttt{hashKJ} \algAssign{} \textsc{Hash}(\texttt{kX}\|\texttt{j})$
    \State $\overline{\texttt{s}}_{i\to j} \algAssign{}
        \texttt{s} \oplus \texttt{HashKJ}$
    \State \Return $\overline{\texttt{s}}_{i\to j}$
\EndProcedure
\State
\Procedure{\Dec{}}{$\sk{}_{i}$,$\pk{}_{j}$,$\overline{\texttt{s}}_{i\to j}$,$j$}
    \State $k \algAssign{} \pk{}_{j}^{\sk{}_{i}}$
    \Comment{Compute the \gls{dhke} \gls{shared secret}}
    \State $\texttt{kX} \algAssign{} \texttt{toBytes}(k_{x})$
    \Comment{Use the $x$ coordinate}
    \State $\texttt{j} \algAssign{} \texttt{toBytes}(j)$
    \State $\texttt{hashKJ} \algAssign{} \textsc{Hash}(\texttt{kX}\|\texttt{j})$
    \State $\texttt{s} \algAssign{}
        \overline{\texttt{s}}_{i\to j} \oplus \texttt{HashKJ}$
    \State $s_{i\to j} \algAssign{} \texttt{fromBytes}(\texttt{s})$
    \State \Return $s_{i\to j}$
\EndProcedure
\State
\Procedure{\Dec{}SS}{$k$,$\overline{\texttt{s}}_{i\to j}$,$j$}
    \State $\texttt{kX} \algAssign{} \texttt{toBytes}(k_{x})$
    \Comment{Use the $x$ coordinate}
    \State $\texttt{j} \algAssign{} \texttt{toBytes}(j)$
    \State $\texttt{hashKJ} \algAssign{} \textsc{Hash}(\texttt{kX}\|\texttt{j})$
    \State $\texttt{s} \algAssign{}
        \overline{\texttt{s}}_{i\to j} \oplus \texttt{HashKJ}$
    \State $s_{i\to j} \algAssign{} \texttt{fromBytes}(\texttt{s})$
    \State \Return $s_{i\to j}$
\EndProcedure
\end{algorithmic}
\end{algorithm}



\subsection{\ShareDispute{}}
\label{ssec:secret_dkg_share_dispute}

We now assume that Eq.~\eqref{eq:dkg_share_equality} does not hold;
that is, $P_{i}$ sent $P_{j}$ an invalid share.
We now require that $P_{j}$ \emph{prove} that $P_{i}$ is malicious.

We know from Eq.~\eqref{eq:dkg_valid_secret} what the desired
equality should be.
Thus, to prove $P_{i}$'s share is invalid,
we must decrypt the encrypted share and then show the desired equality
of Eq.~\eqref{eq:dkg_share_equality} does not hold.
From the decryption algorithm in Alg.~\ref{alg:dkg_encryption},
we know that we must compute the Diffie-Hellman shared secret.

This Diffie-Hellman shared secret $k_{ij}$ can easily be computed.
The challenge is how can we \emph{prove} this is,
in fact, the shared secret?
We recall that we have the following equalities:

\begin{align}
    g_{1}^{\sk_{j}} &= \pk{}_{j} \nonumber\\
    \pk{}_{i}^{\sk_{j}} &= k_{ij}.
\end{align}

\noindent
Because the public keys $\pk{}_{i}$ and $\pk{}_{j}$ are public knowledge,
if we can show that the pairs $\parens{g_{1},\pk_{j}}$ and
$\parens{\pk{}_{i}, k}$ have the same \gls{discrete log},
then this \emph{proves} that $k$ is actually the shared secret $k_{ij}$.
To do this, we can use \gls{discrete log} equality proofs
from Chapter~\ref{sec:zkproofs_2_dlog};
see Alg.~\ref{alg:2_dlog_equality} for full details.

\begin{algorithm}[t]
\caption{Discrete logarithm equality proof and validation}
\label{alg:2_dlog_equality}
\begin{algorithmic}[1]
\Require Group order $q$
\Procedure{DLEQ-Proof}{$x_{1}$,$y_{1}$,$x_{2}$,$y_{2}$,$\alpha$}
    \State $w \chooseRandom{} \Z_{q}^{*}$
    \Comment{Generate proof that $y_{1}=x_{1}^{\alpha}$ and
        $y_{2}=x_{2}^{\alpha}$ for secret $\alpha$}
    \State $t_{1} \algAssign{} x_{1}^{w}$
    \State $t_{2} \algAssign{} x_{2}^{w}$
    \State $c \algAssign{} \textsc{Hash}(x_{1},y_{1},x_{2},y_{2},t_{1},t_{2})$
    \State $r \algAssign{} w - c\alpha \mod q$
    \State $\pi \algAssign{} \parens{c,r}$
    \State \Return $\pi$
\EndProcedure
\State
\Procedure{DLEQ-Verify}{$x_{1}$,$y_{1}$,$x_{2}$,$y_{2}$,$\pi$}
    \Comment{Validate that $y_{1}=x_{1}^{\alpha}$ and $y_{2}=x_{2}^{\alpha}$}
    \State $t_{1}' \algAssign{} y_{1}^{c}x_{1}^{r}$
    \State $t_{2}' \algAssign{} y_{2}^{c}x_{2}^{r}$
    \State $c' \algAssign{}\textsc{Hash}(x_{1},y_{1},x_{2},y_{2},t_{1}',t_{2}')$
    \If{$c' = c$}
        \State \Return \textsf{True}
    \Else
        \State \Return \textsf{False}
    \EndIf
\EndProcedure
\end{algorithmic}
\end{algorithm}


Thus, $P_{j}$ can broadcast the Diffie-Hellman shared secret
$k_{ij}$ along with \gls{discrete log} proof $\pi(k_{ij})$
which may be verified by everyone.
At this point, all users may decrypt $\overline{\texttt{s}}_{i\to j}$
using $\Dec{}SS$:

\begin{equation}
    \hat{s}_{i\to j} \mathDef{} 
        \Dec{}SS\parens{k_{ij},\overline{\texttt{s}}_{i\to j}, j}.
\end{equation}

\noindent
Everyone can verify that

\begin{equation}
    g_{1}^{\hat{s}_{i\to j}} \ne F_{i}(j).
\end{equation}

\noindent
This proves that $P_{i}$ shared an invalid share with $P_{j}$;
this is a malicious action.

When this is implemented on \gls{ethereum},
$P_{i}$'s encrypted shares and commitments would need to be resubmitted.
After confirming their validity by rehashing the data,
all accusation logic may be performed by a \gls{smart contract}.
If we do not have equality Eq.~\eqref{eq:dkg_share_equality},
then $P_{j}$ just proved $P_{i}$ is malicious
because he submitted an invalid share;
if we have equality Eq.~\eqref{eq:dkg_share_equality},
then $P_{j}$ is malicious for submitting an invalid accusation.


\subsection{\KeyShareSubmission{}}
\label{ssec:secret_dkg_key_share}

If no one submitted any invalid shares,
then the next step is for each participant to submit his portion
of the master public key.

To do this, $P_{i}$ must submit $h_{1}^{s_{i}}$ along with proof

\begin{equation}
    \pi(h_{1}^{s_{i}}) \mathDef{}
        \textsc{DLEQ-Proof}(g_{1},g_{1}^{s_{i}},h_{1},h_{1}^{s_{i}},s_{i})
\end{equation}

\noindent
and $h_{2}^{s_{i}}$.
This is possible because $g_{1}^{s_{i}} = C_{i,0}$ is public knowledge.

A \gls{smart contract} call will confirm the validity of $h_{1}^{s_{i}}$
given $\pi(h_{1}^{s_{i}})$.
A \PairingCheck{} will confirm the validity of $h_{2}^{s_{i}}$
by requiring

\begin{equation}
    \PairingCheck{}(h_{1}^{s_{i}},h_{2}^{-1},h_{1},h_{2}^{s_{i}})
        = \texttt{true}.
\end{equation}

\noindent
The values $h_{1}^{s_{i}}$, $\pi(h_{1}^{s_{i}})$, and $h_{2}^{s_{i}}$
are then broadcast to all users.
The values $\braces{h_{1}^{s_{i}}}_{P_{i}\in\mathcal{P}}$ are stored
for the next phase.


\subsection{\MPKSubmission{}}
\label{ssec:secret_dkg_mpk}

After all key shares have been submitted, we can compute
the \emph{master public key} $\mpk{}$:

\begin{equation}
    \mpk{} \mathDef{} \prod_{P_{i}\in\mathcal{P}} h_{2}^{s_{i}}.
\end{equation}

\noindent
The corresponding \emph{master secret key} $\msk{}$ is defined
similarly:

\begin{equation}
    \msk{} \mathDef{} \sum_{P_{i}\in\mathcal{P}} s_{i}.
    \label{eq:dkg_msk_def}
\end{equation}

The $\mpk{}$ may be validated upon submission because we assume
that $\braces{h_{1}^{s_{i}}}_{P_{i}\in\mathcal{P}}$ are stored.
Then, we can form

\begin{equation}
    \mpk{}^{*} \mathDef{} \prod_{P_{i}\in\mathcal{P}} h_{1}^{s_{i}}
\end{equation}

\noindent
and then confirm

\begin{equation}
    \PairingCheck{}(\mpk{}^{*},h_{2}^{-1},h_{1},\mpk{}) = \texttt{true}.
\end{equation}

The reason we should not use $\prod_{P_{i}\in\mathcal{P}} g_{1}^{s_{i}}$
as the master public key is because the $g_{1}^{s_{i}}$ values
are public knowledge after the initial share.
Because of this, some participants could attempt to bias the distribution
of $\prod_{P_{i}\in\mathcal{P}} g_{1}^{s_{i}}$ based on the order
of submission;
see~\cite{gennaro1999secure,gennaro2002revisiting} for further discussion.


\subsection{\GPKSubmission{}}
\label{ssec:secret_dkg_gpk_submission}

At this point, we have constructed the $\mpk{}$ and distributed
the $\msk{}$ between all participants.
We now focus on how to compute valid group signatures.
Doing this requires signing under a special private key.
These partial signatures may then be combined in a deterministic
way to form a valid group signature.

Participant $P_{j}$ has a \emph{group secret key} $\gsk{}_{j}$:

\begin{equation}
    \gsk{}_{j} \mathDef{} \sum_{P_{i}\in\mathcal{P}} s_{i\to j}.
    \label{eq:dkg_gskj_def}
\end{equation}

\noindent
There is also the corresponding \emph{group public key} $\gpk{}_{j}$:

\begin{equation}
    \gpk{}_{j} \mathDef{} h_{2}^{\gsk{}_{j}}.
\end{equation}

\noindent
Based on the definition of $\gsk{}_{j}$, we have the following:

\begin{align}
    g_{1}^{\gsk{}_{j}} &= g_{1}^{\sum_{P_{i}\in\mathcal{P}} s_{i\to j}}
        \nonumber\\
    &= \prod_{P_{i}\in\mathcal{P}} g_{1}^{s_{i\to j}}
        \nonumber\\
    &= \prod_{P_{i}\in\mathcal{P}} g_{1}^{f_{i}(j)}
        \nonumber\\
    &= \prod_{P_{i}\in\mathcal{P}} F_{i}(j).
    \label{eq:dkg_gpkj_star_derivation}
\end{align}

\noindent
In this way, $g_{1}^{\gsk{}_{j}}$ may be constructed from public information.

We now define

\begin{equation}
    \gpk{}_{j}^{*} \mathDef{} \prod_{P_{i}\in\mathcal{P}} F_{i}(j).
    \label{eq:dkg_gpkj_star}
\end{equation}

\noindent
The definition of $\gpk{}_{j}^{*}$ in Eq.~\eqref{eq:dkg_gpkj_star}
along with the derivation in Eq.~\eqref{eq:dkg_gpkj_star_derivation}
shows us that

\begin{equation}
    \gpk{}_{j}^{*} = g_{1}^{\gsk{}_{j}}.
\end{equation}

\noindent
This implies that $\gpk{}_{j}$ is a valid group public key if

\begin{equation}
    \PairingCheck{}(\gpk{}_{j}^{*},h_{2}^{-1},g_{1},\gpk{}_{j})
        = \texttt{true}.
\end{equation}

\noindent
This fact gives us the ability to verify all $\gpk{}_{j}$ submissions.


\subsection{\GPKDispute{}}
\label{ssec:secret_dkg_gpk_dispute}

Given a submission $\gpk{}_{j}$, all parties can compute $\gpk{}_{j}^{*}$
from Eq.~\eqref{eq:dkg_gpkj_star} and verify

\begin{equation}
    \PairingCheck{}(\gpk{}_{j}^{*},h_{2}^{-1},g_{1},\gpk{}_{j})
        = \texttt{true}.
\end{equation}

\noindent
This logic may also be performed by a \gls{smart contract},
although care must be taken to ensure efficient computation.


\subsection{\Completion{}}
\label{ssec:secret_dkg_completion}

After allowing sufficient time for accusations of malicious
$\gpk{}_{j}$ submissions,
the \gls{distributed key generation} process is complete.

\section{Threshold Signatures}
\label{sec:ss_threshold}

\subsection{Deriving the Master Secret Key from the Group Secret Keys}

Now that we have constructed the master public key,
we will focus on computing group signatures.
We let $\mathcal{R}\subseteq\mathcal{P}$ be subset such that
$\abs{\mathcal{R}} = t+1$.

We recall the definition of $\gsk{}_{j}$ in Eq.~\eqref{eq:dkg_gskj_def}:

\begin{equation}
    \gsk{}_{j} = \sum_{P_{i}\in\mathcal{P}} s_{i\to j}.
\end{equation}

\noindent
We remember that $s_{i\to j}$ is the secret share
from $P_{i}$ to $P_{j}$.
This secret share $s_{i\to j}$ is an evaluation of $P_{i}$'s private polynomial.
Written another way, we have

\begin{equation}
    \gsk{}_{j} = \sum_{P_{i}\in\mathcal{P}} f_{i}(j).
\end{equation}

To make valid group signatures, we need to combine signatures
signed by the $\gsk{}_{j}$ keys.
In order to understand how this works, we will show how we can
derive the master secret key from the group secret keys.

We begin by recalling the definition of $\msk{}$
from Eq.~\eqref{eq:dkg_msk_def}:

\begin{equation}
    \msk{} = \sum_{P_{i}\in\mathcal{P}} s_{i}.
\end{equation}

\noindent
From Eq.~\eqref{eq:dkg_private_poly} and the surrounding discussion,
we know that

\begin{equation}
    s_{i} = f_{i}(0).
\end{equation}

\noindent
This implies that $\msk{}$ can be rewritten in terms of
evaluations of a private polynomials:

\begin{equation}
    \msk{} = \sum_{P_{i}\in\mathcal{P}} f_{i}(0).
\end{equation}

As we have stated before, during the share distribution phase
$P_{i}$ performed Shamir's Secret Sharing.
Given $\mathcal{R}$ as before, this means that the members of $\mathcal{R}$
can work together to compute

\begin{align}
    f_{i}(0) &= \sum_{P_{j}\in\mathcal{R}} f_{i}(j)R_{j} \nonumber\\
    R_{j} &= \prod_{\substack{P_{k}\in\mathcal{R} \\ k\ne j}}
        \frac{k}{k-j}.
\end{align}

\noindent
This is coming from Chapter~\ref{sec:ss_shamir};
\emph{note the indices are different.}
Rewriting this, we have

\begin{align}
    s_{i} &= \sum_{P_{j}\in\mathcal{R}} s_{i\to j}R_{j} \nonumber\\
    R_{j} &= \prod_{\substack{P_{k}\in\mathcal{R} \\ k\ne j}}
        \frac{k}{k-j}.
\end{align}

\noindent
If we sum over $P_{i}\in\mathcal{P}$, then we have

\begin{align}
    \msk{} &= \sum_{P_{i}\in\mathcal{P}} s_{i} \nonumber\\
        &= \sum_{P_{i}\in\mathcal{P}}
            \brackets{\sum_{P_{j}\in\mathcal{R}} s_{i\to j}R_{j}}
                \nonumber\\
        &= \sum_{P_{j}\in\mathcal{R}}
            \brackets{\sum_{P_{i}\in\mathcal{P}} s_{i\to j}}R_{j}
                \nonumber\\
        &= \sum_{P_{j}\in\mathcal{R}} \gsk{}_{j}R_{j}.
\end{align}

\noindent
We can use this equality to construct valid group signatures.

In more technical mathematical language, we are performing
\gls{lagrange interpolation} over a \gls{finite field}.

\subsection{Constructing Valid Group Signatures}

We can now use the work from the previous section to show
how we are able to combine partial signatures into a valid group signature.

The group wants to sign the message $m\in\braces{0,1}^{*}$.
We suppose that $\mathcal{R}\subseteq\mathcal{P}$ be as before;
that is, we have $\abs{\mathcal{R}} = t+1$.
We now suppose that they have the individual signatures

\begin{equation}
    \sigma_{j} \mathDef{} \brackets{H(m)}^{\gsk{}_{j}}.
\end{equation}

\noindent
We define the constants

\begin{equation}
    R_{j} = \prod_{\substack{P_{k}\in\mathcal{R} \\ k\ne j}}
        \frac{k}{k-j}.
\end{equation}

\noindent
We set

\begin{equation}
    \sigma \mathDef{} \prod_{P_{j}\in\mathcal{R}} \sigma_{j}^{R_{j}}.
\end{equation}

This is a valid group signature of $m$ under the master public key $\mpk{}$.
To see this, we note

\begin{align}
    e\parens{\sigma, h_{2}}
        &= e\parens{\prod_{P_{j}\in\mathcal{R}}\sigma^{R_{j}}, h_{2}}
            \nonumber\\
        &= \prod_{P_{j}\in\mathcal{R}} e\parens{\sigma^{R_{j}}, h_{2}}
            \nonumber\\
        &= \prod_{P_{j}\in\mathcal{R}} e\parens{H(m)^{\gsk{}_{j}R_{j}}, h_{2}}
            \nonumber\\
        &= \prod_{P_{j}\in\mathcal{R}} e\parens{H(m), h_{2}^{\gsk{}_{j}R_{j}}}
            \nonumber\\
        &= e\parens{H(m), \prod_{P_{j}\in\mathcal{R}} h_{2}^{\gsk{}_{j}R_{j}}}
            \nonumber\\
        &= e\parens{H(m), h_{2}^{\sum_{P_{j}\in\mathcal{R}} \gsk{}_{j}R_{j}}}
            \nonumber\\
        &= e\parens{H(m), h_{2}^{\msk{}}}
            \nonumber\\
        &= e\parens{H(m), \mpk{}}.
\end{align}

\noindent
Thus, we have a valid group signature from combining individual
signatures in a deterministic manner.
The constants only depend on which partial signatures are used
in the construction.
Additionally, they individually may be validated before combining them
because the $\gpk{}_{j}$ are public knowledge.
It may be verified by ensuring

\begin{equation}
    \PairingCheck{}\parens{\sigma, h_{2}^{-1}, H(m), \mpk{}} = \texttt{true}.
\end{equation}

In more technical mathematical language, we are performing
\gls{lagrange interpolation} over a group parameterized by elements
of a \gls{finite field}.

\section{Distributed Key Generation Example}
\label{sec:ss_dkg_examples}

We now walk through a trimmed down example of the
\gls{distributed key generation} protocol
in order to see how it would work out in a smaller setting.
In particular, we walk through the construction of the master secret key.

\begin{example}[Alice, Bob, Charlie, and Dave distribute a key]
\exampleCodeReference{examples/secret\_sharing/dkg.py}

As before, Alice, Bob, Charlie, and Dave have started a business.
This time, though, they wish to perform the \gls{distributed key generation}
protocol in order to construct their own master secret key.
They want to require that 3 members work together to sign a message
for the entire group.
Thus, they want a $\parens{2,4}$ protocol.
As before, we will be working in $\F_{7919}$.

\begin{itemize}
\item Alice has the private polynomial

\begin{equation}
    f_{1}(x) = 5853 + 2695x + 4447x^{2}
\end{equation}

and secret shares

\begin{align}
    s_{1\to 1} &= 5076 \nonumber\\
    s_{1\to 2} &= 5274 \nonumber\\
    s_{1\to 3} &= 6447 \nonumber\\
    s_{1\to 4} &=  676.
\end{align}

\item Bob has the private polynomial

\begin{equation}
    f_{2}(x) = 87 + 1150x + 7782x^{2}
\end{equation}

and secret shares

\begin{align}
    s_{2\to 1} &= 1100 \nonumber\\
    s_{2\to 2} &= 1839 \nonumber\\
    s_{2\to 3} &= 2304 \nonumber\\
    s_{2\to 4} &= 2495.
\end{align}

\item Charlie has the private polynomial

\begin{equation}
    f_{3}(x) = 6991 + 3631x + 5061x^{2}
\end{equation}

and secret shares

\begin{align}
    s_{3\to 1} &= 7764 \nonumber\\
    s_{3\to 2} &= 2821 \nonumber\\
    s_{3\to 3} &=   81 \nonumber\\
    s_{3\to 4} &= 7463.
\end{align}

\item Dave has the private polynomial

\begin{equation}
    f_{4}(x) = 5304 + 2184x + 2803x^{2}
\end{equation}

and secret shares

\begin{align}
    s_{4\to 1} &= 2372 \nonumber\\
    s_{4\to 2} &= 5046 \nonumber\\
    s_{4\to 3} &= 5407 \nonumber\\
    s_{4\to 4} &= 3455.
\end{align}
\end{itemize}

The master secret key is the sum of the constant terms:

\begin{align}
    \msk{} &= 5853 + 87 + 6991 + 5304 \mod 7919 \nonumber\\
        &= 2397.
\end{align}

We now look at computing the $\gsk{}_{j}$'s:

\begin{itemize}
\item Alice has received the secret shares

\begin{align}
    s_{1\to 1} &= 5076 \nonumber\\
    s_{2\to 1} &= 1100 \nonumber\\
    s_{3\to 1} &= 7764 \nonumber\\
    s_{4\to 1} &= 2372.
\end{align}

She then determines

\begin{align}
    \gsk{}_{1} &= s_{1\to 1} + s_{2\to 1} + s_{3\to 1} + s_{4\to 1} \mod q
        \nonumber\\
    &= 5076 + 1100 + 7764 + 2372 \mod 7919
        \nonumber\\
    &= 474.
\end{align}

\item Bob has received the secret shares

\begin{align}
    s_{1\to 2} &= 5274 \nonumber\\
    s_{2\to 2} &= 1839 \nonumber\\
    s_{3\to 2} &= 2821 \nonumber\\
    s_{4\to 2} &= 5046.
\end{align}

He then determines

\begin{align}
    \gsk{}_{2} &= s_{1\to 2} + s_{2\to 2} + s_{3\to 2} + s_{4\to 2} \mod q
        \nonumber\\
    &= 5274 + 1839 + 2821 + 5046 \mod 7919
        \nonumber\\
    &= 7061.
\end{align}

\item Charlie has received the secret shares

\begin{align}
    s_{1\to 3} &= 6447 \nonumber\\
    s_{2\to 3} &= 2304 \nonumber\\
    s_{3\to 3} &=   81 \nonumber\\
    s_{4\to 3} &= 5407.
\end{align}

He then determines

\begin{align}
    \gsk{}_{3} &= s_{1\to 3} + s_{2\to 3} + s_{3\to 3} + s_{4\to 3} \mod q
        \nonumber\\
    &= 6447 + 2304 + 81 + 5407 \mod 7919
        \nonumber\\
    &= 6320.
\end{align}

\item Dave has received the secret shares

\begin{align}
    s_{1\to 4} &=  676 \nonumber\\
    s_{2\to 4} &= 2495 \nonumber\\
    s_{3\to 4} &= 7463 \nonumber\\
    s_{4\to 4} &= 3455.
\end{align}

He then determines

\begin{align}
    \gsk{}_{4} &= s_{1\to 4} + s_{2\to 4} + s_{3\to 4} + s_{4\to 4} \mod q
        \nonumber\\
    &= 676 + 2495 + 7463 + 3455 \mod 7919
        \nonumber\\
    &= 6170.
\end{align}
\end{itemize}

We previously calculated the $R_{j}$ coefficients
in Example~\ref{example:dkg_shamir_4}, so we will reuse those here.

\begin{itemize}
\item Alice, Bob, and Dave can combine their group secret keys
    and coefficients to find

\begin{align}
    \gsk{}_{1}R_{1} + \gsk{}_{2}R_{2} + \gsk{}_{4}R_{4} \mod q
            &= 474\cdot5282 + 7061\cdot7917 + 6170\cdot2640 \mod 7919
            \nonumber\\
        &= 2397.
\end{align}

This is the master secret key.

\item Alice, Charlie, and Dave can combine their group secret keys
    and coefficients to find

\begin{align}
    \gsk{}_{1}R_{1} + \gsk{}_{3}R_{3} + \gsk{}_{4}R_{4} \mod q
            &= 474\cdot2 + 6320\cdot7917 + 6170\cdot1 \mod 7919
            \nonumber\\
        &= 2397.
\end{align}

This is the master secret key.
\end{itemize}
\end{example}


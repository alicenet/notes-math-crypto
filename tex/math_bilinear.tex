\section{Bilinear Pairings}
\label{sec:math_pairings}

\subsection{Why do we care about Bilinear Pairings?}

We care about \glspl{bilinear} because they give us a lot of nice
mathematical features that we can use.
In particular, \glspl{bilinear} give us
\emph{short digital signatures}.
These started with BLS signatures based on the Weil
pairing~\cite{BLSSignatures},
a particular type of \gls{bilinear}.

\Glspl{bilinear} also allow for threshold group signatures.
These can be formed as part of a \gls{distributed key generation}
protocol and are discussed in Chapter~\ref{chap:secret_sharing}.

\subsection{Formal Definition}

A \gls{bilinear} is a special type of \gls{function}:

\begin{defn}[Bilinear Pairing]
We let $G_{1}$, $G_{2}$, and $G_{T}$ be \glspl{finite group} with
$\abs{G_{1}} = \abs{G_{2}} = \abs{G_{T}}$.
We say $e:G_{1}\times G_{2}\to G_{T}$ is a \emph{\gls{bilinear}}
if for all $h_{1}\in G_{1}$, $h_{2}\in G_{2}$, and $a,b\in\Z$
we have

\begin{equation}
    e\parens{h_{1}^{a},h_{2}^{b}} = \brackets{e\parens{h_{1},h_{2}}}^{ab}.
\end{equation}
\end{defn}

\subsection{Discussion}

It is straightforward to \emph{use} the previous definition.
The challenge is \emph{finding} \glspl{group} with which to build such a
\gls{bilinear}.
This is nontrivial.

At this point, all \glspl{bilinear} arise from \glspl{elliptic curve}.
The exact construction is complex and we do not discuss it here.
We will use \glspl{bilinear} in Chapter~\ref{chap:pairing}
when we discuss \gls{pairingcrypto}.
Having a solid understanding of the mathematics behind pairings
and how they are constructed from \glspl{elliptic curve}
requires advanced knowledge of \glspl{elliptic curve} as discussed
in~\cite{AEC}.

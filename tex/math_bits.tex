\section{Bit Operations}

Here we include a brief discussion of bit operations
for the sake of completeness.

\subsection{Definitions}

Each bit is either $\texttt{0}$ or $\texttt{1}$.

\paragraph{NOT}
\emph{NOT} reverses a bit:

\begin{align}
    \lnot\texttt{0} &= \texttt{1} \nonumber\\
    \lnot\texttt{1} &= \texttt{0}.
\end{align}

\paragraph{AND}
\emph{AND} operates on two bits and returns $\texttt{1}$ if both bits are
$\texttt{1}$ and returns $\texttt{0}$ otherwise:

\begin{align}
    \texttt{0}\land\texttt{0} &= \texttt{0} \nonumber\\
    \texttt{1}\land\texttt{0} &= \texttt{0} \nonumber\\
    \texttt{0}\land\texttt{1} &= \texttt{0} \nonumber\\
    \texttt{1}\land\texttt{1} &= \texttt{1}.
\end{align}

\paragraph{OR}
\emph{OR} operates on two bits and returns $\texttt{1}$ if at least one bit
is $\texttt{1}$ and returns $\texttt{0}$ otherwise:

\begin{align}
    \texttt{0}\lor\texttt{0} &= \texttt{0} \nonumber\\
    \texttt{1}\lor\texttt{0} &= \texttt{1} \nonumber\\
    \texttt{0}\lor\texttt{1} &= \texttt{1} \nonumber\\
    \texttt{1}\lor\texttt{1} &= \texttt{1}.
\end{align}

\paragraph{XOR}
\emph{XOR} operates on two bits and returns $\texttt{1}$
if the two bits are different and returns $\texttt{0}$ otherwise:

\begin{align}
    \texttt{0}\oplus\texttt{0} &= \texttt{0} \nonumber\\
    \texttt{1}\oplus\texttt{0} &= \texttt{1} \nonumber\\
    \texttt{0}\oplus\texttt{1} &= \texttt{1} \nonumber\\
    \texttt{1}\oplus\texttt{1} &= \texttt{0}.
\end{align}

\noindent
The XOR operation is important because of the following property:

\begin{equation}
    \parens{\texttt{x} \oplus \texttt{y}} \oplus \texttt{y} = \texttt{x}.
\end{equation}

\noindent
for all bits $\texttt{x}$ and $\texttt{y}$.
We also have

\begin{equation}
    \texttt{x} \oplus \texttt{0} = \texttt{x}.
\end{equation}

\subsection{Bit Operations on Integers}

Using an unsigned integer's binary representation,
these bit operations can be extended to unsigned integers;
the extension is straightforward and we only give examples.


\begin{example}[Bitwise Operations]
We show some examples of bit operations using 8-bit unsigned integers:

\begin{itemize}
\item Bitwise NOT:

\begin{align}
    \lnot\texttt{11110010} &=
         \texttt{00001101} \nonumber\\
    \lnot\texttt{11010011} &=
         \texttt{00101100} \nonumber\\
    \lnot\texttt{00010110} &=
         \texttt{11101001} \nonumber\\
    \lnot\texttt{10111011} &=
         \texttt{01000100}.
\end{align}

\item Bitwise AND:

\begin{align}
    \texttt{11110010} \land
    \texttt{11010011} &=
    \texttt{11010010} \nonumber\\
    \texttt{00010110} \land
    \texttt{10111011} &=
    \texttt{00010010}.
\end{align}

\item Bitwise OR:

\begin{align}
    \texttt{11110010} \lor
    \texttt{11010011} &=
    \texttt{11110011} \nonumber\\
    \texttt{00010110} \lor
    \texttt{10111011} &=
    \texttt{10111111}.
\end{align}

\item Bitwise XOR:

\begin{align}
    \texttt{11110010} \oplus
    \texttt{11010011} &=
    \texttt{00100001} \nonumber\\
    \texttt{00010110} \oplus
    \texttt{10111011} &=
    \texttt{10101101}.
\end{align}

In general, we also have

\begin{equation}
    \parens{x \oplus y} \oplus y = x
\end{equation}

\noindent
and

\begin{equation}
    x \oplus 0 = x
\end{equation}

\noindent
for all unsigned integers $x$ and $y$.
\end{itemize}
\end{example}

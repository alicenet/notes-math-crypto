\section{The Real Numbers}

When listing off the standard symbols in Chapter~\ref{ssec:standard_math_sets},
we defined the real numbers $\R$ with

\begin{equation}
    \R \mathDef{} \braces{\text{the set of all real numbers}},
\end{equation}

\noindent
while the other sets were more specific.
That is, we never took time to \emph{actually define} the real numbers.

The reason for this is that it is \emph{not possible} to define
the real numbers without an advanced mathematical discussion.
This is because the property which distinguishes the real numbers $\R$ from
the rational numbers $\Q$ is the \emph{least-upper-bound property}:
every nonempty set of real numbers which is bounded above has a
least upper bound.
The rationals do not have the least-upper-bound property;
as a result, there are ``gaps'' in the rationals.
Some of the numbers ``missing'' from the rationals are $\sqrt{2}$, $\pi$,
and $e$; these numbers are \emph{irrational}.
The reals \emph{do} have the least upper bound property,
so there are no ``gaps'';
it contains the rational and irrational numbers.

A rigorous discussion of the properties of real numbers is required
before attempting to prove all the results of calculus.
All of this is done in a mathematics course usually called
\emph{Real Analysis}.
One standard reference is~\cite{BabyRudin},
although~\cite{LeblBasicAnalysisI} may provide a better introduction;
also see~\cite{AIMTextbooks}.
